\documentclass[10pt]{amsart}
\usepackage{amsmath,amsthm,amssymb,amsfonts}
\usepackage[mathscr]{euscript}
\usepackage{tikz}
\usepackage{tikz-cd}
\usepackage{enumerate}
\usepackage{enumitem}
\usepackage{mathtools}
\usepackage[colorlinks=true, linkcolor=red, citecolor = blue]{hyperref}
\usepackage[margin=2.5cm]{geometry}
\setlength{\marginparwidth}{2cm}

\usepackage[nameinlink,capitalise,noabbrev]{cleveref}

\usepackage[textwidth=2cm, textsize=small, colorinlistoftodos]{todonotes}

\newcommand{\lA}{\mathcal{A}}
\newcommand{\bA}{\mathbb{A}}
\newcommand{\lB}{\mathcal{B}}
\newcommand{\bB}{\mathbb{B}}
\newcommand{\lC}{\mathcal{C}}
\newcommand{\rC}{\mathscr{C}}
\newcommand{\bC}{\mathbb{C}}
\newcommand{\lD}{\mathcal{D}}
\newcommand{\bE}{\mathbb{E}}
\newcommand{\lE}{\mathcal{E}}
\newcommand{\lF}{\mathcal{F}}
\newcommand{\lG}{\mathcal{G}}
\newcommand{\lH}{\mathcal{H}}
\newcommand{\mH}{\mathrm{H}}
\newcommand{\lI}{\mathcal{I}}
\newcommand{\lJ}{\mathcal{J}}
\newcommand{\lK}{\mathcal{K}}
\newcommand{\lL}{\mathcal{L}}
\newcommand{\lM}{\mathcal{M}}
\newcommand{\bN}{\mathbb{N}}
\newcommand{\mN}{\mathrm{N}}
\newcommand{\lN}{\mathcal{N}}
\newcommand{\fO}{\mathbf{O}}
\newcommand{\rO}{\mathscr{O}}
\newcommand{\lO}{\mathcal{O}}
\newcommand{\lP}{\mathcal{P}}
\newcommand{\bP}{\mathbb{P}}
\newcommand{\bQ}{\mathbb{Q}}
\newcommand{\bR}{\mathbb{R}}
\newcommand{\lR}{\mathcal{R}}
\newcommand{\lS}{\mathcal{S}}
\newcommand{\bS}{\mathbb{S}}
\newcommand{\lT}{\mathcal{T}}
\newcommand{\lU}{\mathcal{U}}
\newcommand{\lV}{\mathcal{V}}
\newcommand{\lX}{\mathcal{X}}
\newcommand{\lY}{\mathcal{Y}}
\newcommand{\bZ}{\mathbb{Z}}

\newcommand{\op}{\mathrm{op}} % opposite
\newcommand{\ch}{\mathrm{ch}}
\newcommand{\Hom}{\lH\mathrm{om}} % enriched hom-spaces, hom-spectrum
\renewcommand{\hom}{\mH\mathrm{om}} % hom-set, hom-space
\newcommand{\Ho}{\mathrm{Ho}} % homotopy category
\newcommand{\set}{\lS\mathrm{et}} % category of sets
\newcommand{\Mon}{\lM\mathrm{on}} % category of monoid objects
\newcommand{\CMon}{\lC\Mon} % category of commutative monoid objects
\newcommand{\CGrp}{\lC\lG\mathrm{on}} % category of commutative group objects
\newcommand{\Bdl}{\lB\mathrm{dl}} % bundles
\newcommand{\Sp}{\lS\mathrm{p}} % category of spectra
\newcommand{\Ch}{\lC\mathrm{h}} % category of chain complexes
\newcommand{\cat}{\lC\mathrm{at}} % category of categories
\newcommand{\scat}{\mathrm{s}\cat} % category of simplicial categories
\newcommand{\sset}{\mathrm{s}\set} % category of simplicial sets
\newcommand{\Line}{\lL\mathrm{ine}} % category of lines
\newcommand{\LineBdl}{\Line\Bdl} % category of line bundles
\newcommand{\Fun}{\lF\mathrm{un}} % category of functors
\newcommand{\Top}{\lT\op} % category of topological spaces 
\newcommand{\Grpd}{\lG\mathrm{rpd}} % category of groupoids
\newcommand{\Grp}{\lG\mathrm{rp}} % category of groups
\newcommand{\Euc}{\lE\mathrm{uc}} % site of Euclidean manifolds
\newcommand{\Mfd}{\lM\mathrm{fd}} % site of smooth manifolds
\newcommand{\Kan}{\lK\mathrm{an}} % category of Kan complexes
\newcommand{\Vect}{\lV\mathrm{ect}} % category of vector spaces
\newcommand{\Mod}{\lM\mathrm{od}} % category of modules
\newcommand{\Ab}{\lA\mathrm{b}} % category of abelian groups
\newcommand{\Shv}{\lS\mathrm{hv}} % category of sheaves
\newcommand{\Open}{\lO\mathrm{pen}} % category of open subspaces
\newcommand{\Pic}{\lP\mathrm{ic}} % Picard category
\newcommand{\PT}{\lP\lT} % Pontryagin-Thom
\newcommand{\Fred}{\lF\mathrm{red}} % Fredholm
\newcommand{\fl}{\mathrm{fl}} % flat
\newcommand{\GrbBdl}{\lG\mathrm{rb}\Bdl} % category of bundle gerbes
\newcommand{\cl}{\mathrm{cl}} % closed forms
\newcommand{\BDelta}{\mathbf{\Delta}} % simplex category
\newcommand{\Sing}{\mathrm{Sing}} % singular simplicial set
\newcommand{\spin}{\lS\mathrm{pin}} % spin group 


\DeclareMathOperator*\colim{colim} % colimit

%% N.R. notes
\newcommand{\nrnote}[1]{\todo[color=green!40,linecolor=green!40!black,size=\tiny]{#1}}
\newcommand{\nrmpar}[1]{\todo[noline,color=green!40,linecolor=green!40!black,
  size=\tiny]{#1}}
\newcommand{\nrnoteil}[1]{\ \todo[inline,color=green!40,linecolor=green!40!black,size=\normalsize]{#1}}

\newtheorem{theorem}[equation]{Theorem}
\newtheorem{lemma}[equation]{Lemma}
\newtheorem{proposition}[equation]{Proposition}
\newtheorem{corollary}[equation]{Corollary}
% \newtheorem{statement}[section]{Statement}

\theoremstyle{definition}
\newtheorem{definition}[equation]{Definition}
\newtheorem{example}[equation]{Example}
% \newtheorem{attone}[equation]{Attention}

\theoremstyle{remark}
\newtheorem{remark}[equation]{Remark}
% \newtheorem{intone}[equation]{Intuition}
\newtheorem{notation}[equation]{Notation}
% \newtheorem{queone}[equation]{Question}
% \newtheorem{conjone}[equation]{Conjecture}
\newtheorem{warning}[equation]{Warning}

\numberwithin{equation}{section}

\title{Differential Cohomology Seminar 11}
\date{14.01.2026}
\author{Talk by Matthias Frerichs}

\begin{document}
\maketitle

Today we look at differential cohomology in the cohesive setting and in particular the generalization to non-abelian differential cohomology in the sense of Schreiber \cite{schreiber2013cohesive,fiorenzasatischreiber2024charactermap}.

We saw previously that differential cohomology lifts order cohomology theories represented by spectra. However, spectra can only classify phenomena that are inherently stable, whereas certain applications require unstable structures (see \cref{rem:motivation} for a more detailed discussion).

We hence discuss a ``non-abelian'' generalization of differential cohomology, that can incorporate such unstable phenomena. Concretely, instead of working with $E_\infty$-groups (which correspond to connective spectra), we want to work with $E_2$-groups, meaning homotopy commutative group objects in spaces.

\section{Non-Abelian Cohomology}
Let us first define non-abelian cohomology in the classical setting. HEre the idea is to generalize from spectra to spaces.

\begin{definition}
	Let $X, A$ be $\infty$-groupoids. The \emph{non-abelian cohomology} of $X$ with coefficients in $A$ is defined as
	\[	H^0(X; A) := \pi_0 \mathrm{Map}(X, A). \]
\end{definition}

We can generalize to higher cohomology if $A$ admits a delooping.

\begin{definition}
	Let $X$ be an $\infty$-groupoid and $A$ an	$E_2$-group with delooping $\mathrm{B}A$. The \emph{non-abelian cohomology} of $X$ with coefficients in $A$ is defined as
	\[	H^1(X; A) := \pi_0 \mathrm{Map}(X, \mathrm{B} A). \]
\end{definition}	

Of course, if $A$	admits further deloopings we can continue and obtain higher non-abelian cohomology groups.

\begin{definition}
	Let $X$ be an $\infty$-groupoid and $A$ admits deloopings $\mathrm{B}^n A$. The \emph{non-abelian cohomology} of $X$ with coefficients in $A$ is defined as
	\[	H^n(X; A) := \pi_0 \mathrm{Map}(X, \mathrm{B}^n A). \]
\end{definition}	

If $A$ admits infinite deloopings, then $A$ is precisely an $E_\infty$-group and hence corresponds to a connective spectrum. In that case we recover the classical definition.


\begin{remark}
	If $A$ is a spectrum, then we have $\mathrm{B}^n A = \Omega^{\infty - n} A$ and we recover the ordinary cohomology groups
	\[	H^n(X; A) = \pi_0 \mathrm{Map}(X, \Omega^{\infty - n} A) \cong [X, \Omega^{\infty - n} A] \cong A^n(X). \]
\end{remark}

Let us motivate this story.

\begin{remark} \label{rem:motivation}
	\begin{enumerate}
		\item The string bundle of a manifold $X$ is classified by homotopy classes of maps $X \to B \mathrm{String}$, which coincides with $H^1(X; \mathrm{String})$, originally defined by Giraud via non-abelian Cech cohomology \cite{giraud1971cohnonabelian}. Here $\mathrm{String}$ is an $E_1$-group (it is not abelian), so $B\mathrm{String}$ is just $E_2$.
		\item Another motivation comes from electro magnetics. Let $P$ be a $U(1)$-principal fiber bundle on $X$, then $[X,BU(1)] \cong [X, K(\bZ,2)] \cong [X,\bC\bP^\infty]$. We know that $\bC\bP^\infty$ is a colimit of $\bC\bP^n$, none of the intermediate stages are $\infty$-loop spaces. Generalizing the definition permits looking at $H(X,\bC\bP^n)$, which should have relevance in physics.
	\end{enumerate}
\end{remark}
	
\section{Differential Non-Abelian Cohomology}
We now want to generalize this approach to differential cohomology. Concretely, we want to look at differential refinements of non-abelian cohomology. We will do so by looking at differential refinements of intrinsic cohomology of a cohesive $\infty$-topos.

Naively, given an $\infty$-topos $\lG$ with cohesive structure and $X,A$ objects in $\lG$, we would like to define the non-abelian cohomology as 
\[ H^n(X; A) := \pi_0 \mathrm{Map}_{\lG}(X, \mathrm{B}^n A). \]
assuming $A$ admits a delooping $\mathrm{B}^n A$ in $\lG$.

Our goal is now to extract the fracture square, analogous to \cite{bunkenikolausvoelkl2016diffcoh}. 

\[
\begin{tikzcd}[row sep=1cm, column sep=2cm]
 \hat{E} \arrow[r] \arrow[d] & Z \arrow[d, "\text{homotopification}"] \\ 
	E \arrow[r, "\text{characteristic map}"] & \lH(Z)
\end{tikzcd} 
\]

How can we do this in the non-abelian case? We first review relevant definitions.

\begin{definition}
 A cohesive	$\infty$-topos $\lG$ is an $\infty$-topos together with an adjoint quadruple
	\[	\Pi \dashv \mathrm{Disc} \dashv \Gamma \dashv \mathrm{Codisc},	\]
	where $\Gamma\colon \lG \to \lS$ is the global sections functor,	such that
	\begin{enumerate}
		\item $\mathrm{Disc}$ and $\mathrm{Codisc}$ are fully faithful,
		\item $\Pi$ preserves finite products.
	\end{enumerate}
\end{definition}

\begin{example}
	Let $\Mfd$ be the category of smooth manifolds with the usual open cover topology. Then the $\infty$-topos of sheaves of $\infty$-groupoids on $\Mfd$ is cohesive, where
	\begin{itemize}
		\item $\Pi$ is given by taking the fundamental $\infty$-groupoid,
		\item $\mathrm{Disc}$ is given by regarding an $\infty$-groupoid as a constant sheaf,
		\item $\Gamma$ is given by taking global sections,
		\item $\mathrm{Codisc}$ is given by regarding an $\infty$-groupoid as a codiscrete sheaf.
	\end{itemize}
\end{example}

We can now define an analogue of the de Rham complex in the setting of a cohesive $\infty$-topos.

\begin{definition}
	Let $\lG$ be a cohesive $\infty$-topos and $A	\in \lG$. The \emph{de Rham complex} of $A$ is defined as
	\[
	\begin{tikzcd}
	 A \arrow[d] \arrow[r] & * \arrow[d] \\
		\mathrm{Disc} \circ \Pi(A) \arrow[r] &  \Pi_{dR}A
	\end{tikzcd}	
	\]
	and for an object $A$ the \emph{coefficient object} of $A$
	\[
	\begin{tikzcd}
		\flat_{dR}A \arrow[d] \arrow[r]	& * \arrow[d] \\
		\mathrm{Disc} \circ \Gamma(A) \arrow[r] & A 
	\end{tikzcd}
	\]
\end{definition}

Note here, there is an adjunction 
\[ 
\begin{tikzcd}
	\lG \arrow[r, shift left=2, "\Pi_{dR}", "\bot"'] & \lG \arrow[l, shift left=2, "\flat_{dR}"]
\end{tikzcd}
\]

With that at hand, we can formulate the following definition.

\begin{definition}
	For objects $X,A$ in a cohesive $\infty$-topos $\lG$ the \emph{differential non-abelian cohomology} of $X$ with coefficients in $A$ is defined as the set
	\[	H_{dR}(X, A) := H(\Pi_{dR}X,A) \simeq H(X, \flat_{dR} A). \]
\end{definition}

Notice this abstract definition recovers the classical case. 

\begin{proposition} \label{prop:cohomologycomputation}
 Let $X$ be a sheaf of manifolds. Then 
	\[ \mathrm{Map}(X,\flat_{dR}	\mathbf{B}^n U(1)) \simeq 
	\begin{cases}
	 H^{n}_{dR} & \text{for } n \geq 2, \\
		\Omega^1_{cl} & \text{for } n = 1, \\
		0 & \text{for } n = 0.
	\end{cases} 	
	\]
\end{proposition}

\begin{remark}
	Recall that in our original approach to differential cohomology we often used the pure sheaf as $\Omega^{\geq n}$. So, this definition does diverge slightly. However, it appears this difference does not play an adverse role.
\end{remark}

We will not prove this result, however, we	sketch the idea of the proof.
\begin{proof}[Idea]
	 Fundamentally the result follows from the following equivalence:
	\[\mathrm{Map}(X, \flat_{dR} B^n	U(1)) \simeq \mathrm{Map}(C(\{U_i\}), \Phi(\Omega^1(-) \to \Omega^2(-) \to ... \Omega^n_{cl})), \]
	where $C(\{U_i\})$ is the Cech nerve of a good open cover of $X$ and $\Phi$ is the Dold-Kan correspondence.

	This observation fundamentally relies on the computation that the sheaf associated to the chain complex
\[	\Phi(\Omega^1(-) \to \Omega^2(-) \to ...	\to \Omega^n_{cl}). \] 
 is given by $\flat_{dR} B^n	U(1)$.

	This allows unwinding the definition of cohomology in \cref{prop:cohomologycomputation} via $\flat_{dR}$ into a computation with differential forms, which recovers the classical de Rham cohomology groups.
\end{proof}

% Fundamentally this result relies on the computation, which shall remain without proof, of $\flat_{dR} B^n	U(1)$ as the sheaf associated to the chain complex
% \[	\Phi(\Omega^1(-) \to \Omega^2(-) \to ...	\to \Omega^n_{cl}). \]

We can now use this generalized definition to define curvature maps.

\begin{definition} \label{def:curvaturemap}
	Let $A$ be an object in $\lG$, such that $B^{n+1} A$ exists. The \emph{curvature map} is	defined as the map pullbacks
	\[ 
	\begin{tikzcd}
		B^nA \arrow[r] \arrow[d, "\mathrm{curv}"] & * \arrow[d]  \\
		\flat{dR} B^{n+1} A \arrow[r]  \arrow[d] & \mathrm{Disc} \circ \Gamma  B^{n+1} A \arrow[d] \\ 
		* \arrow[r] & B^{n+1} 
	\end{tikzcd}
	\]
\end{definition}

With this curvature map we can finally state the fracture square.
\begin{theorem}
	Let $A$ admit deloopings $B^n A$. Then we have a pullback square
	\[ 
		\begin{tikzcd}
			H_{diff}(X,B^nA) \arrow[r] \arrow[d] & H_{dR}(X,A) \arrow[d]  \\ 
			\mathrm{Map}_{sG}(X,B^nA) \arrow[r, "\mathrm{curv}_*"] & \mathrm{Map}_{sG}(X, \flat_{dR} B^{n+1} A)
		\end{tikzcd},
		\]
		where the bottom map is induced by the curvature map from \cref{def:curvaturemap}.
	\end{theorem}

The claim, which shall remain unproven, is that in the case of $A = U(1)$ this recovers the ordinary differential cohomology fracture square.

{\footnotesize
 \bibliographystyle{alpha}
 \bibliography{main}
 }
\end{document}