\documentclass[10pt]{amsart}
\usepackage{amsmath,amsthm,amssymb,amsfonts}
\usepackage[mathscr]{euscript}
\usepackage{tikz}
\usepackage{enumerate}
\usepackage{enumitem}
\usepackage[colorlinks=true, linkcolor=red, citecolor = blue]{hyperref}
\usepackage[margin=2cm]{geometry}

\title{Differential Cohomology Seminar}
\date{Summer Semester 2025}
\author{Nima Rasekh}

\begin{document}
\maketitle


Differential cohomology theory wants to be a cohomology theory (so algebraic topology) that incorporates information about differentiable manifolds (so differential geometry). Various framework have been proposed to make this concept precise.

In this learning seminar we want to focus on a modern approach that uses $\infty$-categorical methods and particularly $\infty$-categorical sheaves. The first half introduces differential cohomology from an abstract higher categorical perspective. The second part aims to apply these definitions and results in various contexts.
% The idea is to learn how higher categorical methods can be used to study differentiable cohomology theories, which combine aspects of homotopy theory and differential geometry.

\section{List of Talks}
Here is the list of talks, dates and speakers:

\[
	\begin{tabular}{|l|l|l|l|l|}
		\hline 
		& \textbf{Talk} & \textbf{Speaker} & \textbf{Day} & \textbf{Date} \\ \hline 
		(1) & Motivation & Nima Rasekh & Wednesday & 09.04.2025 \\ \hline 
		(2) & History & Konrad Waldorf & Wednesday & 30.04.2025 \\ \hline
		(3) & $\infty$-Categorical Background I &	Matthias Frerichs  & Tuesday & 06.05.2025 \\ \hline
		(4) & $\infty$-Categorical Background II &	Matthias Frerichs  & Tuesday & 13.05.2025 \\ \hline
		(5) & Definition I & 	Hannes Berkenhagen & Wednesday & 21.05.2025 \\ \hline
		(6) & Definition II & Hannes Berkenhagen & Tuesday & 27.05.2025 \\ \hline 
		(7) & Examples and Operations & Alessandro Nanto & Wednesday & 03.06.2025 \\ \hline
		(8) & Main Result via Deligne Cohomology & Alessandro Nanto & Tuesday & 18.06.2025 \\ \hline
		(9) & Summary and Future Steps & Nima Rasekh & Wednesday &  16.07.2025 \\ \hline
		(10) & Review and Talk Distribution & Nima Rasekh & Tuesday & 14.10.2025 \\ \hline
  (11) & Further Examples & Matthias Ludewig & Wednesday & 29.10.2025 \\ \hline
  (12) & Twisted Cohomology Theories & Hannes Berkenhagen & Wednesday & 05.11.2025 \\ \hline
  (13) & Twisted Diff. Cohomology Theories & Alessandro Nanto & Wednesday & 12.11.2025 \\ \hline
  (14) & $\infty$-Topoi and Cohesion & Matthias Frerichs & Wednesday & 19.11.2025 \\ \hline
  (15) & Differential Cohomology and Cohesion & Nima Rasekh & Wednesday & 26.11.2025 \\ \hline
  (16) & Differential Cohomology of Lie Groupoids & Christian Becker & Wednesday & 03.12.2025 \\ \hline
	\end{tabular}
\]
\section{List of Topics}
Here is a more detailed breakdown of the topics, with relevant citations:
\begin{enumerate}[itemsep=0.3cm]
	\item \textbf{Motivation $\&$ History:} Very broad historical overview of the rise of differential cohomology theories out of ordinary cohomologies \cite{hopkinssinger2005diffcoh,simonssullivan2008diffcoh,stimpson2011diffcoh}, solution via the differential cohomology hexagon in the context of sheaves of spectra \cite[Section 2]{adh2021differentialcohomology}, \cite{debray2023diffcoh}.
	\item \textbf{Proper History:} More detailed historical development with a focus on the need for more refined invariants in the study of bundles \cite{hopkinssinger2005diffcoh,simonssullivan2008diffcoh,stimpson2011diffcoh}.
	\item[] \textbf{\large Part (I) Theory}
	\item[] The first part focuses on theory with the aim of understanding differential cohomologies and the fracture square. 
	\item \textbf{Categorical background:} The $\infty$-categorical approach to homotopy theory, analysis of accessible and presentable $\infty$-categories, definition of the $\infty$-category of spectra, relation to cohomology theories, Brown representability theorem \cite[Section 1]{lurie2017ha}, \cite{groth2010inftycats}.
	\item \textbf{Differential cohomology:} Reviewing the category of manifolds, $\infty$-categorical sheaves \cite[Section 6]{lurie2009htt}, definition of differential cohomology theories via sheaves	of spectra, first relevant properties, first simple examples such as constant sheaves \cite[Section 2]{adh2021differentialcohomology}
	\item \textbf{Examples and Operations:} Ring structure in differential cohomology via cup product \cite[Section 8]{adh2021differentialcohomology} and fiber integration \cite[Section 9]{adh2021differentialcohomology}
	\item \textbf{Main Result via Deligne Cohomology:} Definition, structures and properties of ordinary differential cohomology and Deligne cohomology and explicit fracture square \cite[Sections 6, 7]{adh2021differentialcohomology}, \cite{bunkenikolausvoelkl2016differentialcohomology}.
	\item \textbf{Further Examples:} Definition, structures and properties of ordinary differential cohomology, differential K-theory as a differential refinement of $ku$ \cite[Section 7]{adh2021differentialcohomology}, \cite{hopkinssinger2005diffcoh,simonssullivan2008diffcoh}. 
	% \item \textbf{$\mathbb{R}$-Invariant Sheaves and Localizations:} Proving $\mathbb{R}$-invariant sheaves are trivial \cite[Section 4]{adh2021differentialcohomology} and how to study arbitrary sheaves via $\mathbb{R}$-invariant ones \cite[Section 5]{adh2021differentialcohomology}
	% \item \textbf{Fracture Square:} recollements, fracture squares, recovering differential cohomology hexagon \cite[Section 6]{adh2021differentialcohomology}, \cite{barwickglasman2016recollements,lurie2018sag}.
	\item[] \textbf{\large Part (II) Applications} 
	\item[] This part will mostly be determined later. You can also make suggestions!
	\item \textbf{Characteristic Classes:} Characteristic classes of differential cohomology theories \cite[Section 14]{adh2021differentialcohomology}, \cite{bunkenikolausvoelkl2016differentialcohomology}, \cite[Section 2]{debray2023diffcoh}
	\item \textbf{Lifting Chern-Weil homomorphism:} \cite[Section 13, Section 15]{adh2021differentialcohomology}
	\item \textbf{Applications in Physics:} Applications to string theory \cite{freed2002ktheoryqft}.
	\item \textbf{Other Applications:} ... 
\end{enumerate}

{\footnotesize
\bibliographystyle{alpha}
\bibliography{main}
}
\end{document}