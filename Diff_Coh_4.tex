\documentclass[10pt]{amsart}
\usepackage{amsmath,amsthm,amssymb,amsfonts}
\usepackage[mathscr]{euscript}
\usepackage{tikz}
\usepackage{tikz-cd}
\usepackage{enumitem}
\usepackage[colorlinks=true, linkcolor=red, citecolor = blue]{hyperref}
\usepackage[margin=2.5cm]{geometry}
\setlength{\marginparwidth}{2cm}
\usepackage{circledsteps}

\usepackage{xr-hyper}
\externaldocument[art]{Diff_Coh_3}

\usepackage[nameinlink,capitalise,noabbrev]{cleveref}

\usepackage[textwidth=2cm, textsize=small, colorinlistoftodos]{todonotes}

\newcommand{\C}{\mathscr{C}}
\newcommand{\bC}{\mathbb{C}}
\newcommand{\kC}{\mathfrak{C}}
\newcommand{\D}{\mathscr{D}}
\newcommand{\E}{\mathscr{E}}
\newcommand{\bE}{\mathbb{E}}
\newcommand{\F}{\mathscr{F}}
\newcommand{\sL}{\mathscr{L}}
\newcommand{\bN}{\mathbb{N}}
\newcommand{\mN}{\mathrm{N}}
\newcommand{\I}{\mathscr{I}}
\newcommand{\s}{\mathscr{S}}
\newcommand{\bR}{\mathbb{R}}
\newcommand{\bS}{\mathbb{S}}
\newcommand{\bZ}{\mathbb{Z}}


\newcommand{\Hom}{\mathrm{Hom}}
\newcommand{\Map}{\mathrm{Map}}
\newcommand{\Ho}{\mathrm{Ho}}
\newcommand{\set}{\mathscr{S}\mathrm{et}}
\newcommand{\Sp}{\mathscr{S}\mathrm{p}}
\newcommand{\Ch}{\mathrm{Ch}}
\newcommand{\cCh}{\mathrm{cCh}}
\newcommand{\cat}{\mathscr{C}\mathrm{at}}
\newcommand{\scat}{s\mathscr{C}\mathrm{at}}
\newcommand{\sset}{s\mathscr{S}\mathrm{et}}
\newcommand{\Fun}{\mathrm{Fun}}
\newcommand{\Nat}{\mathrm{Nat}}
\newcommand{\colim}{\mathrm{colim}}
\newcommand{\Top}{\mathscr{T}\mathrm{op}}
\newcommand{\Grp}{\mathscr{G}\mathrm{rp}}
\newcommand{\Euc}{\mathscr{E}\mathrm{uc}}
\newcommand{\Mfd}{\mathscr{M}\mathrm{fd}}
\newcommand{\Kan}{\mathscr{K}\mathrm{an}}
\newcommand{\Mod}{\mathrm{Mod}}
\newcommand{\Ab}{\mathscr{A}\mathrm{b}}
\newcommand{\Shv}{\mathscr{S}\mathrm{hv}}
\newcommand{\Yon}{\mathscr{Y}\mathrm{on}}
\newcommand{\Open}{\mathscr{O}\mathrm{pen}}
\newcommand{\PSh}{\mathscr{P}\mathscr{S}\mathrm{h}}
\newcommand{\dg}{\mathrm{dg}}
\newcommand{\Sing}{\mathrm{Sing}}
\newcommand{\const}{\mathrm{Lconst}}
\newcommand{\dr}{\mathrm{dR}}
\newcommand{\tot}{\mathrm{tot}}

\newcommand{\bbefamily}{\fontencoding{U}\fontfamily{bbold}\selectfont}
\newcommand{\textbbe}[1]{{\bbefamily #1}}
\DeclareMathAlphabet{\mathbbe}{U}{bbold}{m}{n}

\def\DDelta{{\mathbbe{\Delta}}}
\newcommand{\DD}{\DDelta}

\newcommand{\adjun}[4]{
\begin{tikzcd}[row sep=0.5in, column sep=0.5in]
 #1  \arrow[r, shift left=1.8, "#3"] \pgfmatrixnextcell
 #2 \arrow[l, shift left=1.6, "#4", "\bot"'] 
\end{tikzcd}
}

\newcommand{\simpset}[7]{
 \begin{tikzcd}[row sep=0.5in, column sep=0.5in]
   #1 \arrow[r, shorten >=1ex,shorten <=1ex]
   \pgfmatrixnextcell #2 
   \arrow[l, shift left=1.2, "#5"] \arrow[l, shift right=1.2, "#4"'] 
   \arrow[r, shift right, shorten >=1ex,shorten <=1ex ] \arrow[r, shift left, shorten >=1ex,shorten <=1ex] 
   \pgfmatrixnextcell #3 
   \arrow[l] \arrow[l, shift left=2, "#7"] \arrow[l, shift right=2, "#6 "'] Top
   \arrow[r, shorten >=1ex,shorten <=1ex] \arrow[r, shift left=2, shorten >=1ex,shorten <=1ex] \arrow[r, shift right=2, 
   shorten >=1ex,shorten <=1ex]
   \pgfmatrixnextcell \cdots 
   \arrow[l, shift right=1] \arrow[l, shift left=1] \arrow[l, shift right=3] \arrow[l, shift left=3] 
 \end{tikzcd}
}


%% N.R. notes
\newcommand{\nrnote}[1]{\todo[color=green!40,linecolor=green!40!black,size=\tiny]{#1}}
\newcommand{\nrmpar}[1]{\todo[noline,color=green!40,linecolor=green!40!black,
  size=\tiny]{#1}}
\newcommand{\nrnoteil}[1]{\ \todo[inline,color=green!40,linecolor=green!40!black,size=\normalsize]{#1}}

\newtheorem{theorem}[equation]{Theorem}
\newtheorem{lemma}[equation]{Lemma}
\newtheorem{proposition}[equation]{Proposition}
\newtheorem{corollary}[equation]{Corollary}
% \newtheorem{statement}[section]{Statement}

\theoremstyle{definition}
\newtheorem{definition}[equation]{Definition}
\newtheorem{example}[equation]{Example}
% \newtheorem{attone}[section]{Attention}

\theoremstyle{remark}
\newtheorem{remark}[equation]{Remark}
% \newtheorem{intone}[section]{Intuition}
\newtheorem{notation}[equation]{Notation}
% \newtheorem{queone}[section]{Question}
% \newtheorem{conjone}[section]{Conjecture}
\newtheorem{warning}[equation]{Warning}

\numberwithin{equation}{section}

\title{Differential Cohomology Seminar 4 (Draft)}
\date{03.06.2025 $\&$ 17.06.2025}
\author{Talk by Alessandro Nanto}

\begin{document}

\maketitle

In the last talk we learned the definition of a differential cohomology theory, as a sheaf valued in spectra on the site of manifolds. This talk continues our journey through differential cohomology theories, and focuses on the following three topics:
\begin{enumerate}
  \item We want to learn how to construct non-trivial examples out of sheaves valued in chain complexes.
  \item We want to understand how we can extend classical cohomology operations to the setting of differential cohomology theories.
  \item We want to introduce suitable analogues of fiber-wise integration.  
\end{enumerate}

\section{Abelian Groups, Spectra and the Heart}
Let us start by reviewing the relation between abelian groups, rings and spectra. 
\begin{definition}
    Let $n\in\bZ$ and $X$ be a spectrum, define $\pi_n(X):=\pi_0(\Omega^{\infty+n}X)=\pi_0(X_{-n})$. We call $\pi_n$ the \textit{$n$-th homotopy group} of $X$. 
\end{definition}
\begin{remark}
 Note that since $X_n\simeq\Omega^2X_{n+2}$, for any $n$, the set $\pi_0(X_n)$ underlies the structure of an abelian group.
\end{remark}

  The category $\Sp$ underlies the structure of a symmetric monoidal $\infty$-category (\cite[Corollary 4.8.2.19]{lurie2017ha}). Following \cite{lurie2017ha}, we denote by $\otimes$ the tensor product on $\Sp$.
  \begin{definition}
    A commutative algebra object in $\Sp$ is called an \emph{$\bE_\infty$-ring spectrum}, see \cite[Definition 7.1.0.1]{lurie2017ha}. Given an $\bE_\infty$-ring spectrum $R$, denote by $\Mod_R$ the corresponding category of \emph{left $R$-module spectra}, see \cite[Definition 7.1.1.2]{lurie2017ha}. 
  \end{definition}
  \begin{remark}
    The sphere spectrum $\bS$ acts as the monoidal unit of $\Sp$, therefore it is a $\bE_\infty$-ring spectrum. The category $\Mod_{\bS}$ is canonically equivalent to $\Sp$. 
  \end{remark}
    \begin{definition}
    Denote by $\Sp_{\geq0}\subseteq\Sp$ the full sub-category generated by \emph{connective spectra}, i.e. spectra $X$ such that $\pi_n(X)\simeq0$, for all $n<0$. Denote by $\Sp^\heartsuit\subseteq\Sp_{\geq0}$ the \emph{heart of spectra}, i.e. the full sub-category generated by spectra $X$ such that $\pi_n(X)\simeq0$, for all $n>0$.
  \end{definition}
  We have the following result relating connective spectra and the heart, which follow immediately.
  \begin{lemma}
  Let $X$ be a connective spectrum. The following are equivalent:
  \begin{enumerate}
    \item $X$ is in the heart. 
    \item $\pi_n(\Omega^\infty X)=0$, for all $n>0$.
    \item $\Hom_{\s_*}(S,\Omega^\infty X)\simeq0$, for all connected, pointed spaces $S$.
    \item $X$ is local with respect to the class of maps $\Sigma^\infty S\to0$, for every connected pointed space $S$.
  \end{enumerate}
  \end{lemma}

  The category $\Sp_{\geq0}$ is presentable and $\pi_0$ induces an equivalence between the heart and $\Ab$ (\cite[Proposition 1.4.3.6]{lurie2017ha}). The heart is a sub-category of local objects of connective spectra, therefore the inclusion $\Ab\simeq\Sp^\heartsuit\subseteq\Sp_{\geq0}$ is a right adjoint. The category $\Sp_{\geq0}$ is closed under $\otimes$ and, given $X,Y$ connective spectra, 

\begin{align}\label{eq:stablehurewicz}
  \pi_0(X\otimes Y)\simeq\pi_0(X)\otimes\pi_0(Y)
\end{align}
see \cite[Theorem 2.3.28]{davies2024atii}

\begin{definition}Given an abelian group $A$, denote by $HA$ the (unique up to equivalence) spectrum of the heart such that $\pi_0(HA)\simeq A$. We call $HA$ the \emph{Eilenberg-Mac Lane spectrum} of $A$.
\end{definition}
Using \cref{eq:stablehurewicz}, one can prove $H$, viewed as a functor $\Ab\to\Sp$, is lax monoidal. In particular, if $R$ is a commutative ring, then $HR$ is a connective $\bE_\infty$-ring spectrum. On the other hand, if $R$ is a connective $\bE_\infty$-ring spectrum and $M$ a connective module, then $\pi_0(M)$ is a $\pi_0(R)$-module. 
\begin{definition}
  Given a commutative ring $R$, denote by $\Ch(R)=\Ch(\Mod_{R})$ the ordinary category of unbounded chain complexes. Let $\D(R)$ be the $\infty$-localization of $\Ch(R)$ at the class of quasi-isomorphisms.  
\end{definition} 
Similar to the heart of spectra, given an $\bE_\infty$-ring spectrum $R$, denote by $\Mod_{R}^\heartsuit\subseteq \Mod_{R}$ the full sub-category generated by $R$-modules such that the underlying spectrum belongs to the heart of spectra. 
\begin{theorem}[Stable Dold-Kan Correspondence]\label{thm:stabledk}
  Let $R$ be a commutative ring. 
  \begin{enumerate}
    \item $\Mod_{R}\simeq \Mod_{HR}^\heartsuit$ via taking Eilenberg-Mac Lane spectra.
    \item The equivalence in (1) extends to an equivalence $H:\D(R)\simeq \Mod_{HR}$ of symmetric monoidal $\infty$-categories.
  \end{enumerate} 
\end{theorem}
\begin{proof}
  (1) is \cite[Proposition 7.1.1.13]{lurie2017ha}, while (2) is \cite[Theorem 7.1.2.13]{lurie2017ha}.
\end{proof}
An interesting consequence of \cref{thm:stabledk} is the following:
\begin{corollary}\label{cor:homotopyvshomology}
  Given $F\in\D(R)$, then $\pi_n(HF)\simeq H_n(F)$, for all $n\in\bZ$.
\end{corollary}
\begin{proof}
  \begin{align*}
    \pi_n(HF) & =\pi_0(\Omega^{\infty+n}HF)\\
    & \overset{\Circled{1}}{\simeq} \pi_0(\Hom_{\Sp}(\Sigma^n\bS,HF)) \\
    & \overset{\Circled{2}}{\simeq} \pi_0(\Hom_{\Mod_{HR}}(\Sigma^n HR,HF)) & \\
    & \overset{\Circled{3}}{\simeq} \pi_0(\Hom_{\D(R)}(R[n],F))\\
    & \overset{\Circled{4}}{\simeq} H_n(F)
  \end{align*} \Circled{1} The functor $\Omega^{\infty+n}$ is corepresented by the shifted sphere spectrum $\Sigma^n\bS$. \Circled{2} The forgetful functor $\Mod_{HR}\to\Mod_\bS\simeq\Sp$ is right adjoint to tensoring by $HR$ and $HR\otimes(\Sigma^n\bS)\simeq \Sigma^nHR$. \Circled{3} \cref{thm:stabledk} \Circled{4} $\pi_0$ of the mapping space $\Hom_{\D(R)}(R[n],F)$ is equivalent to the mapping space $R[n]\to F$ in the \emph{ordinary} derived category of $R$, i.e. homotopy classes of maps $R[n]\to F$, which correspond exactly to classes in $H_n(F)$.
\end{proof}

\section{More \texorpdfstring{$\infty$}{oo}-categorical baggage}\label{sec:baggage}

Let $\C$ be a presentable $\infty$-category. The $\infty$-categorical background given in previous talks allows to conclude the existence of a number of functors. Here we give a (somewhat) explicit formula for one. 
\begin{remark}
  Recall $\Euc$, the full sub-category of $\Mfd$ generated by Euclidean manifolds $\bR^n$, for every $n\geq0$. Denote by $j$ the inclusion functor $\Euc\subseteq\Mfd$. Recall that the restriction along $j$ induces an equivalence $\Shv(\Mfd,\C)\simeq\Shv(\Euc,\C)$, see \cite[Corollary A.5.6]{amabeldebrayhaine2021diffcoh}.
\end{remark}
Evaluation at $\{0\}$ induces an adjunction $(\const,\Gamma):\C\to\Shv(\Mfd,\C)$, where the functor $\Gamma$ is evaluation at $\{0\}$, while the left adjoint $\const$ maps $C\in\C$ to the sheafification of the constant pre-sheaf with value $C$. 
\begin{remark}\label{rmk:cotensor}
Every presentable $\infty$-category $\C$ is uniquely \emph{cotensored over} $\s$, see \cite[Remark 5.5.2.6]{lurie2009htt}. More explicitly, for every space $S$ and object $C$, there is an object $C^S$ together with a natural equivalence $$\Hom_\s(S,\Hom_\C(-,C))\simeq\Hom_\C(-,C^S)$$
\end{remark}
\begin{definition}
  Denote by $\Sing$ the functor $\Mfd\to\s$ mapping a manifold to its underlying space. Given a presentable $\infty$-category $\C$, denote by $\flat$ the composition $\C\to\Fun(\s^{op},\C)\to\Fun(\Mfd^{op},\C)$, the first functor coming from \cref{rmk:cotensor}, the second being pre-composition with $\Sing^{op}$.
\end{definition}
Explicitly, given an object $C\in\C$, the associated pre-sheaf $\flat C$ maps a manifold $M$ to $C^{\Sing(M)}$. 
\begin{lemma}[{\cite[Corollary 6.46]{bunkegepner2021differential}}]\label{lem:flatsheaf}
  $\flat$ factors through $\Shv(\Mfd,\C)\subseteq\Fun(\Mfd^{op},\C)$.
\end{lemma}
\cref{lem:flatsheaf} is the direct consequence of a weaker version of a generalized version of Seifert-van Kampen theorem, namely \cite[Proposition A.3.2]{lurie2017ha}, stating that, given a topological space $X$ and a covering sieve $\mathscr{O}$, the space $\Sing(X)$ is the colimit of $\Sing(U)$ over $U\in\mathscr{O}$.
\begin{theorem}\label{thm:leftadjoint}
  $\flat:\C\to\Shv(\Mfd,\C)$ is left adjoint to $\Gamma$.
\end{theorem}
\begin{proof}
  The composition $\C\xrightarrow{\flat}\Shv(\Mfd,\C)\xrightarrow{j_*}\Shv(\Euc,\C)$ maps an object $C$ to the sheaf $\flat C$ restricted to Euclidean spaces. Since $\bR^n$ is contractible, $(\flat C)(\bR^n)=C^{\Sing(\bR^n)}\simeq C$ and so $\flat$ restricted to $\Euc$ is equivalent to $\mathrm{Const}$, the functor taking $C$ to the the pre-sheaf with constant value $C$, which is left adjoint to $\Gamma$ restricted to $\Euc$.
\end{proof}

\section{Sheaves of complexes and spectra}

The stable Dold-Kan correspondence allowes us to move freely between sheaves of $H\bZ$-module spectras and sheaves valued in $\D(\bZ)$.

\begin{remark}\label{rmk:identification}
  We identify the category of cochain complexes with $\Ch(R)$ by reversing grading. Namely, given a cochain $V^*$, we are implicitly identifying it with the chain complex $V_n=V^{-n}$. 
\end{remark}
\begin{definition}[{\cite[Definition 7.14]{bunkenikolausvoelkl2016diffcoh}}]Given $n\in\bZ$, denote by $\tau^{\geq n}$, resp. $\tau^{\leq n}$, the \emph{naive truncation functors}, mapping a cochain complex $V^*$ to \[\cdots\to0\to V^n\to V^{n+1}\to\cdots, \qquad \mbox{resp. }\cdots\to V^{n-1}\to V^n\to0\to\cdots\]Given $F:\Mfd^{op}\to\Ch(\bZ)$, denote by $F^{\geq n}$ the composite $\Mfd^{op}\xrightarrow{F}\Ch(\bZ)\xrightarrow{\tau^{\geq n}}\Ch(\bZ)$, and similarly we define $F^{\leq n}$. If $F$ is a sheaf, then so are its truncations. 
\end{definition}
\begin{lemma}[{\cite[Lemma 7.12]{bunkenikolausvoelkl2016diffcoh}}]\label{lem:sheafloc}
  Let $F:\Mfd^{op}\to\Ch(\bZ)$ a sheaf of chain complexes of $C^\infty$-modules, then $\Mfd^{op}\xrightarrow{F}\Ch(\bZ)\to\D(\bZ)$ is a sheaf. 
\end{lemma}
\begin{definition}\label{def:forms}
  Denote by $\Omega^*$ the sheaf $\Mfd^{op}\to\Ch(\bZ)$ mapping a manifold to its de Rham complex. 
\end{definition}
\cref{lem:sheafloc} ensures that the sheaf in \cref{def:forms} and the corresponding naive truncations remain sheaves after post-composition with the localization functor $\Ch(\bZ)\to\D(\bZ)$.  
\begin{definition}\label{def:EMsheaf}
  Given a sheaf $F:\Mfd^{op}\to\D(\bZ)$, denote by $HF$ the \emph{Eilenberg-Mac Lane sheaf} of $H\bZ$-module spectra obtained by applying point-wise the equivalence of \cref{thm:stabledk}. 
\end{definition}
Recall now the machinery set-up in \cref{sec:baggage}.
\begin{remark}\label{rmk:Dcotensor}
  Since $\D(\bZ)$ is presentable, we know that they is cotensored over $\s$. Given a space $S$ and a chain complex $M_*$, the cotensor $M_*{}^S$ is the chain complex of graded linear maps $C_*(S,\bZ)\to M_*$, from the (normalized) singular chain complex of $S$ to $M_*$, see \cite[Definition 1.3.2.1]{lurie2017ha}. In particular, let $M_*=M$ be concentrated in degree 0, then $M_*{}^S$ is the singular cochain complex of $S$ with values in $M$. 
\end{remark}
\begin{definition}\label{def:derhammorf}
Consider the morphism $\Omega^*(M)\to (\flat\bR)(M)=C^*(M,\bR)$ taking a form $\omega\in\Omega^n(M)$ to the linear map $\int\omega:C_{n}(M,\bZ)\to\bR$. We call the induced transformation $\dr:\Omega^*\to\flat\bR$ the \emph{de Rham morphism}.  
\end{definition}
\begin{lemma}[{\cite[Theorem 3.25]{abad2010ainftyrhamtheoremintegration}}]\label{lem:derham} $\dr$ is point-wise an equivalence of $A_\infty$-algebras.
\end{lemma}
\section{Deligne Cohomology}

Finally, we have enough machinery to talk about Deligne cohomology. 
\begin{definition}\label{def:deligne}
  Given $\ell\in\bN$, define $\hat{\bZ}(\ell):\Mfd^{op}\to\D(\bZ)$ as the limit of 
  \[\begin{tikzcd}
     & \Omega^{\geq \ell}\arrow[d] \\
    \flat\bZ \arrow[r] & \flat\bR
  \end{tikzcd}\]
  The vertical morphism being the composition $\Omega^{\geq \ell}\subseteq\Omega^*\xrightarrow{\dr}\flat\bR$. We call the corresponding sheaf of $H\bZ$-modules spectra $H\hat{\bZ}(\ell)$ the $\ell$\emph{-th Deligne sheaf}.
\end{definition}
\begin{remark}[Model A, see {\cite[\S 3.2]{hopkinssinger2005diffcoh}}]\label{rmk:A} Let $\acute{C}(\ell)^n(M)\subseteq C^n(M,\bZ)\oplus C^{n-1}(M,\bR)\oplus\Omega^n(M)$ consist of triples $(c,h,\omega)$ for which $\omega=0$ if $n<\ell$, with differential $\delta(c,h,\omega)=(\delta c,\dr(\omega)-c-\delta h,d\omega)$. This complex $\acute{C}^*(\ell)(M)$ fits into a diagram
\[\begin{tikzcd}
    \acute{C}^*(\ell)(M) \arrow[d]\arrow[r] & \Omega^{\geq \ell}(M)\arrow[d] \\
    C^*(M,\bZ) \arrow[r] & C^*(M,\bR)
  \end{tikzcd}\]which commutes up to homotopy given by the projections $\acute{C}^n(\ell)(M)\to C^{n-1}(M,\bR)$. The diagram above model the homotopy pullback of \cref{def:deligne}, hence $\acute{C}^*(\ell)(M)$ is a model for $\hat{\bZ}(\ell)(M)$. 
\end{remark}
\begin{remark}[Model B]\label{rmk:B} Recall that $\flat$ preserves cofiber sequences, since it is left adjoint, and that fiber sequences are the same a cofiber sequences in stable $\infty$-categories. Consider the diagram
  \[\begin{tikzcd}
    \hat{\bZ}(\ell)\arrow[r]\arrow[d] & \Omega^{\geq \ell}\arrow[d]\arrow[dd,bend left=5em,color=red] \\
    \flat\bZ \arrow[r]\arrow[d] & \flat\bR\arrow[d] \\
    0\arrow[r,color=red] & \flat(\bR/\bZ)
  \end{tikzcd}\]
  Since the bottom square is an homotopy pullback, $\hat{\bZ}(\ell)(M)$ is equivalent to the homotopy pullback of the diagram in \textcolor{red}{red}. Let $\breve{C}^{n}(\ell)(M)\subseteq C^{n-1}(M,\bR/\bZ)\oplus\Omega^n(M)$ consist of pairs $(\chi,\omega)$ for which $\omega=0$ if $n<\ell$, with differential $\delta(\chi,\omega)=(e^{2\pi i\dr(\omega)}-\delta\chi,d\omega)$. Similar to \cref{rmk:A}, the complex $\breve{C}^*(\ell)(M)$ fits into the above diagram so that the outer square is an homotopy pullback, hence it is equivalent to $\hat{\bZ}(\ell)(M)$.
\end{remark}
Take an $n$-cocycle ($n\geq\ell$) in the model from \cref{rmk:B}, i.e. $(\chi,\omega)\in C^{n-1}(M,\bR/\bZ)\oplus\Omega^{n}(M)$ such that $d\omega=0$ and $\delta\chi=e^{2\pi i\dr(\omega)}$. Such a cocycle determines a differential character of degree $n-1$ for $M$, in the sense of the following definition:
\begin{definition}[{\cite[Definition 3.4]{hopkinssinger2005diffcoh}}, see also {\cite[Chapter 5]{barbecker2014diffchar}}] Consider a manifold $M$, a \emph{differential character} of degree $n$ consists of a character $\chi:Z^\infty_{n}(M,\bZ)\to\bR/\bZ$ on the group of smooth $n$-cycles of $M$ together with a $n$-form $\omega\in\Omega^{n+1}(M)$, such that, for every smooth $(n+1)$-chain $c$, \[\chi(\partial c)=e^{2\pi i\int_c\omega}\]
\end{definition}

\begin{remark}[Model C, see {\cite[Lemma 7.3.4]{amabeldebrayhaine2021diffcoh}}]\label{rmk:C} Consider the following diagram in the category of {sheaves} on $\Euc$
  \[\begin{tikzcd}
     j_*\hat{\bZ}(\ell)\arrow[d]\arrow[r] & \Omega^{\geq \ell}\arrow[d] \arrow[r] & 0\arrow[d,color=red] \\
    \bZ \arrow[r]\arrow[rr,bend right=2.5em,color=red] & \Omega^*\arrow[r] & \Omega^{\leq\ell-1}
  \end{tikzcd}\] The left square is the pullback square of $\hat{\bZ}(\ell)$ restricted to $\Euc$. Since the right square is a pullback, $j_*\hat{\bZ}(\ell)$ is equivalent to the pullback of the diagram in \textcolor{red}{red}. Let $\check{C}^*(\ell)$ be the sheaf of chain complexes $\bZ\to\Omega^0\to\cdots\cdots\to\Omega^{\ell-1}\to0\to\cdots$, where $\bZ$ is in degree 0. The complex $\check{C}^*(\ell)$ fits into the above diagram, so that the outer square is an homotopy pullback, and thus $j_*\hat{\bZ}(\ell)\simeq\check{C}^*(\ell)$ and $j^*\check{C}^*(\ell)\simeq\hat{\bZ}(\ell)$. 

  Given a manifold $M$, let $\mathscr{O}$ be a good open cover and $\I(\mathscr O)$ the closure of $\mathscr{O}$ under finite intersections, then
\[
    \hat{\bZ}(\ell)(M) \simeq\lim_{U\in\I(\mathscr O)} \check{C}^*(\ell)(U)\simeq\lim_{n\in\Delta}\prod_{U_1,\cdots,U_n\in\mathscr{O}}\check{C}^*(\ell)(U_1\cap\cdots\cap U_n) 
  \]
  Finally, we can apply \cite[Lemma 7.10]{bunkenikolausvoelkl2016diffcoh} to calculate the last limit as a the total complex functor applied to the bicomplex \[\check{C}^{m,n}(\ell)(\mathscr{O}):=\prod_{U_1,\cdots,U_n\in\mathscr{O}}\check{C}^m(\ell)(U_1\cap\cdots\cap U_n)\]
\end{remark}

% TO BE CONTINUED

\section{Unfolding the fracture square of Deligne Cohomology}

{\footnotesize
\bibliographystyle{alpha}
\bibliography{main}
}
\end{document}