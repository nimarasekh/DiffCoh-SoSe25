\documentclass[10pt]{amsart}
\usepackage{amsmath,amsthm,amssymb,amsfonts}
\usepackage[mathscr]{euscript}
\usepackage{tikz}
\usepackage{tikz-cd}
\usepackage{enumitem}
\usepackage[colorlinks=true, linkcolor=red, citecolor = blue]{hyperref}
\usepackage[margin=2.5cm]{geometry}
\setlength{\marginparwidth}{2cm}
\usepackage{circledsteps}

\usepackage[nameinlink,capitalise,noabbrev]{cleveref}

\usepackage[textwidth=2cm, textsize=small, colorinlistoftodos]{todonotes}

\newcommand{\bA}{\mathrm{A}}
\newcommand{\bbA}{\mathbb{A}}
\newcommand{\B}{\mathscr{B}}
\newcommand{\bbB}{\mathbb{B}}
\newcommand{\C}{\mathscr{C}}
\newcommand{\bC}{\mathbb{C}}
\newcommand{\kC}{\mathfrak{C}}
\newcommand{\D}{\mathscr{D}}
\newcommand{\E}{\mathscr{E}}
\newcommand{\bE}{\mathbb{E}}
\newcommand{\F}{\mathscr{F}}
\newcommand{\sL}{\mathscr{L}}
\newcommand{\bN}{\mathbb{N}}
\newcommand{\mN}{\mathrm{N}}
\newcommand{\I}{\mathscr{I}}
\newcommand{\s}{\mathscr{S}}
\newcommand{\bR}{\mathbb{R}}
\newcommand{\bS}{\mathbb{S}}
\newcommand{\bZ}{\mathbb{Z}}


\newcommand{\Hom}{\mathrm{Hom}}
\newcommand{\Map}{\mathrm{Map}}
\newcommand{\Ho}{\mathrm{Ho}}
\newcommand{\set}{\mathscr{S}\mathrm{et}}
\newcommand{\Sp}{\mathscr{S}\mathrm{p}}
\newcommand{\Ch}{\mathrm{Ch}}
\newcommand{\cCh}{\mathrm{cCh}}
\newcommand{\cat}{\mathscr{C}\mathrm{at}}
\newcommand{\scat}{s\mathscr{C}\mathrm{at}}
\newcommand{\sset}{s\mathscr{S}\mathrm{et}}
\newcommand{\Fun}{\mathrm{Fun}}
\newcommand{\Nat}{\mathrm{Nat}}
\newcommand{\colim}{\mathrm{colim}}
\newcommand{\Top}{\mathscr{T}\mathrm{op}}
\newcommand{\Grp}{\mathscr{G}\mathrm{rp}}
\newcommand{\Euc}{\mathscr{E}\mathrm{uc}}
\newcommand{\Mfd}{\mathscr{M}\mathrm{fd}}
\newcommand{\Kan}{\mathscr{K}\mathrm{an}}
\newcommand{\Mod}{\mathrm{Mod}}
\newcommand{\Ab}{\mathscr{A}\mathrm{b}}
\newcommand{\Shv}{\mathscr{S}\mathrm{hv}}
\newcommand{\Yon}{\mathscr{Y}\mathrm{on}}
\newcommand{\Open}{\mathscr{O}\mathrm{pen}}
\newcommand{\PSh}{\mathscr{P}\mathscr{S}\mathrm{h}}
\newcommand{\dg}{\mathrm{dg}}
\newcommand{\Sing}{\mathrm{Sing}}
\newcommand{\const}{\mathrm{L}}
\newcommand{\dr}{\mathrm{dR}}
\newcommand{\tot}{\mathrm{tot}}
\newcommand{\op}{\mathrm{op}}
\newcommand{\sm}{\mathrm{sm}}
\newcommand{\Dr}{\D\mathrm{r}}
\newcommand{\fib}{\mathrm{fib}}

\newcommand{\bbefamily}{\fontencoding{U}\fontfamily{bbold}\selectfont}
\newcommand{\textbbe}[1]{{\bbefamily #1}}
\DeclareMathAlphabet{\mathbbe}{U}{bbold}{m}{n}

\def\DDelta{{\mathbbe{\Delta}}}
\newcommand{\DD}{\DDelta}

\newcommand{\adjun}[4]{
\begin{tikzcd}[row sep=0.5in, column sep=0.5in]
 #1  \arrow[r, shift left=1.8, "#3"] \pgfmatrixnextcell
 #2 \arrow[l, shift left=1.6, "#4", "\bot"'] 
\end{tikzcd}
}

\newcommand{\simpset}[7]{
 \begin{tikzcd}[row sep=0.5in, column sep=0.5in]
   #1 \arrow[r, shorten >=1ex,shorten <=1ex]
   \pgfmatrixnextcell #2 
   \arrow[l, shift left=1.2, "#5"] \arrow[l, shift right=1.2, "#4"'] 
   \arrow[r, shift right, shorten >=1ex,shorten <=1ex ] \arrow[r, shift left, shorten >=1ex,shorten <=1ex] 
   \pgfmatrixnextcell #3 
   \arrow[l] \arrow[l, shift left=2, "#7"] \arrow[l, shift right=2, "#6 "'] Top
   \arrow[r, shorten >=1ex,shorten <=1ex] \arrow[r, shift left=2, shorten >=1ex,shorten <=1ex] \arrow[r, shift right=2, 
   shorten >=1ex,shorten <=1ex]
   \pgfmatrixnextcell \cdots 
   \arrow[l, shift right=1] \arrow[l, shift left=1] \arrow[l, shift right=3] \arrow[l, shift left=3] 
 \end{tikzcd}
}


%% N.R. notes
\newcommand{\nrnote}[1]{\todo[color=green!40,linecolor=green!40!black,size=\tiny]{#1}}
\newcommand{\nrmpar}[1]{\todo[noline,color=green!40,linecolor=green!40!black,
  size=\tiny]{#1}}
\newcommand{\nrnoteil}[1]{\ \todo[inline,color=green!40,linecolor=green!40!black,size=\normalsize]{#1}}

\newtheorem{theorem}[equation]{Theorem}
\newtheorem{lemma}[equation]{Lemma}
\newtheorem{proposition}[equation]{Proposition}
\newtheorem{corollary}[equation]{Corollary}
% \newtheorem{statement}[section]{Statement}

\theoremstyle{definition}
\newtheorem{definition}[equation]{Definition}
\newtheorem{example}[equation]{Example}
% \newtheorem{attone}[section]{Attention}

\theoremstyle{remark}
\newtheorem{remark}[equation]{Remark}
% \newtheorem{intone}[section]{Intuition}
\newtheorem{notation}[equation]{Notation}
% \newtheorem{queone}[section]{Question}
% \newtheorem{conjone}[section]{Conjecture}
\newtheorem{warning}[equation]{Warning}

\numberwithin{equation}{section}

\title{Differential Cohomology Seminar 4 (Draft)}
\date{03.06.2025 $\&$ 18.06.2025}
\author{Talk by Alessandro Nanto}

\begin{document}

\maketitle

In the last talk we learned the definition of a differential cohomology theory, as a sheaf valued in spectra on the site of manifolds. This talk continues our journey through differential cohomology theories, and focuses on the following three topics:
\begin{enumerate}
  \item We want to learn how to construct non-trivial examples out of sheaves valued in chain complexes.
  \item We want to understand how we can extend classical cohomology operations to the setting of differential cohomology theories.
  \item We want to introduce suitable analogues of fiber-wise integration.  
\end{enumerate}

\section{Abelian Groups, Spectra and the Heart}
Let us start by reviewing the relation between abelian groups, rings and spectra. 
\begin{definition}
    Let $n\in\bZ$ and $X$ be a spectrum, define $\pi_n(X):=\pi_0(\Omega^{\infty+n}X)=\pi_0(X_{-n})$. We call $\pi_n$ the \textit{$n$-th homotopy group} of $X$. 
\end{definition}
\begin{remark}
 Note that since $X_n\simeq\Omega^2X_{n+2}$, for any $n$, the set $\pi_0(X_n)$ underlies the structure of an abelian group.
\end{remark}

  The category $\Sp$ underlies the structure of a symmetric monoidal $\infty$-category (\cite[Corollary 4.8.2.19]{lurie2017ha}). Following \cite{lurie2017ha}, we denote by $\otimes$ the tensor product on $\Sp$.
  \begin{definition}
    A (commutative) algebra object in $\Sp$ is called an ($\bE_\infty$-)\emph{ring spectrum}, see \cite[Definition 7.1.0.1]{lurie2017ha}. Given a ring spectrum $R$, denote by $\Mod_R$ the corresponding category of \emph{left $R$-module spectra}, see \cite[Definition 7.1.1.2]{lurie2017ha}. 
  \end{definition}
  \begin{remark}
    The sphere spectrum $\bS$ acts as the monoidal unit of $\Sp$, therefore it is a $\bE_\infty$-ring spectrum. The category $\Mod_{\bS}$ is canonically equivalent to $\Sp$. 
  \end{remark}
    \begin{definition}
    Denote by $\Sp_{\geq0}\subseteq\Sp$ the full sub-category generated by \emph{connective spectra}, i.e. spectra $X$ such that $\pi_n(X)\simeq0$, for all $n<0$. Denote by $\Sp^\heartsuit\subseteq\Sp_{\geq0}$ the \emph{heart of spectra}, i.e. the full sub-category generated by spectra $X$ such that $\pi_n(X)\simeq0$, for all $n>0$.
  \end{definition}
  We have the following result relating connective spectra and the heart, which follow immediately.
  \begin{lemma}
  Let $X$ be a connective spectrum. The following are equivalent:
  \begin{enumerate}
    \item $X$ is in the heart. 
    \item $\pi_n(\Omega^\infty X)=0$, for all $n>0$.
    \item $\Hom_{\s_*}(S,\Omega^\infty X)\simeq0$, for all connected, pointed spaces $S$.
    \item $X$ is local with respect to the class of maps $\Sigma^\infty S\to0$, for every connected pointed space $S$.
  \end{enumerate}
  \end{lemma}

  The category $\Sp_{\geq0}$ is presentable and $\pi_0$ induces an equivalence between the heart and $\Ab$ (\cite[Proposition 1.4.3.6]{lurie2017ha}). The heart is a sub-category of local objects of connective spectra, therefore the inclusion $\Ab\simeq\Sp^\heartsuit\subseteq\Sp_{\geq0}$ is a right adjoint. The category $\Sp_{\geq0}$ is closed under $\otimes$ and, given $X,Y$ connective spectra, 

\begin{align}\label{eq:stablehurewicz}
  \pi_0(X\otimes Y)\simeq\pi_0(X)\otimes\pi_0(Y)
\end{align}
see \cite[Theorem 2.3.28]{davies2024atii}

\begin{definition}Given an abelian group $A$, denote by $HA$ the (unique up to equivalence) spectrum of the heart such that $\pi_0(HA)\simeq A$. We call $HA$ the \emph{Eilenberg-Mac Lane spectrum} of $A$.
\end{definition}
Using \cref{eq:stablehurewicz} and the adjunction between $H$ and $\pi_0$, one can prove $H$, viewed as a functor $\Ab\to\Sp$, is lax monoidal. In particular, if $R$ is a commutative ring, then $HR$ is a connective $\bE_\infty$-ring spectrum. On the other hand, if $R$ is a connective $\bE_\infty$-ring spectrum and $M$ a connective module, then $\pi_0(M)$ is a $\pi_0(R)$-module. 
\begin{definition}
  Given a commutative ring $R$, denote by $\Ch(R)=\Ch(\Mod_{R})$ the ordinary category of unbounded chain complexes. Let $\D(R)$ be the $\infty$-localization of $\Ch(R)$ at the class of quasi-isomorphisms, called the \emph{derived category}.  
\end{definition} 
Similar to the heart of spectra, given an $\bE_\infty$-ring spectrum $R$, denote by $\Mod_{R}^\heartsuit\subseteq \Mod_{R}$ the full sub-category generated by $R$-modules such that the underlying spectrum belongs to the heart of spectra. 
\begin{theorem}[Stable Dold-Kan Correspondence]\label{thm:stabledk}
  Let $R$ be a commutative ring. 
  \begin{enumerate}
    \item $\Mod_{R}\simeq \Mod_{HR}^\heartsuit$ via taking Eilenberg-Mac Lane spectra.
    \item The above equivalence extends to an equivalence $H:\D(R)\simeq \Mod_{HR}$ of symmetric monoidal $\infty$-categories.
  \end{enumerate} 
\end{theorem}
\begin{proof}
  (1) is \cite[Proposition 7.1.1.13]{lurie2017ha}, while (2) is \cite[Theorem 7.1.2.13]{lurie2017ha}.
\end{proof}
An interesting consequence of \cref{thm:stabledk} is the following:
\begin{corollary}\label{cor:homotopyvshomology}
  Given $F\in\D(R)$, then $\pi_n(HF)\simeq H_n(F)$, for all $n\in\bZ$.
\end{corollary}
\begin{proof}
  \begin{align*}
    \pi_n(HF) & =\pi_0(\Omega^{\infty+n}HF)\\
    & \overset{\Circled{1}}{\simeq} \pi_0(\Hom_{\Sp}(\Sigma^n\bS,HF)) \\
    & \overset{\Circled{2}}{\simeq} \pi_0(\Hom_{\Mod_{HR}}(\Sigma^n HR,HF)) & \\
    & \overset{\Circled{3}}{\simeq} \pi_0(\Hom_{\D(R)}(R[n],F))\\
    & \overset{\Circled{4}}{\simeq} H_n(F)
  \end{align*} \Circled{1} The functor $\Omega^{\infty+n}$ is corepresented by the shifted sphere spectrum $\Sigma^n\bS$. \Circled{2} The forgetful functor $\Mod_{HR}\to\Mod_\bS\simeq\Sp$ is right adjoint to tensoring with $HR$ and $HR\otimes(\Sigma^n\bS)\simeq \Sigma^nHR$. \Circled{3} \cref{thm:stabledk} \Circled{4} $\pi_0$ of the mapping space $\Hom_{\D(R)}(R[n],F)$ is equivalent to the mapping space $R[n]\to F$ in the \emph{ordinary} derived category of $R$, i.e. homotopy classes of maps $R[n]\to F$, which correspond exactly to classes in $H_n(F)$.
\end{proof}

\section{Locally constant sheaves on manifolds}\label{sec:constant sheaves}

Let $\C$ be a presentable $\infty$-category. The $\infty$-categorical background given in previous talks allows to conclude the existence of a number of functors. Here we give a (somewhat) explicit formula for one. 
\begin{remark}
  Let $\B$ be a full, dense sub-site of $\Mfd$, recall then that, for every presentable category $\C$, the restriction functor induces an equivalence $\Shv(\Mfd,\C)\simeq\Shv(\B,\C)$. 
\end{remark}
Evaluation at $\{0\}$ induces an adjunction $(\const,\Gamma):\C\to\Shv(\Mfd,\C)$, where the functor $\Gamma$ is evaluation at $\{0\}$, while the left adjoint $\const$ maps $C\in\C$ to the sheafification of the constant pre-sheaf with value $C$. 
\begin{remark}\label{rmk:cotensor}
Every presentable $\infty$-category $\C$ is uniquely \emph{cotensored over} $\s$, see \cite[Remark 5.5.2.6]{lurie2009htt}. More explicitly, for every space $S$ and object $C$, there is an object $C^S$ together with a natural equivalence $$\Hom_\s(S,\Hom_\C(-,C))\simeq\Hom_\C(-,C^S)$$
\end{remark}
\begin{definition}\label{def:flat}
  Denote by $\Sing$ the functor $\Mfd\to\s$ mapping a manifold to its underlying space of \emph{smooth} simplexes. Given a presentable $\infty$-category $\C$, denote by $\flat$ the composition $\C\to\Fun(\s^{\op},\C)\to\Fun(\Mfd^{\op},\C)$, the first functor coming from \cref{rmk:cotensor}, the second being pre-composition with $\Sing^{\op}$.
\end{definition}
Explicitly, given an object $C\in\C$, the associated pre-sheaf $\flat C$ maps a manifold $M$ to $C^{\Sing(M)}$.
\begin{remark}
  Given a topological space $X$, denote by $\widehat{\Sing}(X)$ the corresponding space, i.e. its singular simplicial set. If $X=M$ is a smooth manifold, then $\Sing(M)\subseteq\widehat{\Sing}(M)$ and the inclusion is a homotopy equivalence by Whitney's Approximation Theorem, see \cite[Theorem 1.6]{zuoqin2021whitney}. 
\end{remark}
% Smooth singular simplicial set is homotopy equivalent to continuous singular simplicial set. Lurie proves continuous sss is a cosheaf, but this is naturally equivalent to $\Sing$. 

\begin{lemma}[{\cite[Corollary 6.46]{bunkegepner2021differential}}]\label{lem:flatsheaf}
  $\Sing$ is a cosheaf. In particular,
  $\flat$ factors through $\Shv(\Mfd,\C)\subseteq\Fun(\Mfd^{\op},\C)$.
\end{lemma}
\cref{lem:flatsheaf} is essentially the consequence of a generalized version of Seifert-van Kampen theorem, namely \cite[Proposition A.3.2]{lurie2017ha}, stating that, given a topological space $X$ and a covering sieve $\mathscr{O}$, the space $\widehat{\Sing}(X)$ is the colimit of $\widehat{\Sing}(U)$, over $U\in\I(\mathscr{O})$.
\begin{theorem}\label{thm:leftadjoint}
  $\flat:\C\to\Shv(\Mfd,\C)$ is left adjoint to $\Gamma$.
\end{theorem}
\begin{proof}
  The composition $\C\xrightarrow{\flat}\Shv(\Mfd,\C)\xrightarrow{|_{\Euc}}\Shv(\Euc,\C)$ maps an object $C$ to the sheaf $\flat C$ restricted to Euclidean spaces. Since $\bR^n$ is contractible, $(\flat C)(\bR^n)=C^{\Sing(\bR^n)}\simeq C$ and so $\flat$ restricted to $\Euc$ is equivalent to $L$, the functor taking $C$ to the the pre-sheaf with constant value $C$, which is left adjoint to $\Gamma$ restricted to $\Euc$.
\end{proof}
\begin{remark}\label{rmk:constsheaf}
  The proof of \cref{thm:leftadjoint} shows that, given an object $C\in\C$ and a dense sub-site $\B\subseteq\Mfd$ of \emph{contractible} manifolds, the constant pre-sheaf with value $C$ is equivalent to the sheaf $\flat C$ restricted to $\B$, hence it is already a sheaf. 
\end{remark}

\section{Chain complexes and sheaves}

Denote by $\Ch$ the category of chain complexes of abelian groups and $\Dr$ the associated derived category. Moreover, denote by $\Mod$ the category of $H\bZ$-module spectra. The equivalence of \cref{thm:stabledk} applied point-wise induces an equivalence of sheaf categories
\[\Shv(\Mfd,\Dr)\simeq\Shv(\Mfd,\Mod)\] 
\begin{definition}\label{def:EMsheaf}
  Given a sheaf $V:\Mfd^{\op}\to\Dr$, denote by $HV$ the corresponding sheaf of $H\bZ$-module spectra, called \emph{Eilenberg-Mac Lane sheaf}. 
\end{definition}
In this section, we introduce some technical lemmas regarding sheaves on manifolds valued in $\Dr$. We implicitly identify (sheaves of) cochain complexes $V^*$ with (sheaves of) chain complexes with reversed indexing, i.e. $V_n:=V^{-n}$.
\begin{definition}[{\cite[Definition 7.14]{bunkenikolausvoelkl2016diffcoh}}]Given $n\in\bZ$ and a (sheaf of) cochain complex $V^*$, denote by $V^{\geq n}$, resp. $V^{\leq n}$, the \emph{naive truncations} of $V^*$ \[\cdots\to0\to V^n\to V^{n+1}\to\cdots, \qquad \mbox{resp. }\cdots\to V^{n-1}\to V^n\to0\to\cdots\]
\end{definition}
Denote by $C^0$ the sheaf of {sets} associated to $\bR$ by the Yoneda embedding, mapping a manifold $M$ to the set of smooth maps $M\to\bR$. The sheaf $C^\infty$ is a ring object in $\Shv(\Mfd,\set)$, a $C^\infty$-module consisting of a sheaf $V$ together with a natural $C^\infty(M)$-module structure on $V(M)$, for every $M$.
\begin{lemma}[{\cite[Lemma 7.12]{bunkenikolausvoelkl2016diffcoh}}]\label{lem:sheafloc}
  Let $V:\Mfd^{\op}\to\Ch$ a sheaf of chain complexes of $C^\infty$-modules, then the composition $\Mfd^{\op}\xrightarrow{V}\Ch\to\Dr$ is a sheaf. 
\end{lemma}
\begin{definition}\label{def:forms}
  Denote by $\Omega^*$ the sheaf $\Mfd^{\op}\to\Ch$ mapping $M$ to its de Rham complex $\Omega^*(M)$. 
\end{definition}
\cref{lem:sheafloc} ensures that the sheaves in \cref{def:forms} is a sheaf after post-composition with the localization functor $\Ch\to\Dr$. 

\section{Higher de Rham theorem}

\begin{remark}
  Given a smooth manifold $M$, denote by $C_*^\sm(M,\bZ)$ the singular complex generated by $\Sing(M)$. Given an abelian group $A$, denote by $C^*_\sm(M,A)$ the singular cochain complex valued in $A$ associated to $C_*^\sm(M,\bZ)$. 
\end{remark}
The de Rham homomorphism is a natural chain homomorphism $\Omega^*\to C_\sm^*(-,\bR)$. By a theorem of de Rham, the induced morphism on cohomology is an isomorphism of graded commutative algebras. A stronger result, which we mention in this section, proves that the morphism of chain complexes is also compactible with the DG algebra structures on both complexes, but {up to coherent homotopies}. 
\begin{remark}\label{rmk:Dcotensor}
  Since $\Dr$ is presentable, we know that it is cotensored over $\s$. Given a space $S$ and a chain complex $V_*$, the cotensor $V_*{}^S$ is the cochain complex (viewed as a chain complex) of graded linear maps $C_*(S,\bZ)\to V_*$, from the (normalized) singular chain complex of $S$ to $V_*$, see \cite[Definition 1.3.2.1]{lurie2017ha}. In particular, let $V_*=V$ be concentrated in degree 0, then $V_*{}^S$ is the singular cochain complex of $S$ with values in $V$. 
\end{remark}
Recall the functor $\flat$ defined in \cref{def:flat}. By definition of $\Sing:\Mfd\to\s$ and \cref{rmk:Dcotensor}, given $V_*$ chain complex, $\flat V$ maps a manifold $M$ to $V_*{}^{\Sing(M)}$. 
\begin{definition}\label{def:derhammorf}
Consider the natural morphism $\Omega^*(M)\to (\flat\bR)(M)=C_\sm^*(M,\bR)$ taking a form $\omega\in\Omega^n(M)$ to the linear map $\int\omega:C^\sm_{n}(M,\bZ)\to\bR$. We call $\dr:\Omega^*\to\flat\bR$ the \emph{de Rham transformation}.  
\end{definition}
\begin{remark}\label{rmk:ainfmor}
  For the explicit data and conditions that determine an $A_\infty$-morphism between $A_\infty$-algebras, see \cite[Proposition 10.2.12]{loday2012algebraic}. Every $A_\infty$-homomorphism $X\to Y$, where $X$ and $Y$ are $A_\infty$-algebras in cochain complexes, has an underlying cochain momorphism. If $\phi:X\to Y$ is the underlying cochain morphism of an $A_\infty$-homomorphism, there is a specified morphism $\mu_2:X\otimes X\to Y$ such that $$\phi(vw)-\phi(v)\phi(w)=\mu_2(dv,w)+(-1)^{|v|}\mu_2(v,dw)-d\mu_2(v,w)$$for all $v,w\in X$. 
\end{remark}
\begin{theorem}[{\cite[Theorem 3.25]{abad2010ainftyrhamtheoremintegration}}]\label{thm:derham} 
  $\dr_M:\Omega^*(M)\to C^*_\sm(M,\bR)$ underlies an $A_\infty$-homomorphism of DG algebras. 
\end{theorem}

\section{Deligne Cohomology}

In this section, we give the definition of the $k$-th Deligne sheaf as the Eilenberg-Mac Lane sheaf (see \cref{def:EMsheaf}) associated to a sheaf $V(k):\Mfd^{\op}\to\Dr$. Following that, we give explicit cochain complexes that are quasi-isomorphic to the value of $V(k)$ at a manifold $M$. 
\begin{definition}\label{def:deligne}
  Given $k\in\bN$, define ${\bZ}(k):\Mfd^{\op}\to\Dr$ as the pullback 
  \[\begin{tikzcd}
    {\bZ}(k)\arrow[r]\arrow[d] & \Omega^{\geq k}\arrow[d,"\dr"] \\
    \flat\bZ \arrow[r] & \flat\bR
  \end{tikzcd}\]
  We call the corresponding Eilenberg-Mac Lane spectrum $H{\bZ}(k)$ the $k$\emph{-th Deligne sheaf}.
\end{definition}
\subsection{Model A}\label{mod:A} See {\cite[\S 3.2]{hopkinssinger2005diffcoh}}. Let $\bbA(k)^*(M)$ to the cochain complex where degree $n$ elements are triples $(c,h,\omega)\in C_\sm^n(M,\bZ)\oplus C_\sm^{n-1}(M,\bR)\oplus\Omega^n(M)$ for which $\omega=0$ if $n<k$, with differential $\delta(c,h,\omega)=(\delta c,\dr_M(\omega)-c-\delta h,d\omega)$. The complex $\bbA^*(k)(M)$ fits into a diagram
\[\begin{tikzcd}
    \bbA^*(k)(M)\arrow[d]\arrow[r] & \Omega^{\geq k}(M)\arrow[d] \\
    C_\sm^*(M,\bZ) \arrow[r] & C_\sm^*(M,\bR)
  \end{tikzcd}\]which commutes up to homotopy given by the projections $\bbA^n(k)(M)\to C_\sm^{n-1}(M,\bR)$. Applying the localization functor $\Ch\to\Dr$ to the above diagram we obtain a pullback square, hence $\bbA^*(k)(M)\simeq\bZ(k)(M)$. 

\subsection{Model B}\label{mod:B} Recall that every cofiber sequence is a fiber sequence in a stable $\infty$-category, and that $\flat$ preserves the zero object and pushout squares. Consider the diagram
\[
  \begin{tikzcd}
    {\bZ}(k)\arrow[r]\arrow[d] & \Omega^{\geq k}\arrow[d]\arrow[dd,bend left=5em,color=red] \\
    \flat\bZ \arrow[r]\arrow[d] & \flat\bR\arrow[d] \\
    0\arrow[r] & \flat(\bR/\bZ)
  \end{tikzcd}
\]
Since both inner squares are pullbacks, ${\bZ}(k)$ is equivalent to the fiber of the \textcolor{red}{red arrow}. Similar to \cref{mod:A}, consider a smooth manifold $M$, let $\bbB(k)^*(M)$ be the cochain complex where degree $n$ elements are pairs $(\chi,\omega)\in C_\sm^{n-1}(M,\bR/\bZ)\oplus\Omega^n(M)$ for which $\omega=0$ if $n<k$, with differential $\delta(\chi,\omega)=(e^{2\pi i\dr(\omega)}-\delta\chi,d\omega)$. Similar to \cref{mod:A}, the complex $\bbB(k)^*(M)$ fits into the above diagram so that the outer square is an pullback square in $\Dr$, hence it is equivalent to $\hat{\bZ}(k)(M)$.
\begin{definition}[{\cite[Definition 3.4]{hopkinssinger2005diffcoh}}, see also {\cite[Chapter 5]{barbecker2014diffchar}}] Consider a manifold $M$, a \emph{differential character} of degree $n$ consists of a character $\chi:Z^\sm_{n}(M,\bZ)\to\bR/\bZ$ on the group of smooth $n$-cycles of $M$ together with a closed $n$-form $\omega\in\Omega^{n+1}(M)$, such that, for every smooth $(n+1)$-chain $c$, \[\chi(\partial c)=e^{2\pi i\int_c\omega}\]
\end{definition}
It is immediate to see that a cocycle in $\bbB(k)^n(M)$ (for $n\geq k$) determines a differential character of degree $n$. 
\subsection{Interlude}\label{int:goodcover}
\begin{definition}
  Given a full sub-site $\B\subseteq\Mfd$ and a manifold $M$, a $\B$\emph{-good open cover} of $M$ consists of an open cover $\mathscr{O}$ such that every finite intersection of elements in $\mathscr{O}$ is either empty or diffeomorphic to an object of $\B$. If $\B=\Euc$ the sub-site of Euclidean spaces, we call a $\B$-good open cover a \emph{differentiably good open cover}, see \cite[Definition 6.3.9]{fiorenza2011cechcocyclesdifferentialcharacteristic}.
\end{definition}If $\B$ is the sub-site of contractible manifolds, we recover the notion of a \emph{good open cover}. Let $V$ be a sheaf on manifolds and $W$ a sheaf on $\B$ together with an equivalence $V|_\B\simeq W$, then $V$ can be evaluated on a manifold $M$ using $W$, provided that $M$ admits a $\B$-good open cover, since \[V(M)\simeq\lim_{U\in\I(\mathscr{O})}V(U)\simeq\lim_{U\in\I(\mathscr{O})}W(U)\]
If every manifolds admits a $\B$-good open cover, then $V$ is completely determined by $W$. Since we mainly care for $\B=\Euc$, we need the following result. 
\begin{theorem}[{\cite[Theorem A.1]{fiorenza2011cechcocyclesdifferentialcharacteristic}}]
  Every paracompact smooth manifold admits a differentiably good open cover. 
\end{theorem}
\subsection{Model C}\label{mod:C} See {\cite[Lemma 7.3.4]{amabeldebrayhaine2021diffcoh}}. Let $\B\subseteq\Mfd$ be a dense sub-site of contractible manifolds, consider the following diagram in the category of {sheaves} on $\B$
\[\begin{tikzcd}
     {\bZ}(k)|_\B\arrow[d]\arrow[r] & \Omega^{\geq k}|_\B\arrow[d] \arrow[r] & 0\arrow[d] \\
    \bZ \arrow[r]\arrow[rr,bend right=2.5em,color=red] & \Omega^*|_\B\arrow[r] & \Omega^{\leq k-1}|_\B
\end{tikzcd}\] The left square is obtained by: $\Circled{1}$ Apply the restriction to $\B$ functor to the pullback diagram of \cref{def:deligne} $\Circled{2}$ Substitute $\flat\bR|_\B$ with $\Omega^*|_\B$ (see \cref{thm:derham}) and $\flat\bZ|_\B$ with $\bZ$ (see \cref{rmk:constsheaf}). Since the right square is a pullback, ${\bZ}(k)|_\B$ is equivalent to the fiber of the \textcolor{red}{red arrow}. Let $\bC(k)^*$ be the sheaf of chain complexes $\bZ\to\Omega^0\to\cdots\to\Omega^{k-1}\to0\to\cdots$, where $\bZ$ is in degree 0 and $\bZ\hookrightarrow\Omega^0$ is the inclusion as constant functions. The complex $\bC(k)^*$ fits into the above diagram, so that the outer square is an pullback, therefore ${\bZ}(k)|_\B\simeq\bC(k)^*$. 

Since $\bZ(k)|_\B\simeq\bC(k)^*$, we can use the argument from \cref{int:goodcover} to conclude that, assuming $M$ has a $\B$-good open cover, 
\[
    {\bZ}(k)(M)\simeq\lim_{U\in\I(\mathscr O)} \bC(k)^*(U)
  \]
  The limit is then equivalent to the limit of the cosimplicial object $\bC(k)^{*,*}:\Delta\to\Dr$ defined as follows:
  \begin{enumerate}
    \item $\bC(k)^{*,n}$ is the product of all $\bC(k)^*(U_0\cap\cdots\cap U_n)$, ranging over the $(n+1)$-tuples of $\mathscr{O}^{n+1}$. 
    \item Given a monotone map $\alpha:\langle n\rangle\to\langle m\rangle$, define $\alpha^*:\bC(k)^{*,n}\to\bC(k)^{*,m}$ as follows: Given $\theta\in\bC(k)^{*,n}$, the $(U_0,\cdots,U_m)$-component of $\alpha^*(\theta)$ is the $(U_{\alpha(0)},\cdots,U_{\alpha(n)})$-component of $\theta$ restricted to $U_0\cap\cdots\cap U_m$.
  \end{enumerate}
  We can evaluate the limit by lifting $\bC(k)^{*,*}$ to a cosimplicial object in $\Ch$ and apply \cite[Lemma 7.10]{bunkenikolausvoelkl2016diffcoh}, the limit is then equivalent to (the image under localization $\Ch\to\Dr$ of) the total complex of the cochain bicomplex $\cdots\to 0\to \bC(k)^{*,0}\to\bC(k)^{*,1}\to\cdots$, with differential \[\delta^{*,n}=\sum_{\ell=0}^{n+1}(-1)^\ell(\ell\mbox{-th face map})^*:\bC(k)^{*,n}\to\bC(k)^{*,n+1}\] 

\section{Multiplicative structure on \texorpdfstring{$\bZ(k)$}{Z(k)}}

Consider the following diagram:
\begin{center}
  \begin{tikzcd}[column sep={2.5cm,between origins},row sep={1.5cm,between origins}]
    & \bZ(k)\otimes\bZ(\ell)\arrow[dr]\arrow[dl]\arrow[d,color=red] & \\
    \flat\bZ\otimes\flat\bZ\arrow[d,"\cup"]\arrow[dr] & \bZ(k+\ell)\arrow[dr]\arrow[dl] & \Omega^{\geq k}\otimes\Omega^{\geq\ell}\arrow[d,"\wedge"]\arrow[dl]\\
    \flat\bZ\arrow[dr] & \flat\bR\otimes\flat\bR\arrow[d,"\cup"] & \Omega^{\geq k+\ell}\arrow[dl]\\
    & \flat\bR & 
  \end{tikzcd}
\end{center}The vertical-outward facing-left square is commutes on the nose, while the vertical-outward facing-right square commutes up to homotopy (see \cref{thm:derham}). Since the horizontal-bottom square is a pullback, there is a contractible space of \textcolor{red}{red arrows} making the above cube commute. Using model A (see \cref{mod:A}) for $\bZ(k)$ and $\bZ(\ell)$, consider the following morphism $\mu:\bZ(k)\otimes\bZ(\ell)\to\bZ(k+\ell)$ \[(c,h,\omega)\otimes(b,u,\eta)\mapsto(c\cup b,h\cup \dr(\eta)+(-1)^{|c|}c\cup u+\mu_2(\omega,\eta),\omega\wedge\eta)\]where $\mu_2$ is the homotopy between $\dr(-)\cup\dr(-)$ and $\dr(-\wedge-)$, see \cref{thm:derham} and \cref{rmk:ainfmor}. One can check that $\mu$ fits into the above diagram, making it commute up to homotopy. 


{\footnotesize
\bibliographystyle{alpha}
\bibliography{main}
}


\end{document}