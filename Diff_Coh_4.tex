\documentclass[10pt]{amsart}
\usepackage{amsmath,amsthm,amssymb,amsfonts}
\usepackage[mathscr]{euscript}
\usepackage{tikz}
\usepackage{tikz-cd}
\usepackage{enumitem}
\usepackage[colorlinks=true, linkcolor=red, citecolor = blue]{hyperref}
\usepackage[margin=2.5cm]{geometry}
\setlength{\marginparwidth}{2cm}

\usepackage[nameinlink,capitalise,noabbrev]{cleveref}

\usepackage[textwidth=2cm, textsize=small, colorinlistoftodos]{todonotes}


\newcommand{\C}{\mathscr{C}}
\newcommand{\bC}{\mathbb{C}}
\newcommand{\kC}{\mathfrak{C}}
\newcommand{\D}{\mathscr{D}}
\newcommand{\E}{\mathscr{E}}
\newcommand{\bE}{\mathbb{E}}
\newcommand{\F}{\mathscr{F}}
\newcommand{\sL}{\mathscr{L}}
\newcommand{\bN}{\mathbb{N}}
\newcommand{\I}{\mathscr{I}}
\newcommand{\s}{\mathscr{S}}
\newcommand{\bR}{\mathbb{R}}
\newcommand{\bS}{\mathbb{S}}
\newcommand{\bZ}{\mathbb{Z}}


\newcommand{\Map}{\mathrm{Map}}
\newcommand{\Hom}{\mathrm{Hom}}
\newcommand{\Ho}{\mathrm{Ho}}
\newcommand{\set}{\mathscr{S}\mathrm{et}}
\newcommand{\Sp}{\mathscr{S}\mathrm{p}}
\newcommand{\Ch}{\mathrm{Ch}}
\newcommand{\cat}{\mathscr{C}\mathrm{at}}
\newcommand{\scat}{s\mathscr{C}\mathrm{at}}
\newcommand{\sset}{s\mathscr{S}\mathrm{et}}
\newcommand{\Fun}{\mathrm{Fun}}
\newcommand{\Nat}{\mathrm{Nat}}
\newcommand{\colim}{\mathrm{colim}}
\newcommand{\Top}{\mathscr{T}\mathrm{op}}
\newcommand{\Grp}{\mathscr{G}\mathrm{rp}}
\newcommand{\Euc}{\mathscr{E}\mathrm{uc}}
\newcommand{\Mfd}{\mathscr{M}\mathrm{fd}}
\newcommand{\Kan}{\mathscr{K}\mathrm{an}}
\newcommand{\Mod}{\mathrm{Mod}}
\newcommand{\Ab}{\mathscr{A}\mathrm{b}}
\newcommand{\Shv}{\mathscr{S}\mathrm{hv}}
\newcommand{\Yon}{\mathscr{Y}\mathrm{on}}
\newcommand{\Open}{\mathscr{O}\mathrm{pen}}
\newcommand{\PSh}{\mathscr{P}\mathscr{S}\mathrm{h}}
\newcommand{\cn}{\mathrm{cn}}

\newcommand{\bbefamily}{\fontencoding{U}\fontfamily{bbold}\selectfont}
\newcommand{\textbbe}[1]{{\bbefamily #1}}
\DeclareMathAlphabet{\mathbbe}{U}{bbold}{m}{n}

\def\DDelta{{\mathbbe{\Delta}}}
\newcommand{\DD}{\DDelta}

\newcommand{\adjun}[4]{
\begin{tikzcd}[row sep=0.5in, column sep=0.5in]
 #1  \arrow[r, shift left=1.8, "#3"] \pgfmatrixnextcell
 #2 \arrow[l, shift left=1.6, "#4", "\bot"'] 
\end{tikzcd}
}

\newcommand{\simpset}[7]{
 \begin{tikzcd}[row sep=0.5in, column sep=0.5in]
   #1 \arrow[r, shorten >=1ex,shorten <=1ex]
   \pgfmatrixnextcell #2 
   \arrow[l, shift left=1.2, "#5"] \arrow[l, shift right=1.2, "#4"'] 
   \arrow[r, shift right, shorten >=1ex,shorten <=1ex ] \arrow[r, shift left, shorten >=1ex,shorten <=1ex] 
   \pgfmatrixnextcell #3 
   \arrow[l] \arrow[l, shift left=2, "#7"] \arrow[l, shift right=2, "#6 "'] Top
   \arrow[r, shorten >=1ex,shorten <=1ex] \arrow[r, shift left=2, shorten >=1ex,shorten <=1ex] \arrow[r, shift right=2, 
   shorten >=1ex,shorten <=1ex]
   \pgfmatrixnextcell \cdots 
   \arrow[l, shift right=1] \arrow[l, shift left=1] \arrow[l, shift right=3] \arrow[l, shift left=3] 
 \end{tikzcd}
}


%% N.R. notes
\newcommand{\nrnote}[1]{\todo[color=green!40,linecolor=green!40!black,size=\tiny]{#1}}
\newcommand{\nrmpar}[1]{\todo[noline,color=green!40,linecolor=green!40!black,
  size=\tiny]{#1}}
\newcommand{\nrnoteil}[1]{\ \todo[inline,color=green!40,linecolor=green!40!black,size=\normalsize]{#1}}

\newtheorem{theorem}[equation]{Theorem}
\newtheorem{lemma}[equation]{Lemma}
\newtheorem{proposition}[equation]{Proposition}
\newtheorem{corollary}[equation]{Corollary}
% \newtheorem{statement}[section]{Statement}

\theoremstyle{definition}
\newtheorem{definition}[equation]{Definition}
\newtheorem{example}[equation]{Example}
% \newtheorem{attone}[equation]{Attention}

\theoremstyle{remark}
\newtheorem{remark}[equation]{Remark}
% \newtheorem{intone}[equation]{Intuition}
\newtheorem{notation}[equation]{Notation}
% \newtheorem{queone}[equation]{Question}
% \newtheorem{conjone}[equation]{Conjecture}
\newtheorem{warning}[equation]{Warning}

\title{Differential Cohomology Seminar 4 (Draft)}
\date{03.06.2025 $\&$ 17.06.2025}
\author{Talk by Alessandro Nanto}

\begin{document}
\maketitle

In the last talk we learned the definition of a differential cohomology theory, as a sheaf valued in spectra on the site of manifolds. This talk continues our journey through differential cohomology theories, and focuses on the following three topics:
\begin{enumerate}
  \item We want to learn how to construct non-trivial examples out of sheaves valued in chain complexes.
  \item We want to understand how we can extend classical cohomology operations to the setting of differential cohomology theories.
  \item We want to introduce suitable analogues of fiber-wise integration.  
\end{enumerate}

\section{Abelian Groups, Spectra and the Heart}
Let us start by reviewing the relation between abelian groups, rings and spectra. 
\begin{definition}
    Let $n\in\bZ$ and $X$ be a spectrum, define $\pi_n(X):=\pi_0(\Omega^{\infty+n}X)=\pi_0(X_{-n})$. We call $\pi_n$ the \textit{$n$-th homotopy group}\footnote{Since $X_n\simeq\Omega^2X_{n+2}$, for any $n$, the set $\pi_0(X_n)$ underlies the structure of an abelian group.} of $X$. 
  \end{definition}
  The category $\Sp$ underlies the structure of a symmetric monoidal $\infty$-category (\cite[Corollary 4.8.2.19]{lurie2017ha}). Following \cite{lurie2017ha}, we denote by $\otimes$ the tensor product on $\Sp$.
  \begin{definition}A commutative algebra object in $\Sp$ is called an \emph{$\bE_\infty$-ring spectrum}, see \cite[Definition 7.1.0.1]{lurie2017ha}. Given an $\bE_\infty$-ring spectrum $R$, denote by $\Mod_R$ the corresponding category of \emph{left $R$-module spectra}, see \cite[Definition 7.1.1.2]{lurie2017ha}. 
  \end{definition}
  \begin{remark}
    The sphere spectrum $\bS$ acts as the monoidal unit of $\Sp$, therefore it is a $\bE_\infty$-ring spectrum. The category $\Mod_\bS$ is canonically equivalent to $\Sp$. 
  \end{remark}
    \begin{definition}
    Denote by $\Sp_{\geq0}\subseteq\Sp$ the full sub-category generated by \emph{connective spectra}, i.e. spectra $X$ such that $\pi_n(X)\simeq0$, for all $n<0$. Denote by $\Sp^\heartsuit\subseteq\Sp_{\geq0}$ the \emph{heart of spectra}, i.e. the full sub-category generated by spectra $X$ such that $\pi_n(X)\simeq0$, for all $n>0$.
  \end{definition}
  The category $\Sp_{\leq0}$ is presentable and $\pi_0$ induces an equivalence between the heart and $\Ab$ (\cite[Proposition 1.4.3.6]{lurie2017ha}). The heart is a sub-category of local objects of connective spectra\footnote{A connective spectrum $X$ belongs to the heart if and only if $\pi_n(\Omega^\infty X)=0$, for all $n>0$, which is equivalent to $\Map_{\s_*}(S,\Omega^\infty X)\simeq0$, for all connected, pointed spaces $S$. Using the adjunction $(\Sigma^\infty,\Omega^\infty)$, we can conclude $X$ belongs to the heart if and only if $X$ is local with respect to class of maps $\Sigma^\infty S\to0$, for every $S$ connected, pointed space.}, therefore the inclusion $\Ab\simeq\Sp^\heartsuit\subseteq\Sp_{\geq0}$ is a right adjoint. The category $\Sp_{\geq0}$ is closed under $\otimes$ and, given $X,Y$ connective spectra, 
\begin{align}\label{eq}
  \pi_0(X\otimes Y)\simeq\pi_0(X)\otimes\pi_0(Y)
\end{align}
(see \hyperlink{https://www.dropbox.com/scl/fi/j73uavecb4gxkmb6uk80b/ATII_script.pdf?rlkey=2v7ezb2ie3bmu8rlig3kwq7qa&e=1&st=pu7m5wgm&dl=0}{these notes} by Jack Davies, Theorem 2.3.28).
\begin{definition}Given an abelian group $A$, denote by $HA$ the (unique up to equivalence) spectrum of the heart such that $\pi_0(HA)\simeq A$. We call $HA$ the \emph{Eilenberg-Mac Lane spectrum} of $A$.
\end{definition}
Using \cref{eq}, one can prove $H$, viewed as a functor $\Ab\to\Sp$, is lax monoidal. In particular, if $R$ is a commutative ring, then $HR$ is a connective $\bE_\infty$-ring spectrum. 
\begin{definition}
  Given a commutative ring $R$, denote by $\Ch(R)=\Ch(\Mod_R)$ the ordinary category of unbounded chain complexes. Let $\D(R)$ be the $\infty$-localization of $\Ch(R)$ at the class of quasi-isomorphisms.  
\end{definition} 
\begin{theorem}
  Let $R$ be a commutative ring. The functor $\pi_0$ induces an equivalence between $\Mod_R$ and the heart of $HR$-module spectra, i.e. $HR$-modules $M$ such that the underlying spectrum belongs to the heart. This equivalence extends to a symmetric monoidal equivalence between $\D(R)$ and $\Mod_{HR}$.
\end{theorem}
\begin{proof}
  The first part is \cite[Proposition 7.1.1.13]{lurie2017ha}. The second is \cite[Theorem 7.1.2.13]{lurie2017ha}.
\end{proof}


\section{From Chain Complexes to Spectra via stable Dold-Kan}

\begin{definition}
  Let $\Omega^*$ be the sheaf $\Mfd^{op}\to\Ch(\bZ)$ mapping a manifold $M$ to its de Rham complex. 
\end{definition}

% TO BE CONTINUED

Moreover, by \cref{cor:rings}, the ring map $\bZ \to \bR$, gives us a map of ring spectra $H\bZ \to H\bR$. Ideally Deligne cohomology should be characterized as the pullback of some sort of truncated deRham complex along the map $H\bZ \to H\bR$. This requires a precise definition of the spectrum associated to the $k$-truncated de Rham complex $\Omega^{\leq k}$. For this we use advanced result from stable homotopy theory.

\begin{theorem}
Let $R$ be a ring. The functor $H\colon \Mod_{R} \to \Mod_{H(R)}$ lifts
\[
\begin{tikzcd}
 \Mod_R \arrow[r, "H"] \arrow[d] & \Mod_{H(R)} \\
 \D(R) \arrow[ur, dashed]
\end{tikzcd}
\]
where $\D(R)$ is the derived category of $R$-modules.
\end{theorem}

Recall that a $\bZ$-module is just an abelian group. Hence, applying this result to $R = \bZ$, we get the following corollary.

\begin{corollary}
  The functor $H\colon \Mod_{\bZ} \to \Mod_{H\bZ}$ lifts to a functor
  \[
  \begin{tikzcd}
    \Mod_{\bZ} \arrow[r, "H"] \arrow[d] & \Mod_{H\bZ} \\
    D(\bZ) \arrow[ur, dashed, "DK_{st}"']
  \end{tikzcd}
  \]
  We call this lift $D(\bZ)\to \Mod_{H\bZ}$ the \emph{stable Dold-Kan correspondence}.
\end{corollary}

One thing one might wonder is how this relates to the more classical Dold-Kan correspondence, which relates chain complexes of abelian groups to simplicial abelian groups. Let $\Ch^+$ be the category of bounded below chain complexes of abelian groups. The classical Dold-Kan correspondence gives us a functor
\[DK\colon\Ch^+ \to s\Ab \]
from bounded below chain complexes of abelian groups to simplicial abelian groups. However, every simplicial abelian group comes with an abelian group structure on a simplicial set, meaning it is in particular an $E_\infty$-group in spaces. This means we have a functor \nrnote{What do we know about the properties of this functor?}
\[s\Ab \to \Grp_{E_\infty}(\s) \]
However, $\Grp_{E_\infty}(\s)$ fully faithfully embeds in $\Sp$ as connected spectra. Composing all these functors, we get a functor 
\[DK\colon \Ch^+ \to \Sp,\]
which is fully faithful and recovers the classical Dold-Kan correspondence\nrnote{Why?}. The stable Dold-Kan correspondence is a lift of this functor to $\D(\bZ)$ i.e. 
\[
\begin{tikzcd}
  \Ch^+ \arrow[r, "DK"] \arrow[d, hookrightarrow] & \Sp \\
  D(\bZ) \arrow[ur, "DK_{st}"']
\end{tikzcd} 
\]

relates to the stable Dold-Kan correspondence.\nrnote{How?} Finally, we can now use stable Dold-Kan to get a functor of sheaves.

% \[
% \begin{tikzcd}
%  \Ch^+ \arrow[r] & s\Ab \arrow[r] & \s \\ 
%  & infinite loop spaces & E_\infty-Alg(\s) \arrow[u] \\ 
%  & \Sp 
% \end{tikzcd}
% \]

\begin{definition} \label{def:EMsheaf}
  Let 
  \[H \colon \Shv(Mfd; \D(\bZ)) \to \Shv(\Mfd;\Sp)\]
  denote the functor that post-composes a sheaf of chain complexes on manifolds with the stable Dold-Kan correspondence and then sheafifies.\nrnote{Do we really need sheafification? This might need some checking.} For a given sheaf of chain complexes $F$, we call the image the associated \emph{Eilenberg-MacLane sheaf}.
\end{definition}

\section{Deligne Cohomology as a Differential Cohomology Theory}
Now equipped with \cref{def:EMsheaf}, we can finally define Deligne cohomology as a differential cohomology theory.

\begin{definition}
  Let $k \geq 0$. The \emph{Deligne cohomology sheaf} $\E(k)$ is defined via the following pullback square in $\Shv(\Mfd; \Sp)$:
  \[
  \begin{tikzcd}
   \E(k) \arrow[r] \arrow[d] & H(\Omega_{dR}^{\leq k}) \arrow[d] \\
   H\bZ \arrow[r] & H \bR 
  \end{tikzcd}
  \]
  Here $H$ is the  Eilenberg-MacLane sheaf.
\end{definition}

% Here $\Omega_{dR}$ is the deRham complex and $\Omega_{dR}^{\leq k}$ is its truncation to degrees $0$ to $k$. Here  $H\Omega_{dR}^{\geq k}$ is a black box, which is meant to be a sheaf of spectra. Let us try to better understand it.


% We call this $DK$ because it generalizes the classical Dold-Kan correspondence. Recall that the normal Dold-Kan correspondence relates chain complexes of abelian groups and simplicial abelian groups. 


% This gives us a functor $\Ch^+ \to \Sp$, which is fully faithful and we can consider the regular Dold-Kan.\nrnote{Why?}

% Now, stable Dold-Kan is a lift from $D(\bZ)^+$ to $D(\bZ)$.

\begin{remark}
  If we take $k = \infty$, then the map $H(\Omega_{dR}) \to H\bR$ is an equivalence, meaning $\E(\infty)$ is equivalent to $H\bZ$ i.e. singular cohomology. On the other side, the individual $\E(k)$ are highly non-trivial and help classify many geometric invariants of interest (as we saw in the first talk). So, the $\E(k)$ are a non-trivial filtration of $H\bZ$ by differential cohomology theories, in the sense that there are map $\E(k+1) \to \E(k)$, the limit of which is $H\bZ$.
\end{remark}

% Note that the $\E(k)$ assemble into a filtered diagram the limit of which is $\E$, which is itself equivalent to $H\bZ$. So it is really those intermediate steps are interesting to us.  

\section{Cohomology Operations for Deligne Cohomology}
Now that we have a rigorous definition of Deligne cohomology, we can start to think about operations on it. First of all, we need a suitable monoidal structure. 

\begin{definition}
  Let $F, G$ be two differential cohomology theories. The \emph{monoidal product} $F \otimes G$ is defined as the sheafification of the presheaf $F \wedge G$, which is the point-wise wedge product of spectra.\nrnote{It is expected that sheafification is necessary, but example is missing.}
\end{definition}

Now, recall there is a map of differential forms
\[ \Omega^{\leq k} \otimes \Omega^{\leq m} \to \Omega^{\leq k + m},\]
which induces a map of differential cohomology theories
\[ \E(k) \otimes \E(m) \to \E(k+ m)\]
Ideally, we would like to describe such an operation in a very explicit manner, however, in the realm of spectra this can be very challenging. This suggests an alternative perspective.

\begin{definition}
  Let $\sL(k)$ be the sheaf of chain complexes defined as the pullback in $\Shv(\Mfd,D(\bZ))$ of the following diagram
\[
\begin{tikzcd}
  \sL(k) \arrow[d] \arrow[r] & \Omega^{\leq k} \arrow[d, "dR"]\\
 \bZ \arrow[r] &  \bR
\end{tikzcd},
\]
where $\bZ$ is the functor $M \mapsto C^\bullet(M,\bZ)$ and $\bR$ is the functor $M \mapsto C^\bullet(M,\bR)$
\end{definition}

\begin{remark} \label{rem:slk chains}
  We can explicitly describe the chain complex $\sL(k)$ as follows.
\[\sL(k)^n = \{(c, \omega, h) \in C^n(-\bZ) \oplus \Omega^n(-)\oplus C^{n-1}(-\bR) | \omega = 0 if n > k and c - dR(\omega) = dh\} \] 
  
\end{remark}
 
\begin{remark}
  We expect that $H\sL(k)$ in fact recovers $\E(k)$, meaning operations on $\sL(k)$ help us understand operations on Deligne cohomology.\nrnote{This needs to be checked.}
\end{remark}

Using the explicit description from \cref{rem:slk chains}, we can define an operation on $\sL(k)$ as follows: 
\[ (c_1, \omega_1, h_1) \otimes (c_2, \omega_2, h_2) = (c_1 \cup c_2, \omega_1 \wedge \omega_2, (-1)^{|c_1|}c_1 \cup h_2  + h_1 \cup \omega_2 + B(\omega_1,\omega_2)), \]
where 
\[dR(\omega_1) \cup dR (\omega_2) =- dR(\omega_1 \wedge \omega_2)  = dB(\omega_1,\omega_2) \] 
\begin{remark}
  Intuitively $B(\omega_1,\omega_2)$ measures the failure of $dR$ taking $\wedge$ to $\cup$.\nrnote{Is there a reasonable way to pick $B(\omega_1,\omega_2)$?}
\end{remark}

\begin{remark}
  Ideally we would expect this formula to be well-defined, meaning $(c_1, \omega_1, h_1) \otimes (c_2, \omega_2, h_2)$ should satisfy the conditions in \cref{rem:slk chains}. In general, this is only true if $c_1,\omega_2$ satisfy $dc_1 = d\omega_2 = 0$. In particular, it is well-defined at the level of cohomology classes, as any element is closed therein.
\end{remark}




{\footnotesize
\bibliographystyle{alpha}
\bibliography{main}
}
\end{document}