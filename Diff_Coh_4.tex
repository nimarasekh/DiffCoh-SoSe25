\documentclass[10pt]{amsart}
\usepackage{amsmath,amsthm,amssymb,amsfonts}
\usepackage[mathscr]{euscript}
\usepackage{tikz}
\usepackage{tikz-cd}
\usepackage{enumitem}
\usepackage[colorlinks=true, linkcolor=red, citecolor = blue]{hyperref}
\usepackage[margin=2.5cm]{geometry}
\setlength{\marginparwidth}{2cm}
\usepackage{circledsteps}

\usepackage[nameinlink,capitalise,noabbrev]{cleveref}

\usepackage[textwidth=2cm, textsize=small, colorinlistoftodos]{todonotes}

\newcommand{\C}{\mathscr{C}}
\newcommand{\bC}{\mathbb{C}}
\newcommand{\kC}{\mathfrak{C}}
\newcommand{\D}{\mathscr{D}}
\newcommand{\E}{\mathscr{E}}
\newcommand{\bE}{\mathbb{E}}
\newcommand{\F}{\mathscr{F}}
\newcommand{\sL}{\mathscr{L}}
\newcommand{\bN}{\mathbb{N}}
\newcommand{\mN}{\mathrm{N}}
\newcommand{\I}{\mathscr{I}}
\newcommand{\s}{\mathscr{S}}
\newcommand{\bR}{\mathbb{R}}
\newcommand{\bS}{\mathbb{S}}
\newcommand{\bZ}{\mathbb{Z}}


\newcommand{\Hom}{\mathrm{Hom}}
\newcommand{\Map}{\mathrm{Map}}
\newcommand{\Ho}{\mathrm{Ho}}
\newcommand{\set}{\mathscr{S}\mathrm{et}}
\newcommand{\Sp}{\mathscr{S}\mathrm{p}}
\newcommand{\Ch}{\mathrm{Ch}}
\newcommand{\cCh}{\mathrm{cCh}}
\newcommand{\cat}{\mathscr{C}\mathrm{at}}
\newcommand{\scat}{s\mathscr{C}\mathrm{at}}
\newcommand{\sset}{s\mathscr{S}\mathrm{et}}
\newcommand{\Fun}{\mathrm{Fun}}
\newcommand{\Nat}{\mathrm{Nat}}
\newcommand{\colim}{\mathrm{colim}}
\newcommand{\Top}{\mathscr{T}\mathrm{op}}
\newcommand{\Grp}{\mathscr{G}\mathrm{rp}}
\newcommand{\Euc}{\mathscr{E}\mathrm{uc}}
\newcommand{\Mfd}{\mathscr{M}\mathrm{fd}}
\newcommand{\Kan}{\mathscr{K}\mathrm{an}}
\newcommand{\Mod}{\mathrm{Mod}}
\newcommand{\Ab}{\mathscr{A}\mathrm{b}}
\newcommand{\Shv}{\mathscr{S}\mathrm{hv}}
\newcommand{\Yon}{\mathscr{Y}\mathrm{on}}
\newcommand{\Open}{\mathscr{O}\mathrm{pen}}
\newcommand{\PSh}{\mathscr{P}\mathscr{S}\mathrm{h}}
\newcommand{\dg}{\mathrm{dg}}
\newcommand{\Sing}{\mathrm{Sing}}

\newcommand{\bbefamily}{\fontencoding{U}\fontfamily{bbold}\selectfont}
\newcommand{\textbbe}[1]{{\bbefamily #1}}
\DeclareMathAlphabet{\mathbbe}{U}{bbold}{m}{n}

\def\DDelta{{\mathbbe{\Delta}}}
\newcommand{\DD}{\DDelta}

\newcommand{\adjun}[4]{
\begin{tikzcd}[row sep=0.5in, column sep=0.5in]
 #1  \arrow[r, shift left=1.8, "#3"] \pgfmatrixnextcell
 #2 \arrow[l, shift left=1.6, "#4", "\bot"'] 
\end{tikzcd}
}

\newcommand{\simpset}[7]{
 \begin{tikzcd}[row sep=0.5in, column sep=0.5in]
   #1 \arrow[r, shorten >=1ex,shorten <=1ex]
   \pgfmatrixnextcell #2 
   \arrow[l, shift left=1.2, "#5"] \arrow[l, shift right=1.2, "#4"'] 
   \arrow[r, shift right, shorten >=1ex,shorten <=1ex ] \arrow[r, shift left, shorten >=1ex,shorten <=1ex] 
   \pgfmatrixnextcell #3 
   \arrow[l] \arrow[l, shift left=2, "#7"] \arrow[l, shift right=2, "#6 "'] Top
   \arrow[r, shorten >=1ex,shorten <=1ex] \arrow[r, shift left=2, shorten >=1ex,shorten <=1ex] \arrow[r, shift right=2, 
   shorten >=1ex,shorten <=1ex]
   \pgfmatrixnextcell \cdots 
   \arrow[l, shift right=1] \arrow[l, shift left=1] \arrow[l, shift right=3] \arrow[l, shift left=3] 
 \end{tikzcd}
}


%% N.R. notes
\newcommand{\nrnote}[1]{\todo[color=green!40,linecolor=green!40!black,size=\tiny]{#1}}
\newcommand{\nrmpar}[1]{\todo[noline,color=green!40,linecolor=green!40!black,
  size=\tiny]{#1}}
\newcommand{\nrnoteil}[1]{\ \todo[inline,color=green!40,linecolor=green!40!black,size=\normalsize]{#1}}

\newtheorem{theorem}[equation]{Theorem}
\newtheorem{lemma}[equation]{Lemma}
\newtheorem{proposition}[equation]{Proposition}
\newtheorem{corollary}[equation]{Corollary}
% \newtheorem{statement}[section]{Statement}

\theoremstyle{definition}
\newtheorem{definition}[equation]{Definition}
\newtheorem{example}[equation]{Example}
% \newtheorem{attone}[section]{Attention}

\theoremstyle{remark}
\newtheorem{remark}[equation]{Remark}
% \newtheorem{intone}[section]{Intuition}
\newtheorem{notation}[equation]{Notation}
% \newtheorem{queone}[section]{Question}
% \newtheorem{conjone}[section]{Conjecture}
\newtheorem{warning}[equation]{Warning}

\numberwithin{equation}{section}

\title{Differential Cohomology Seminar 4 (Draft)}
\date{03.06.2025 $\&$ 17.06.2025}
\author{Talk by Alessandro Nanto}

\begin{document}

\maketitle

In the last talk we learned the definition of a differential cohomology theory, as a sheaf valued in spectra on the site of manifolds. This talk continues our journey through differential cohomology theories, and focuses on the following three topics:
\begin{enumerate}
  \item We want to learn how to construct non-trivial examples out of sheaves valued in chain complexes.
  \item We want to understand how we can extend classical cohomology operations to the setting of differential cohomology theories.
  \item We want to introduce suitable analogues of fiber-wise integration.  
\end{enumerate}

\section{Abelian Groups, Spectra and the Heart}
Let us start by reviewing the relation between abelian groups, rings and spectra. 
\begin{definition}
    Let $n\in\bZ$ and $X$ be a spectrum, define $\pi_n(X):=\pi_0(\Omega^{\infty+n}X)=\pi_0(X_{-n})$. We call $\pi_n$ the \textit{$n$-th homotopy group} of $X$. 
\end{definition}
\begin{remark}
 Note that since $X_n\simeq\Omega^2X_{n+2}$, for any $n$, the set $\pi_0(X_n)$ underlies the structure of an abelian group.
\end{remark}

  The category $\Sp$ underlies the structure of a symmetric monoidal $\infty$-category (\cite[Corollary 4.8.2.19]{lurie2017ha}). Following \cite{lurie2017ha}, we denote by $\otimes$ the tensor product on $\Sp$.
  \begin{definition}
    A commutative algebra object in $\Sp$ is called an \emph{$\bE_\infty$-ring spectrum}, see \cite[Definition 7.1.0.1]{lurie2017ha}. Given an $\bE_\infty$-ring spectrum $R$, denote by $\Mod_R$ the corresponding category of \emph{left $R$-module spectra}, see \cite[Definition 7.1.1.2]{lurie2017ha}. 
  \end{definition}
  \begin{remark}
    The sphere spectrum $\bS$ acts as the monoidal unit of $\Sp$, therefore it is a $\bE_\infty$-ring spectrum. The category $\Mod_{\bS}$ is canonically equivalent to $\Sp$. 
  \end{remark}
    \begin{definition}
    Denote by $\Sp_{\geq0}\subseteq\Sp$ the full sub-category generated by \emph{connective spectra}, i.e. spectra $X$ such that $\pi_n(X)\simeq0$, for all $n<0$. Denote by $\Sp^\heartsuit\subseteq\Sp_{\geq0}$ the \emph{heart of spectra}, i.e. the full sub-category generated by spectra $X$ such that $\pi_n(X)\simeq0$, for all $n>0$.
  \end{definition}
  We have the following result relating connective spectra and the heart, which follow immediately.
  \begin{lemma}
  Let $X$ be a connective spectrum. The following are equivalent:
  \begin{enumerate}
    \item $X$ is in the heart. 
    \item $\pi_n(\Omega^\infty X)=0$, for all $n>0$.
    \item $\Hom_{\s_*}(S,\Omega^\infty X)\simeq0$, for all connected, pointed spaces $S$.
    \item $X$ is local with respect to the class of maps $\Sigma^\infty S\to0$, for every connected pointed space $S$.
  \end{enumerate}
  \end{lemma}

  The category $\Sp_{\geq0}$ is presentable and $\pi_0$ induces an equivalence between the heart and $\Ab$ (\cite[Proposition 1.4.3.6]{lurie2017ha}). The heart is a sub-category of local objects of connective spectra, therefore the inclusion $\Ab\simeq\Sp^\heartsuit\subseteq\Sp_{\geq0}$ is a right adjoint. The category $\Sp_{\geq0}$ is closed under $\otimes$ and, given $X,Y$ connective spectra, 

\begin{align}\label{eq:stablehurewicz}
  \pi_0(X\otimes Y)\simeq\pi_0(X)\otimes\pi_0(Y)
\end{align}
see \cite[Theorem 2.3.28]{davies2024atii}

\begin{definition}Given an abelian group $A$, denote by $HA$ the (unique up to equivalence) spectrum of the heart such that $\pi_0(HA)\simeq A$. We call $HA$ the \emph{Eilenberg-Mac Lane spectrum} of $A$.
\end{definition}
Using \cref{eq:stablehurewicz}, one can prove $H$, viewed as a functor $\Ab\to\Sp$, is lax monoidal. In particular, if $R$ is a commutative ring, then $HR$ is a connective $\bE_\infty$-ring spectrum. On the other hand, if $R$ is a connective $\bE_\infty$-ring spectrum and $M$ a connective module, then $\pi_0(M)$ is a $\pi_0(R)$-module. 
\begin{definition}
  Given a commutative ring $R$, denote by $\Ch(R)=\Ch(\Mod_{R})$ the ordinary category of unbounded chain complexes. Let $\D(R)$ be the $\infty$-localization of $\Ch(R)$ at the class of quasi-isomorphisms.  
\end{definition} 
Similar to the heart of spectra, given an $\bE_\infty$-ring spectrum $R$, denote by $\Mod_{R}^\heartsuit\subseteq \Mod_{R}$ the full sub-category generated by $R$-modules such that the underlying spectrum belongs to the heart of spectra. 
\begin{theorem}[Stable Dold-Kan Correspondence]\label{thm:stabledk}
  Let $R$ be a commutative ring. 
  \begin{enumerate}
    \item $\Mod_{R}\simeq \Mod_{HR}^\heartsuit$ via taking Eilenberg-Mac Lane spectra.
    \item The equivalence in (1) extends to an equivalence $H:\D(R)\simeq \Mod_{HR}$ of symmetric monoidal $\infty$-categories.
  \end{enumerate} 
\end{theorem}
\begin{proof}
  (1) is \cite[Proposition 7.1.1.13]{lurie2017ha}, while (2) is \cite[Theorem 7.1.2.13]{lurie2017ha}.
\end{proof}
An interesting consequence of \cref{thm:stabledk} is the following:
\begin{corollary}\label{cor:homotopyvshomology}
  Given $F\in\D(R)$, then $\pi_n(HF)\simeq H_n(F)$, for all $n\in\bZ$.
\end{corollary}
\begin{proof}
  \begin{align*}
    \pi_n(HF) & =\pi_0(\Omega^{\infty+n}HF)\\
    & \overset{\Circled{1}}{\simeq} \pi_0(\Hom_{\Sp}(\Sigma^n\bS,HF)) \\
    & \overset{\Circled{2}}{\simeq} \pi_0(\Hom_{\Mod_{HR}}(\Sigma^n HR,HF)) & \\
    & \overset{\Circled{3}}{\simeq} \pi_0(\Hom_{\D(R)}(R[n],F))\\
    & \overset{\Circled{4}}{\simeq} H_n(F)
  \end{align*} \Circled{1} The functor $\Omega^{\infty+n}$ is corepresented by the shifted sphere spectrum $\Sigma^n\bS$. \Circled{2} The forgetful functor $\Mod_{HR}\to\Mod_\bS\simeq\Sp$ is right adjoint to tensoring by $HR$ and $HR\otimes(\Sigma^n\bS)\simeq \Sigma^nHR$. \Circled{3} \cref{thm:stabledk} \Circled{4} $\pi_0$ of the mapping space $\Hom_{\D(R)}(R[n],F)$ is equivalent to the mapping space $R[n]\to F$ in the \emph{ordinary} derived category of $R$, i.e. homotopy classes of maps $R[n]\to F$, which correspond exactly to classes in $H_n(F)$.
\end{proof}

\section{More \texorpdfstring{$\infty$}{oo}-categorical baggage}

Let $\C$ be a presentable $\infty$-category. The $\infty$-categorical background given in previous talks allows to conclude the existence of a number of functors. Here we give (somewhat) explicit formulas for one. 
\begin{remark}
  Recall $\Euc$, the full sub-category of $\Mfd$ generated by Euclidean manfiolds $\bR^n$, for every $n\geq0$. Denote by $j$ the inclusion functor $\Euc\subseteq\Mfd$. Recall that the restriction along $j$ induces an equivalence $\Shv(\Mfd,\C)\simeq\Shv(\Euc,\C)$, see \cite[Corollary A.5.6]{amabeldebrayhaine2021diffcoh}.
\end{remark}
Evaluation at $\{0\}$ induces an adjunction $(\Gamma^*,\Gamma_*):\C\to\Shv(\Mfd,\C)$. The functor $\Gamma_*$ is evaluation at $\{0\}$, while the left adjoint $\Gamma^*$ maps $C\in\C$ to the sheafification of the constant pre-sheaf with value $C$. 
\begin{remark}\label{rmk:cotensor}
Every presentable $\infty$-category $\C$ is \emph{cotensored over} $\s$, i.e. a functor $-^-:\C\to\Fun^R(\s^{op},\C)$ exists such that, for every $C',C\in\C$ and space $S$, there is an natural equivalence $$\Hom_\s(S,\Hom_\C(C',C))\simeq\Hom_\C(C',C^S)$$ see \cite[Remark 5.5.2.6]{lurie2009htt}.
\end{remark}
\begin{definition}
  Denote by $\Sing$ the functor $\Mfd\to\s$ mapping a manifold to its underlying space. Given a presentable $\infty$-category $\C$, denote by $\flat$ the composition $\C\to\Fun(\s^{op},\C)\to\Fun(\Mfd^{op},\C)$, the first functor coming from \cref{rmk:cotensor}, the second being pre-composition with $\Sing^{op}$.
\end{definition}
Explicitly, given an object $C\in\C$, the associated pre-sheaf $\flat C$ maps a manifold $M$ to $C^{\Sing(M)}$. 
\begin{lemma}[{\cite[Corollary 6.46]{bunkegepner2021differential}}]\label{lem:flatsheaf}
  The functor $\flat$ factors through $\Shv(\Mfd,\C)\subseteq\Fun(\Mfd^{op},\C)$.
\end{lemma}
\cref{lem:sheafloc} is the direct consequence of a weaker version of a generalized version of Seifert-van Kampen theorem, namely \cite[Proposition A.3.2]{lurie2017ha}, stating that, given a topological space $X$ and an open cover $\mathscr{O}$ closed under finite intersections, $\Sing(X)$ is the colimit over $U\in\mathscr{O}$ of $\Sing(U)$.
\begin{theorem}
  The functor $\flat:\C\to\Shv(\Mfd,\C)$ is left adjoint to the functor $\Gamma_*$.
\end{theorem}
\begin{proof}
  The composition $\C\xrightarrow{\flat}\Shv(\Mfd,\C)\xrightarrow{j_*}\Shv(\Euc,\C)$ maps an object $C$ to the sheaf $\flat C$ restricted to Euclidean spaces. Since $\bR^n$ is contractible, $(\flat C)(\bR^n)=C^{\Sing(\bR^n)}\simeq C$ and so $\flat$ restricted to $\Euc$ is equivalent to the functor taking $C$ to the the sheaf with constant value $C$, which is the left adjoint to $\Gamma_*$ restricted to $\Euc$.
\end{proof}

\section{Sheaves of complexes and spectra}

The stable Dold-Kan correspondence allowes us to move freely between sheaves of $H\bZ$-module spectras and sheaves valued in $\D(\bZ)$.

\begin{remark}\label{rmk:identification}
  We identify the category of cochain complexes with $\Ch(R)$ by reversing grading. Namely, given a cochain $V^*$, we are implicitly identifying it with the chain complex $V_n=V^{-n}$. 
\end{remark}
\begin{definition}[{\cite[Definition 7.14]{bunkenikolausvoelkl2016diffcoh}}]Given $n\in\bZ$, denote by $\sigma^{\geq n}$, resp. $\sigma^{\leq n}$, the \emph{naive truncation functors}, mapping a cochain complex $V^*$ to \[\cdots\to0\to V^n\to V^{n+1}\to\cdots\]
  resp. \[\cdots\to V^{n-1}\to V^n\to0\to\cdots\]Given $F:\Mfd^{op}\to\Ch(\bZ)$ and $\sharp\in\{\geq n,\leq n\}$, denote by $F^\sharp$ the composite $\Mfd^{op}\to\Ch(\bZ)\xrightarrow{\sigma^\sharp}\Ch(\bZ)$. Notice that if $F$ is a sheaf, then $F^\sharp$ is also a sheaf.
\end{definition}
\begin{lemma}[{\cite[Lemma 7.12]{bunkenikolausvoelkl2016diffcoh}}]\label{lem:sheafloc}
  Let $F:\Mfd^{op}\to\Ch(\bZ)$ a sheaf of chain complexes of $C^\infty$-modules, then $\Mfd^{op}\xrightarrow{F}\Ch(\bZ)\to\D(\bZ)$ is a sheaf. 
\end{lemma}
\begin{definition}\label{def:forms}
  Denote by $\Omega^*$ the sheaf $\Mfd^{op}\to\Ch(\bZ)$ mapping a manifold to its de Rham complex. 
\end{definition}
\cref{lem:sheafloc} ensures that the sheaf in \cref{def:forms} and the corresponding naive truncations remain sheaves after post-composition with the localization functor $\Ch(\bZ)\to\D(\bZ)$.  
\begin{definition}\label{def:EMsheaf}
  Given a sheaf $F:\Mfd^{op}\to\D(\bZ)$, denote by $HF$ the \emph{Eilenberg-Mac Lane sheaf} of $H\bZ$-module spectra obtained by applying point-wise the equivalence of \cref{thm:stabledk}. 
\end{definition}

% $\Omega^*$ is equivalent to the constant sheaf $\bR$.

% TO BE CONTINUED 

\section{Deligne Cohomology}


\begin{definition}
  Given $n\in\bN$, define $\widehat{\bZ}(n):\Mfd^{op}\to\D(\bZ)$ by the pullback 
  \[\begin{tikzcd}
    \widehat{\bZ}(n)\arrow[r]\arrow[d] & \Omega^{\geq n}\arrow[d] \\
    \bZ \arrow[r] & \Omega^*
  \end{tikzcd}\]We call the corresponding sheaf of $H\bZ$-modules spectra $H\widehat{\bZ}(n)$ the $n$\emph{-th Deligne sheaf}.
\end{definition}

% $\bE_\infty$-algebra structure on $\widehat{\bZ}(n)$ coming from the pullback square. 

\section{Unfolding the fracture square of Deligne Cohomology}
Now that we have a rigorous definition of Deligne cohomology, we can start to think about operations on it. First of all, we need a suitable monoidal structure. 

\begin{definition}
  Let $F, G$ be two differential cohomology theories. The \emph{monoidal product} $F \otimes G$ is defined as the sheafification of the presheaf $F \wedge G$, which is the point-wise wedge product of spectra.\nrnote{It is expected that sheafification is necessary, but example is missing.}
\end{definition}

Now, recall there is a Hom of differential forms
\[ \Omega^{\leq k} \otimes \Omega^{\leq m} \to \Omega^{\leq k + m},\]
which induces a Hom of differential cohomology theories
\[ \E(k) \otimes \E(m) \to \E(k+ m)\]
Ideally, we would like to describe such an operation in a very explicit manner, however, in the realm of spectra this can be very challenging. This suggests an alternative perspective.

\begin{definition}
  Let $\sL(k)$ be the sheaf of chain complexes defined as the pullback in $\Shv(\Mfd,D(\bZ))$ of the following diagram
\[
\begin{tikzcd}
  \sL(k) \arrow[d] \arrow[r] & \Omega^{\leq k} \arrow[d, "dR"]\\
 \bZ \arrow[r] &  \bR
\end{tikzcd},
\]
where $\bZ$ is the functor $M \mapsto C^\bullet(M,\bZ)$ and $\bR$ is the functor $M \mapsto C^\bullet(M,\bR)$
\end{definition}

\begin{remark} \label{rem:slk chains}
  We can explicitly describe the chain complex $\sL(k)$ as follows.
\[\sL(k)^n = \{(c, \omega, h) \in C^n(-\bZ) \oplus \Omega^n(-)\oplus C^{n-1}(-\bR) | \omega = 0 if n > k and c - dR(\omega) = dh\} \] 
  
\end{remark}
 
\begin{remark}
  We expect that $H\sL(k)$ in fact recovers $\E(k)$, meaning operations on $\sL(k)$ help us understand operations on Deligne cohomology.\nrnote{This needs to be checked.}
\end{remark}

Using the explicit description from \cref{rem:slk chains}, we can define an operation on $\sL(k)$ as follows: 
\[ (c_1, \omega_1, h_1) \otimes (c_2, \omega_2, h_2) = (c_1 \cup c_2, \omega_1 \wedge \omega_2, (-1)^{|c_1|}c_1 \cup h_2  + h_1 \cup \omega_2 + B(\omega_1,\omega_2)), \]
where 
\[dR(\omega_1) \cup dR (\omega_2) =- dR(\omega_1 \wedge \omega_2)  = dB(\omega_1,\omega_2) \] 
\begin{remark}
  Intuitively $B(\omega_1,\omega_2)$ measures the failure of $dR$ taking $\wedge$ to $\cup$.\nrnote{Is there a reasonable way to pick $B(\omega_1,\omega_2)$?}
\end{remark}

\begin{remark}
  Ideally we would expect this formula to be well-defined, meaning $(c_1, \omega_1, h_1) \otimes (c_2, \omega_2, h_2)$ should satisfy the conditions in \cref{rem:slk chains}. In general, this is only true if $c_1,\omega_2$ satisfy $dc_1 = d\omega_2 = 0$. In particular, it is well-defined at the level of cohomology classes, as any element is closed therein.
\end{remark}




{\footnotesize
\bibliographystyle{alpha}
\bibliography{main}
}
\end{document}