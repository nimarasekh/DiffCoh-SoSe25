\documentclass[10pt]{amsart}
\usepackage{amsmath,amsthm,amssymb,amsfonts}
\usepackage[mathscr]{euscript}
\usepackage{tikz}
\usepackage{tikz-cd}
\usepackage{enumerate}
\usepackage{enumitem}
\usepackage{mathtools}
\usepackage[colorlinks=true, linkcolor=red, citecolor = blue]{hyperref}
\usepackage[margin=2.5cm]{geometry}
\setlength{\marginparwidth}{2cm}

\usepackage[nameinlink,capitalise,noabbrev]{cleveref}

\usepackage[textwidth=2cm, textsize=small, colorinlistoftodos]{todonotes}

\newcommand{\A}{\mathscr{A}}
\newcommand{\cA}{\mathcal{A}}
\newcommand{\B}{\mathscr{B}}
\newcommand{\cB}{\mathcal{B}}
\newcommand{\sC}{\mathscr{C}}
\newcommand{\sD}{\mathscr{D}}
\newcommand{\sE}{\mathscr{E}}
\newcommand{\cF}{\mathcal{F}}
\newcommand{\sG}{\mathscr{G}}
\newcommand{\cG}{\mathcal{G}}
\newcommand{\cH}{\mathcal{H}}
\newcommand{\cL}{\mathcal{L}}
\newcommand{\sS}{\mathscr{S}}
\newcommand{\bS}{\mathbb{S}}
\newcommand{\bR}{\mathbb{R}}
\newcommand{\bZ}{\mathbb{Z}}
\newcommand{\bC}{\mathbb{C}}
\newcommand{\fO}{\mathbf{O}}
\newcommand{\rO}{\mathscr{O}}
\newcommand{\M}{\mathcal{M}}
\newcommand{\bQ}{\mathbb{Q}}

\newcommand{\aff}{\mathrm{Aff}}
\newcommand{\Cyc}{\mathrm{Cyc}}
\newcommand{\tmf}{\mathrm{tmf}}
\newcommand{\TMF}{\mathrm{TMF}}
\newcommand{\Tmf}{\mathrm{Tmf}}
\newcommand{\Def}{\mathrm{Def}}
\newcommand{\curv}{\mathrm{curv}}
\newcommand{\CS}{\mathrm{CS}}
\newcommand{\ch}{\mathrm{ch}}
\newcommand{\dKU}{\smash{\widehat{ku}}}
\newcommand{\dKUnabla}{\smash{\widehat{ku}^\nabla}}
\renewcommand{\sp}{\mathrm{sp}}

\DeclareMathOperator{\tr}{tr}
\newcommand{\Map}{\mathrm{Map}}
\newcommand{\uMap}{\underline{\mathrm{Map}}}
\newcommand{\Hom}{\mathrm{Hom}}
\newcommand{\Ho}{\mathrm{ho}}
\newcommand{\set}{\mathscr{S}\mathrm{et}}
\newcommand{\CMon}{\mathrm{CMon}}
\newcommand{\CGrp}{{\normalfont\texttt{CMon}}}
\newcommand{\fib}{{\normalfont\texttt{fib}}}
\newcommand{\cofib}{{\normalfont\texttt{cofib}}}
\newcommand{\Bun}{{\normalfont\texttt{Bun}}}
\newcommand{\Sp}{\mathscr{S}\mathrm{p}}
\newcommand{\Ch}{\mathscr{C}\mathrm{h}}
\newcommand{\cat}{\mathscr{C}\mathrm{at}}
\newcommand{\scat}{s\mathscr{C}\mathrm{at}}
\newcommand{\sset}{s\mathscr{S}\mathrm{et}}
\newcommand{\Line}{\mathscr{L}\mathrm{ine}}
\newcommand{\Fun}{\mathrm{Fun}}
\newcommand{\Nat}{\mathrm{Nat}}
\newcommand{\colim}{\mathrm{colim}}
\newcommand{\Top}{\mathscr{T}\mathrm{op}}
\newcommand{\Grpd}{\mathscr{G}\mathrm{rpd}}
\newcommand{\Grp}{\mathscr{G}\mathrm{rp}}
\newcommand{\Euc}{\mathscr{E}\mathrm{uc}}
\newcommand{\Mfd}{\mathscr{M}\mathrm{fd}}
\newcommand{\Kan}{\mathscr{K}\mathrm{an}}
\newcommand{\Vect}{\mathscr{V}\mathrm{ect}}
\newcommand{\Mod}{\mathscr{M}\mathrm{od}}
\newcommand{\Proj}{\mathscr{P}\mathrm{roj}}
\newcommand{\Ab}{\mathscr{A}\mathrm{b}}
\newcommand{\Shv}{\mathscr{S}\mathrm{hv}}
\newcommand{\Yon}{\mathscr{Y}\mathrm{on}}
\newcommand{\Open}{\mathscr{O}\mathrm{pen}}
\newcommand{\PSh}{\mathscr{P}\mathscr{S}\mathrm{h}}
\newcommand{\Pic}{\mathscr{P}\mathrm{ic}}
\newcommand{\triv}{\mathscr{T}\mathrm{riv}}
\newcommand{\aut}{\mathrm{Aut}}
\newcommand{\Th}{\mathrm{Th}}
\newcommand{\fr}{\mathrm{Fr}}
\newcommand{\Arr}{\mathrm{Arr}}
\newcommand{\ev}{\mathrm{ev}}
\newcommand{\dis}{\mathrm{dis}}
\newcommand{\Mon}{\mathrm{Mon}}
\newcommand{\Rep}{\mathrm{Rep}}
\newcommand{\PT}{\mathrm{PT}}
\newcommand{\Fred}{\mathcal{F}\mathrm{red}}
\newcommand{\inv}{\mathrm{inv}}
\newcommand{\const}{\mathrm{const}}
\newcommand{\spin}{\mathrm{spin}}
\newcommand{\scal}{\mathrm{scal}}
\newcommand{\HS}{\mathcal{H}\mathcal{S}}
\newcommand{\PU}{\mathcal{P}\mathcal{U}}

\newcommand{\bbefamily}{\fontencoding{U}\fontfamily{bbold}\selectfont}
\newcommand{\textbbe}[1]{{\bbefamily #1}}
\DeclareMathAlphabet{\mathbbe}{U}{bbold}{m}{n}

\def\DDelta{{\mathbbe{\Delta}}}
\newcommand{\DD}{\DDelta}



%% N.R. notes
\newcommand{\nrnote}[1]{\todo[color=green!40,linecolor=green!40!black,size=\tiny]{#1}}
\newcommand{\nrmpar}[1]{\todo[noline,color=green!40,linecolor=green!40!black,
  size=\tiny]{#1}}
\newcommand{\nrnoteil}[1]{\ \todo[inline,color=green!40,linecolor=green!40!black,size=\normalsize]{#1}}

\newtheorem{theorem}[equation]{Theorem}
\newtheorem{lemma}[equation]{Lemma}
\newtheorem{proposition}[equation]{Proposition}
\newtheorem{corollary}[equation]{Corollary}
% \newtheorem{statement}[section]{Statement}

\theoremstyle{definition}
\newtheorem{definition}[equation]{Definition}
\newtheorem{example}[equation]{Example}
% \newtheorem{attone}[equation]{Attention}

\theoremstyle{remark}
\newtheorem{remark}[equation]{Remark}
% \newtheorem{intone}[equation]{Intuition}
\newtheorem{notation}[equation]{Notation}
% \newtheorem{queone}[equation]{Question}
% \newtheorem{conjone}[equation]{Conjecture}
\newtheorem{warning}[equation]{Warning}

\numberwithin{equation}{section}

\title{Differential Cohomology Seminar 10}
\date{17.12.2025}
\author{Talk by Konrad Waldorf}

\begin{document}
\maketitle

Today we look at the construction of pushforward maps in twisted K-theory due to Carey and Wang \cite{careywang2008thomisomorphism}.

\section{The Classical Case}
Let us do a quick review of the classical case. Given a map $f\colon X \to Y$ and induced map on K-theory $f^*\colon K(Y) \to K(X)$, we would like to construct a pushforward map $f_!\colon K(X) \to K(Y)$. In the classical case, this requires $w_2(X)	= f^* w_2(Y)$. This precisely corresponds to choosing a $\spin^c$-structure on the virtual bundle $TX - f^* TY$. This is the necessary condition to define this map.

\section{Reviewing twisted K-Theory}
Before we proceed to generalize the pushforward map to twisted K-theory, we first review the definition of twisted K-theory. Recall that 
\[ H^3(M,\bZ) \cong \PU(\cH) - \Bun(M)/ \sim \] 
where $\cH$ is a separable Hilbert space and $\PU(\cH)$ is the projective unitary group. Choosing $P$ a principal $\PU(\cH)$-bundle over $M$ representing a class in $H^3(M,\bZ)$, we can form the associated bundle of Fredholm operators $\Fred(P) := P \times_{\PU(\cH)} \Fred(\cH)$. Then the $0$-th twisted K-theory is defined as
\[K^0(M,P) := \pi_0(\Map(M, P\times_{\PU(\cH)} \Fred(\cH))) \cong \pi_0(\Map(P, \Fred(\cH)))_{\PU(\cH)}. \]
More generally 
\[ K^n(M,P) := \pi_0(\Map(P, \Omega^n \Fred(\cH)))_{\PU(\cH)}. \]

In the coming sections we consider the pushforward map in two cases: the torsion case and the non-torsion case.

\section{Understanding the Torsion Case}
Let us consider the case when the bundle is torsion. Then $P$ is a principal $\PU(n)$-bundle, using $U(n) \hookrightarrow	U(\cH)$. In this case we can obtain a bundle gerbe $\cL_P$, which has the following form 
 \[ 
	\begin{tikzcd}
	U(n) \arrow[d] & \Gamma_P \arrow[d] \arrow[l] & E \arrow[d] \\ 
	\PU(n) & P \times_M P \arrow[l] \arrow[r, shift left = 1.8] \arrow[r, shift right = 1.8] & P \arrow[d] \\ 
	 & & M  
	\end{tikzcd}
	\]  
	This structure gives us a map 
	\[ 
	\phi\colon (\Gamma_P|_{p_1,p_2} \times_{U(1)} \bC) \otimes E_{p_2} \to E_{p_1}
	\]
	meaning $(E,\phi)$	is a $\cL_P$-twisted vector bundle.	We now can prove	the following theorem.

	\begin{theorem}
		Let $P$ be a principal $\PU(n)$-bundle. Then, we have an isomorphism $K(\cL-\Mod) \cong K_{U(n), \scal}(P)$.
	\end{theorem}

	Here $K_{U(n), \scal}(P)$ is the $U(n)$-equivariant K-theory of $P$ with scalar action of the center $U(1) \subset U(n)$. The proof proceeds by mapping a pair $(E,\phi)$ to the data $(U(n) \times E \to E)$ that maps $(A,v) \mapsto \phi(A,v)$.

	We now have the following second theorem, relating twisted K-theory to twisted vector bundles.

	\begin{theorem}
	 Let $P$ be a principal $\PU(n)$-bundle. Then we have an isomorphism
	 \[ K^0(M,P) \cong K(\cL_P-\Mod). \]
	\end{theorem}

	Here we are implicitly using the $\PU(\cH)$-bundle $\tilde{P} = P \times_{\PU(n)} \PU(\cH)$, meaning 
	\[K^0(M,P) = K^0(M,\tilde{P}) = \pi_0(\Map(\tilde{P}, \Fred(\cH)))_{\PU(\cH)}.\] 
Now we also have $\Fred(\cH)	\cong \Fred_{U(n)-cts}(\cH)$, meaning it is the Fredholm operators with continuous $U(n)$-action.

Now before we proceed, recall that we have a an equivariant version of Atiyah-J\"anich theorem (which holds if certain conditions are satisfied), meaning 
\[ [M,\Fred_{G-cts}(\cH)]_G \cong K^0(\Vect_G(M)). \]

We can now restrict this bijection to a bijection of subsets 
\[[M,\Fred_{U(n)-cts}(\cH^{\scal})]_G \cong K^0(\Vect_{U(n),\scal}(M)), \]
where $\cH^{\scal}$ is the Hilbert space with scalar action of the center $U(1) \subset U(n)$. 

The construction proceeds now by replacing $\cH$ with $\cH^{\scal}$ in the definition of twisted K-theory, meaning we go back to the first step and do 
	\[K^0(M,P) = K^0(M,\tilde{P}) = \pi_0(\Map(\tilde{P}, \Fred(\cH^{\scal})))_{\PU(\cH)}.\]
	which via the bijection above is isomorphic to $K_{U(n),\scal}(P)$, which by the previous theorem is isomorphic to $K(\cL_P-\Mod)$. 

\begin{remark}
	Due to recent work, we know that if $P$ is torsion, then $\cL_P \cong \cA_P = P \times_{\PU(n)}\bC^{n \times n}$, which is an isomorphism of $2$-vector bundles, and induces an isomorphism $\cL_P-\Mod \cong \cA_P-\Mod$.
\end{remark}

\section{Thom Isomorphism in the Torsion Case}
Let $V \to M$ be a $\bR$-vector bundle. Then we get $Tr(V) \to M$,	the associated $SO(n)$-bundle, which comes with a sub-bundle with a $\spin^c(n)$-structure. This gives us a bijection $\cL_{Tr(V)} \cong CL(V)$, where $CL(V)$ is the complex Clifford bundle of $V$.

Here we can use the result by Karoubi.

\begin{theorem}
	We have an isomorphism $K(CL(V)-\Mod) \cong K(\Th(V))$.
\end{theorem}

We now want to generalize this to more general	twists, which should be the following result.

\begin{theorem}
	There is an isomorphism
	\[K^0(M,P+W_3(V)) \cong K^0(\Th(V),\pi^*P). \]
\end{theorem}

\section{Non-Torsion Case}
We now aim to generalize these results to the non-torsion case. Concretely, we want prove an analogue of the following theorem:
\[K(\cL_P-\Mod) \cong K_{U(n),\scal}(P) \]
We can try to reproduce the same diagram 
\[ 
	\begin{tikzcd}
	 \Gamma_P \arrow[d] & E \arrow[d] \\ 
	 P \times_M P \arrow[r, shift left = 1.8] \arrow[r, shift right = 1.8] & P \arrow[d] \\ 
	 & M  
	\end{tikzcd},
	\]
	but we cannot compare to $U(n)$ anymore, since $P$ is not a $\PU(n)$-bundle. So, we do not get the original result, but rather a restricted version.

	\begin{theorem}
	 $K^0(M,P) \cong K(\cL_P-\Mod^{U_2})$.
	\end{theorem}
	Here $U_2 \subseteq U(\cH)$	is the group of unitary objects that differ from the identity by a Hilbert-Schmidt operator, and $\cL_P-\Mod^{U_2}$ is the category of $\cL_P$-twisted vector bundles whose transition functions take values in $U_2$.

	Here $U^2$ can also be described via colimits, as $U^2 \cong \colim_{n \to \infty} U(n)$. So this result can be interpreted as the fact that filtered colimits preserve the original theorem.

Now we analogously have an isomorphism 
\[\cL_P-\Mod^{U_2} \cong \cA_p- \Mod^{U_2}. \]

\begin{remark}
	Similar to above, we anticipate an isomorphism	of $2$-vector bundles $\cL_P \cong P \times_{\PU(\cH)} \HS(\cH)$, that induces the isomorphism above as a special case. This would require a suitable Morita category for these Hilbert-Schmidt operators, which is currently unknown.
\end{remark}

\section{Twisted Pushforward}
As a last step we can deduce the Thom	isomorphism in this non-torsion case. 
Let $f\colon X \to Y$. Then we have $W_3(f) \coloneq W_3(TX \oplus f^* TY)$. This gives us a pushforward map 
\[f_!\colon K^*(X,W_3(f) + f^*P) \to K^{(*  + (\dim X - \dim Y) \mod 2)}(Y,P), \]
for $P \to Y$ some $\PU(\cH)$-bundle.

The construction of this map proceeds as follows: We can factor $X \to Y$ into $X \hookrightarrow Y \times S^n \to Y$. The first step uses the Thom isomorphism, and the second step uses $C^*$-algebra techniques.

\begin{remark}
 If $Y$ is trivial, then this construction gives us a version of an index. 
\end{remark}

\section{D-Branes}
	There is an application of these methods to physics. Given a $U(n)$-bundle $\sG$ over $X$ with sub objects $Q$, such that the restriction along $Q$ has a  $\sG|_Q$-module $\sE$.

	It is claimed by Witten that these d-branes are classified by some version of K-theory. This can be made precise as follows (with some adjustments). Given $\sE$ and $\sG|_Q \otimes \cL_f$-module we get an element in  $K^0(Q, \sG|_Q \otimes \cL_f)$. Using the pushforward map, we have a map
	\[K^0(Q, \sG|_Q \otimes \cL_f) \to K^*(X,\sG) \] 
giving us an element	in $K^*(X,\sG)$, as desired.

We can in particular consider the case when $X = \sG$, which is known as the WZW-model in physics. In this case, $Q$ are conjugacy classes $C_g$ of some element	$g$ in the group and for $f\colon C_g \to \cG$, we have 
\[W_3(f) = (f^* \sG_{bas})^{\check{c}}\]  
	
This is a non-trivial observation. Indeed comparing with the Freed-Hopkins-Teleman approach \cite{freedhopkinsteleman2011loopgroupsi}, we see that 
\[K_{\sG}(\sG, \sG^k_{bas}) \cong \Rep^{k + \check{c}}(L\sG), \]
meaning this $\check{c}$ Coxeter shift appears naturally	in the representation theory of the loop group. 

Hence this approach suggests a possible geometric interpretation of these constructions.

	{\footnotesize
 \bibliographystyle{alpha}
 \bibliography{main}
 }
\end{document}