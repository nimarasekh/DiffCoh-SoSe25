\documentclass[10pt]{amsart}
\usepackage{amsmath,amsthm,amssymb,amsfonts}
\usepackage[mathscr]{euscript}
\usepackage{tikz}
\usepackage{tikz-cd}
\usepackage{enumerate}
\usepackage{enumitem}
\usepackage{mathtools}
\usepackage[colorlinks=true, linkcolor=red, citecolor = blue]{hyperref}
\usepackage[margin=2.5cm]{geometry}
\setlength{\marginparwidth}{2cm}

\usepackage[nameinlink,capitalise,noabbrev]{cleveref}

\usepackage[textwidth=2cm, textsize=small, colorinlistoftodos]{todonotes}

\newcommand{\A}{\mathcal{A}}
\newcommand{\B}{\mathcal{B}}
\newcommand{\cB}{\mathcal{B}}
\newcommand{\D}{\mathcal{D}}
\newcommand{\sC}{\mathscr{C}}
\newcommand{\F}{\mathcal{F}}
\newcommand{\G}{\mathcal{G}}
\newcommand{\g}{\mathfrak{g}}
\newcommand{\cH}{\mathcal{H}}
\newcommand{\sS}{\mathscr{S}}
\newcommand{\bN}{\mathbb{N}}
\newcommand{\bS}{\mathbb{S}}
\newcommand{\bR}{\mathbb{R}}
\newcommand{\bZ}{\mathbb{Z}}
\newcommand{\bC}{\mathbb{C}}
\newcommand{\fO}{\mathbf{O}}
\newcommand{\rO}{\mathscr{O}}
\newcommand{\M}{\mathcal{M}}
\newcommand{\ft}{\mathfrak{t}}
\newcommand{\bQ}{\mathbb{Q}}
\newcommand{\U}{\mathcal{U}}

\newcommand{\aff}{\mathrm{Aff}}
\newcommand{\Cyc}{\mathrm{Cyc}}
\newcommand{\tmf}{\mathrm{tmf}}
\newcommand{\TMF}{\mathrm{TMF}}
\newcommand{\Tmf}{\mathrm{Tmf}}
\newcommand{\Def}{\mathrm{Def}}
\newcommand{\curv}{\mathrm{curv}}
\newcommand{\CS}{\mathrm{CS}}
\newcommand{\ch}{\mathrm{ch}}
\newcommand{\dKU}{\smash{\widehat{ku}}}
\newcommand{\dKUnabla}{\smash{\widehat{ku}^\nabla}}
\renewcommand{\sp}{\mathrm{sp}}
\DeclareMathOperator{\tr}{tr}
\newcommand{\Map}{\mathrm{Map}}
\newcommand{\uMap}{\underline{\mathrm{Map}}}
\newcommand{\Hom}{\mathrm{Hom}}
\newcommand{\Ho}{\mathrm{ho}}
\newcommand{\set}{\mathscr{S}\mathrm{et}}
\newcommand{\CMon}{\mathrm{CMon}}
\newcommand{\CGrp}{{\normalfont\texttt{CMon}}}
\newcommand{\fib}{{\normalfont\texttt{fib}}}
\newcommand{\cofib}{{\normalfont\texttt{cofib}}}
\newcommand{\bdl}{\mathcal{B}\mathrm{dl}}
\newcommand{\Sp}{\mathscr{S}\mathrm{p}}
\newcommand{\Ch}{\mathscr{C}\mathrm{h}}
\newcommand{\cat}{\mathscr{C}\mathrm{at}}
\newcommand{\scat}{s\mathscr{C}\mathrm{at}}
\newcommand{\sset}{s\mathscr{S}\mathrm{et}}
\newcommand{\Line}{\mathcal{L}\mathrm{ine}}
\newcommand{\LineBdl}{\Line\bdl}
\newcommand{\Fun}{\mathrm{Fun}}
\newcommand{\Nat}{\mathrm{Nat}}
\newcommand{\colim}{\mathrm{colim}}
\newcommand{\Top}{\mathscr{T}\mathrm{op}}
\newcommand{\Grpd}{\mathscr{G}\mathrm{rpd}}
\newcommand{\Grp}{\mathscr{G}\mathrm{rp}}
\newcommand{\Euc}{\mathscr{E}\mathrm{uc}}
\newcommand{\Mfd}{\mathscr{M}\mathrm{fd}}
\newcommand{\Kan}{\mathscr{K}\mathrm{an}}
\newcommand{\Vect}{\mathscr{V}\mathrm{ect}}
\newcommand{\Mod}{\mathscr{M}\mathrm{od}}
\newcommand{\Proj}{\mathscr{P}\mathrm{roj}}
\newcommand{\Ab}{\mathscr{A}\mathrm{b}}
\newcommand{\Shv}{\mathscr{S}\mathrm{hv}}
\newcommand{\Yon}{\mathscr{Y}\mathrm{on}}
\newcommand{\Open}{\mathscr{O}\mathrm{pen}}
\newcommand{\PSh}{\mathscr{P}\mathscr{S}\mathrm{h}}
\newcommand{\Pic}{\mathscr{P}\mathrm{ic}}
\newcommand{\triv}{\mathscr{T}\mathrm{riv}}
\newcommand{\aut}{\mathrm{Aut}}
\newcommand{\Th}{\mathrm{Th}}
\newcommand{\fr}{\mathrm{Fr}}
\newcommand{\Arr}{\mathrm{Arr}}
\newcommand{\ev}{\mathrm{ev}}
\newcommand{\dis}{\mathrm{dis}}
\newcommand{\Mon}{\mathrm{Mon}}
\newcommand{\PT}{\mathrm{PT}}
\newcommand{\Fred}{\mathcal{F}\mathrm{red}}
\newcommand{\inv}{\mathrm{inv}}
\newcommand{\const}{\mathrm{const}}
\newcommand{\fl}{\mathrm{fl}}
\newcommand{\Grb}{\mathcal{G}\mathrm{rb}}
\newcommand{\cl}{\mathrm{cl}}

\newcommand{\bbefamily}{\fontencoding{U}\fontfamily{bbold}\selectfont}
\newcommand{\textbbe}[1]{{\bbefamily #1}}
\DeclareMathAlphabet{\mathbbe}{U}{bbold}{m}{n}

\def\DDelta{{\mathbbe{\Delta}}}
\newcommand{\DD}{\DDelta}



%% N.R. notes
\newcommand{\nrnote}[1]{\todo[color=green!40,linecolor=green!40!black,size=\tiny]{#1}}
\newcommand{\nrmpar}[1]{\todo[noline,color=green!40,linecolor=green!40!black,
  size=\tiny]{#1}}
\newcommand{\nrnoteil}[1]{\ \todo[inline,color=green!40,linecolor=green!40!black,size=\normalsize]{#1}}

\newtheorem{theorem}[equation]{Theorem}
\newtheorem{lemma}[equation]{Lemma}
\newtheorem{proposition}[equation]{Proposition}
\newtheorem{corollary}[equation]{Corollary}
% \newtheorem{statement}[section]{Statement}

\theoremstyle{definition}
\newtheorem{definition}[equation]{Definition}
\newtheorem{example}[equation]{Example}
% \newtheorem{attone}[equation]{Attention}

\theoremstyle{remark}
\newtheorem{remark}[equation]{Remark}
% \newtheorem{intone}[equation]{Intuition}
\newtheorem{notation}[equation]{Notation}
\newtheorem{question}[equation]{Question}
% \newtheorem{conjone}[equation]{Conjecture}
\newtheorem{warning}[equation]{Warning}

\numberwithin{equation}{section}

\title{Differential Cohomology Seminar 1}
\date{30.04.2025}
\author{Talk by Konrad	Waldorf}

\begin{document}
\maketitle


We are trying to understand historically why we need differential cohomology as a generalization of the classical cohomology theory.
Roughly speaking, the story goes through work of Deligne \cite{deligne1971mixedhodge}, Beilinson \cite{beilinson1984regulators}, and Brylinski  \cite{brylinski1993geomquant}. It then proceeds with work of Cheeger--Simons \cite{cheegersimons1985diffchar}, Freed \cite{freed2000diffcoh} and Hopkins--Singer \cite{hopkinssinger2005diffcoh}. It makes significant advances with Simons--Sullivan \cite{simonssullivan2008diffcoh} and Bunke--Schick \cite{bunkeschick2009smoothk} and finally Bunke-Nikolaus-V{\"o}lkl \cite{bunkenikolausvoelkl2016diffcoh}, getting us to our modern definition. 

This talk is primarily a historical overview and motivational talk reviewing some of these concepts and witnessing the needs for more general notions of cohomology. This in particular means that most mathematical details are missing and/or skipped.

\section{Why are Cohomology Theories not Enough?}
Consider the following fact about integral singular cohomology: Let $M$ be a smooth manifold, $\bC^*\bdl(M)$ the category of $\bC^*$-principal bundles over $M$, and $\LineBdl$ the category of complex line bundles over $M$, then there is a sequence of isomorphisms:
\begin{equation}\label{eq:cech}
	\pi_0(\LineBdl(M))\simeq\pi_0(\bC^*\bdl(M))\xrightarrow{\delta}\check{H}^2(M,\bZ)\simeq H^2(M,\bZ)
\end{equation}
\begin{remark}
	A short exact sequence of abelian sheaves $0\to\F\to\G\to\cH\to 0$ induces an exact sequence of \v Cech cohomology groups, see \cite{project2026cechsing}, including a connecting homomorphism $\delta:\check{H}^1(M,\cH)\to\check{H}^2(M,\F)$. The morphism $\delta$ in \cref{eq:cech} is the connecting homomorphism associated to the sequence $0\to2\pi i\bZ\to\bC\to\bC^*\to0$. 
\end{remark}
In particular, isomorphism classes of line bundles only depend on the underlying topological space of $M$. To recover more information about the smooth geometry of $M$, let us consider the categories $\bC^*\bdl_\nabla(M)$ and $\LineBdl_\nabla(M)$ of $\bC^*$-principal bundles, resp. line bundles, with connection and connection-preserving isomorphisms, then 
\[\pi_0(\LineBdl_\nabla(M))\simeq\pi_0(\bC^*\bdl_\nabla(M))\]
\begin{question}
Is there a generalized version of cohomology theory $E^*$ such that $E^2(M)\simeq\pi_0(\bC^*\bdl_\nabla(M))$? 
\end{question}
Imitating \v Cech and de Rham cohomology, consider the following modified de Rham chain complex of sheaves: Let $\Theta\in\Omega^1(\bC^*)$ be the Maurer-Cartan form, 
\begin{equation}
	\begin{tikzcd}
		\D(n):C^\infty_{\bC^*}\arrow[r," d\log"] & \Omega^1 \arrow[r," d"] & \cdots \arrow[r," d"] & \Omega^{n-1}
	\end{tikzcd}
\end{equation}where $C^\infty_{\bC^*}$ is the sheaf of $\bC^*$-valued functions on $M$ and $ d\log$ maps $f\in C^{\infty}(U,\bC^*)$ to $f^*\Theta\in\Omega^1(U)$. Given a cover $U\to M$, construct $\D(2)(U)$ as the total complex of the following bi-complex 
\begin{equation}
	\begin{tikzcd}
		\vdots & \vdots & \vdots   \\
		C^\infty(U^{[3]},\bC^*) \arrow[r]\arrow[u] & \Omega^1(U^{[3]}) \arrow[r]\arrow[u] &  \Omega^2(U^{[3]}) \arrow[u] \arrow[r] & \cdots \\ 
		C^\infty(U^{[2]},\bC^*) \arrow[r," d^{1,1}"]\arrow[u,"\delta^{0,2}"] & \Omega^1(U^{[2]}) \arrow[u]\arrow[r] & \Omega^2(U^{[2]}) \arrow[u]\arrow[r] & \cdots\\
		C^\infty(U,\bC^*) \arrow[r]\arrow[u] & \Omega^1(U) \arrow[r," d^{2,0}"]\arrow[u,"\delta^{1,1}"] & \Omega^2(U) \arrow[u] \arrow[r] & \cdots
	\end{tikzcd}
\end{equation}where $U^{[n]}$ denotes the $n$-fold fiber product over $M$. Given $(L,\nabla)\in\bC^*\bdl_\nabla(M)$, we construct a class in $C^\infty(U^{[2]},\bC^*)\oplus\Omega^1(U)$ as follows: Take $U=L\to M$ as cover, $\mu:L^{[2]}=L\times_ML\to\bC^*$ the difference map, characterized by $\mu(p_1,p_0)\triangleright p_0=p_1$, and $\nabla\in\Omega^1(L)$ the connection 1-form. Unfolding the definitions, notice the following:
\begin{enumerate}
	\item $\delta^{0,2}\mu=1$ is equivalent to the cocycle condition for $\mu$.
	\item $ d^{1,1}\mu=\delta^{1,1}\nabla$ is equivalent to $\nabla$ being $\bC^*$-equivariant.
	\item $ d^{2,0}\nabla=0$ appears only for $n>2$, and is equivalent to flatness of $\nabla$.  
\end{enumerate}
Notice that (1)-(2) are satisfied by definition, while (3) is an additional condition on $\nabla$.
\begin{definition}
	Let $M$ be a smooth manifold, denote by $\cH^n(M)$ the colimit over all covers $U\to M$ of the $n$-th cohomology group of $\D(n)(U)$, called the \emph{$n$-th Deligne cohomology group}. 
\end{definition}
\begin{theorem}
	$\cH^2(M)\simeq\pi_0(\bC^*\bdl_\nabla(M))$.
\end{theorem} 

\section{From Exact Sequences to the Hexagon}
A differential cohomology theory is meant as a coupling of ordinary cohomology classes with differential data, such as connections. The coherence conditions of this coupling are captured by axioms, in the style of Steenrod-Eilenberg axioms for ordinary cohomology theories, which include the existence of a certain diagram of natural transformations, called the \emph{hexagon}.

Taking a closer look at $\cH^n(M)$ we see the following: A $n$-cocycle representing a class in $\cH^n(M)$ consists of a cover $U\to M$, together with \[\mu:U^{[n]}\to\bC^*,\A_1\in\Omega^1(U^{[n-1]}),\A_2\in\Omega^2(U^{[n-2]}),\cdots,\A_{n-1}\in\Omega^{n-1}(U)\]
The condition $\delta^{0,n}\mu=1$ is equivalent to $\mu$ being a $(n-1)$-cocycle for $\bC^*$-valued \v Cech cohomology of $M$, therefore there is a map $\cH^n(M)\to\check{H}^{n-1}(M,\bC^*)\simeq H^n(M,\bZ)$. Using \[\delta^{n-1,1}\A_{n-1}= d^{n-1,1}\A_{n-2}\]we can show that $\delta^{n,1} d^{n,0}\A_{n-1}=0\in\Omega^n(U^{[2]})$, i.e. $ d^{n,0}\A_{n-1}$ is constant on fibers of $\pi:U\to M$, so there is a unique $F\in\Omega^{n}(M)$ such that $\pi^*F= d^{n,0}\A_{n-1}$ and is closed, therefore there is a map $\cH^n(M)\to\Omega^{n}_{\cl}(M)$.\nrnote{How to prove $F$ has integral periods?} The 
\begin{remark}
	$F$ being closed can be proved as follows: $\pi^* d F= d^{n+1,0}\pi^*F= d^{n+1,0} d^{n,0}\A_{n-1}=0$, by uniqueness we conclude $dF=0$. 
\end{remark}
\begin{remark}
	For $n=2$, we recover the usual notion of curvature 2-form. 
\end{remark}
For semplicity, consider $n=2$ and a $2$-cocycle of $\cH^2(M)$ such that $\mu$ is a \v Cech $2$-coboundary. This means that the underlying $\bC^*$-principal bundle is trivial, the choice of a section corresponds to the choice of $\beta\in C^\infty(U,\bC^*)$ such that $\delta^{0,1}\beta=\mu$. In that case, $(\mu,\A)$ is cohomologous to $(0,\A-d^{1,0}\beta)$ and identified by a unique closed for $\rho\in\Omega^1(M)$ such that $\pi^*\rho=\A-d^{1,0}\beta$. In conclusion, there is a short exact sequence \begin{equation}
	\begin{tikzcd}
		0 \arrow[r] & H^{1}(M) \arrow[r] & \cH^2(M) \arrow[r] & H^2(M,\bZ) \arrow[r] & 0
	\end{tikzcd}
\end{equation}
\begin{remark}
	$\rho$ exists by the same argument as $F$ above, since $\delta^{1,1}(\A-d^{1,0}\beta)=\delta^{1,1}\A-d^{1,1}\delta^{0,1}\beta=\delta^{1,1}\A-d^{1,1}\mu=0$ in $\Omega^1(U^{[2]})$. 
\end{remark}
\begin{remark}
	Surjectivity is a corollary of the fact that every line bundle admits a connection, proven using partitions of unity. 
\end{remark}
%----------TO BE CONTINUED------------%
Having witnessed one interesting exact sequence, let us observe a second one, which is the following:
\[ 0 \to H^{n-1}(M, \bC^\times) \to \hat{H}^n(M) \xrightarrow{curv} \Omega_{cl,\bZ}^n(M)\to 0 \]
We will explicate why this diagram is exact, but let us make the following observation. First of all, the kernel is indeed reasonably defined, as $H^{n-1}(M, \bC^\times)$ classifies line bundles with equal curvature, meaning there is a mapping $(g_{\alpha\beta\gamma}, A_{\alpha\beta}, B_\alpha) \mapsto d B_\alpha$  Notice, this is indeed well-defined as $(\delta B)_{\alpha\beta} = d A_{\alpha\beta}$.\nrnote{How does that relate to the previous step?}

It is an amazing fact about $\hat{H}$ that these two seemingly independent exact sequences actually fit together with $\hat{H}$, via two maps from $\Omega^n_{cl,\bZ}(M) \to H^n(M,\bR)$ and $H^n(M,\bZ)  \to H^n(M,\bR)$. This is known as the \emph{hexagonal diagram} in differential cohomology theory. 

\[ 
\begin{tikzcd}
	&H^{n-1}(M,\bC^\times) \arrow[rr] \arrow[dr]& & H^n(M,\bZ) \arrow[dr] & \\
	H^{n-1}(M,\bR) \arrow[ur] \arrow[dr] & & \hat{H}^n(M) \arrow[ur, "cc"] \arrow[dr, "\curv"]  & & H^n(M,\bR) \\ 
	& \Omega^{n-1} \arrow[rr, "q"] \arrow[ur] & & \Omega^{n}_{cl,\bZ} \arrow[ur] & 
\end{tikzcd}
\]

This was first discovered by Simons and Sullivan  in the 2000s \cite{simonssullivan2008diffcoh}.

\section{Holonomy and Characters}
We have seen that differential cohomology is a generalization of ordinary cohomology. We can hence wonder how differential cohomology interacts with other aspects of ordinary cohomology theory, such as holonomy and characters. 
Intuitively, holonomy should give us new elements via parallel transport along loops. First of all, we can generalize this to arbitrary dimensions. For example, let $\Phi\colon \Sigma \to M$ be a smooth map, where $\Sigma$ is an $(n-1)$-dimensional closed manifold, and $\xi \in \hat{H}^n(M)$ be a differential cohomology class. Now if we take the pullback $\Phi^*\xi \in \hat{H}^n(\Sigma)$, it lands in the kernel of $cc\colon \hat{H}^n(\Sigma) \to H^n(\Sigma,\bZ)$. By exactness, it hence must be of the form $i(\varphi)$, where $\varphi \in \Omega^{n-1}(\Sigma)$. This allows us to define  
\[Hol_\xi(\Phi) = e^{2\pi i\int_{\Sigma}\varphi} \in \bC^\times\] 
Here the exponential is crucial as the integral is only well-defined up to integer multiples of $2\pi i$. Indeed, if $\varphi, \varphi'$ are in the fiber, then $\int_\Sigma \varphi - \int_\Sigma \varphi' \in \bZ$. We can now apply Stokes' theorem to this situation. Concretely, if $\Sigma = \partial B$, where $B$ is an $n$-dimensional manifold and $\phi = \Phi|_\Sigma$, then we have

\[ Hol(\phi) = e^{2\pi i \int_B \Phi^*\curv(\xi)}\]

Let us look at a manifestation thereof. In the WZW (Wess–Zumino–Witten) Model one has a surface $\Sigma$, $\phi\colon \Sigma \to M$, and a $B$-field $B \in \Omega^2(M)$ . Then we have $S(\phi) = \int_\Sigma \phi^*B$. \nrnote{How does that relate to the previous explanation?}  

If $M = G$ is a Lie group, then WZW model is not globally defined i.e. $dB = H = <\theta \wedge [\theta \wedge \theta]>$. The solution is to take a differential cohomology class $\xi \in \hat{H}^3(G)$, such that  $dB = \curv(\xi)$. Then $S(\phi) = Hol_\xi(\phi)$. This indeed matches our intuition, as $Hol_{i(\alpha)}(\phi) = e^{2\pi i \int_\Sigma \phi^*\alpha}$. Witten showed that if $G$ is a simply connected Lie group, then the setup admits an extension to $3$ dimensions (we can skip dimension $2$), which implies that every map $\Sigma \to G$ must be nul-homotopic. Finally, let $[\Sigma]$ be the fundamental class of $\Sigma$. Then $\phi_*[\Sigma] \in H_{n-1}(M)$ and $\xi\colon H_{n-1} \to \bC^\times$, reconstructs differential cohomology classes.

\section{Fiber Integration} 
We will end with an honorable mention of fiber integration. Recall that, given a bundle $E \to X$, with fiber $F$ closed (compact and no boundary), there is a fiber integration map $\pi_!\colon \hat{H}^n(E) \to \hat{H}^{n-dim(F)}(X)$ with a variety of important properties. One particular application is transgression. In that case $S^1 \times LM \to LM$ is a bundle, with evaluation $S^1 \times LM \to M$. Then $\xi \in \hat{H}^n(M)$ gives us a class $\pi_!(ev^*\xi) \in \hat{H}^{n-1}(LM)$.


{\footnotesize
\bibliographystyle{alpha}
\bibliography{main}
}
\end{document}