\documentclass[10pt]{amsart}
\usepackage{amsmath,amsthm,amssymb,amsfonts}
\usepackage{eucal}
\usepackage{tikz}
\usepackage{tikz-cd}
\usepackage{enumerate}
\usepackage{enumitem}
\usepackage[colorlinks=true,linkcolor=red, citecolor = blue]{hyperref}
\usepackage[margin=2.5cm]{geometry}
\setlength{\marginparwidth}{2cm}

\usepackage[nameinlink,capitalise,noabbrev]{cleveref}

\usepackage[textwidth=2cm, textsize=small, colorinlistoftodos]{todonotes}

\newcommand{\lA}{\mathcal{A}}
\newcommand{\bA}{\mathbb{A}}
\newcommand{\lB}{\mathcal{B}}
\newcommand{\kC}{\mathfrak{C}}
\newcommand{\lC}{\mathcal{C}}
\newcommand{\rC}{\mathscr{C}}
\newcommand{\bC}{\mathbb{C}}
\newcommand{\lD}{\mathcal{D}}
\newcommand{\lE}{\mathcal{E}}
\newcommand{\lF}{\mathcal{F}}
\newcommand{\lG}{\mathcal{G}}
\newcommand{\lH}{\mathcal{H}}
\newcommand{\mH}{\mathrm{H}}
\newcommand{\lI}{\mathcal{I}}
\newcommand{\lJ}{\mathcal{J}}
\newcommand{\lK}{\mathcal{K}}
\newcommand{\lL}{\mathcal{L}}
\newcommand{\mL}{\mathrm{L}}
\newcommand{\lM}{\mathcal{M}}
\newcommand{\bN}{\mathbb{N}}
\newcommand{\mN}{\mathrm{N}}
\newcommand{\lN}{\mathcal{N}}
\newcommand{\fO}{\mathbf{O}}
\newcommand{\rO}{\mathscr{O}}
\newcommand{\lO}{\mathcal{O}}
\newcommand{\lP}{\mathcal{P}}
\newcommand{\bQ}{\mathbb{Q}}
\newcommand{\bR}{\mathbb{R}}
\newcommand{\lR}{\mathcal{R}}
\newcommand{\lS}{\mathcal{S}}
\newcommand{\bS}{\mathbb{S}}
\newcommand{\lT}{\mathcal{T}}
\newcommand{\lU}{\mathcal{U}}
\newcommand{\lV}{\mathcal{V}}
\newcommand{\lX}{\mathcal{X}}
\newcommand{\lY}{\mathcal{Y}}
\newcommand{\bZ}{\mathbb{Z}}

\newcommand{\op}{\mathrm{op}} % opposite
\newcommand{\Hom}{\lH\mathrm{om}} % enriched hom-spaces, hom-spectrum
\renewcommand{\hom}{\mathrm{Hom}} % hom-set, hom-space
\newcommand{\Ho}{\mathrm{Ho}} % homotopy category
\newcommand{\set}{\lS\mathrm{et}} % category of sets
\newcommand{\Sp}{\lS\mathrm{p}} % category of spectra
\newcommand{\PSp}{\lP\Sp} % category of pre-spectra
\newcommand{\Ch}{\lC\mathrm{h}} % category of chain complexes
\newcommand{\cat}{\lC\mathrm{at}} % category of categories
\newcommand{\scat}{\mathrm{s}\cat} % category of simplicial categories
\newcommand{\sset}{\mathrm{s}\set} % category of simplicial sets
\newcommand{\Fun}{\lF\mathrm{un}} % category of functors
\newcommand{\Top}{\lT\op} % category of topological spaces 
\newcommand{\Grpd}{\lG\mathrm{rpd}} % category of groupoids
\newcommand{\Kan}{\lK\mathrm{an}} % category of Kan complexes
\newcommand{\Mod}{\lM\mathrm{od}} % category of modules
\newcommand{\Ab}{\lA\mathrm{b}} % category of abelian groups
\newcommand{\Open}{\lO\mathrm{pen}} % category of open subspaces
\newcommand{\BDelta}{\mathbf{\Delta}} % simplex category
\newcommand{\Sing}{\mathrm{Sing}} % singular simplicial set 


\DeclareMathOperator*\colim{colim} % colimit

\newcommand{\bbefamily}{\fontencoding{U}\fontfamily{bbold}\selectfont}
\newcommand{\textbbe}[1]{{\bbefamily #1}}
\DeclareMathAlphabet{\mathbbe}{U}{bbold}{m}{n}

\def\DDelta{{\mathbbe{\Delta}}}
\newcommand{\DD}{\DDelta}

\newcommand{\adjun}[4]{
\begin{tikzcd}[row sep=0.5in, column sep=0.5in]
 #1  \arrow[r, shift left=1.8, "#3"] \pgfmatrixnextcell
 #2 \arrow[l, shift left=1.6, "#4", "\bot"'] 
\end{tikzcd}
}

\newcommand{\simpset}[7]{
 \begin{tikzcd}[row sep=0.5in, column sep=0.5in]
   #1 \arrow[r, shorten >=1ex,shorten <=1ex]
   \pgfmatrixnextcell #2 
   \arrow[l, shift left=1.2, "#5"] \arrow[l, shift right=1.2, "#4"'] 
   \arrow[r, shift right, shorten >=1ex,shorten <=1ex ] \arrow[r, shift left, shorten >=1ex,shorten <=1ex] 
   \pgfmatrixnextcell #3 
   \arrow[l] \arrow[l, shift left=2, "#7"] \arrow[l, shift right=2, "#6 "'] 
   \arrow[r, shorten >=1ex,shorten <=1ex] \arrow[r, shift left=2, shorten >=1ex,shorten <=1ex] \arrow[r, shift right=2, 
   shorten >=1ex,shorten <=1ex]
   \pgfmatrixnextcell \cdots 
   \arrow[l, shift right=1] \arrow[l, shift left=1] \arrow[l, shift right=3] \arrow[l, shift left=3] 
 \end{tikzcd}
}


%% N.R. notes
\newcommand{\nrnote}[1]{\todo[color=green!40,linecolor=green!40!black,size=\tiny]{#1}}
\newcommand{\nrmpar}[1]{\todo[noline,color=green!40,linecolor=green!40!black,
  size=\tiny]{#1}}
\newcommand{\nrnoteil}[1]{\ \todo[inline,color=green!40,linecolor=green!40!black,size=\normalsize]{#1}}

\newtheorem{theorem}[equation]{Theorem}
\newtheorem{lemma}[equation]{Lemma}
\newtheorem{proposition}[equation]{Proposition}
\newtheorem{corollary}[equation]{Corollary}
% \newtheorem{statement}[section]{Statement}

\theoremstyle{definition}
\newtheorem{definition}[equation]{Definition}
\newtheorem{example}[equation]{Example}
% \newtheorem{attone}[equation]{Attention}

\theoremstyle{remark}
\newtheorem{remark}[equation]{Remark}
% \newtheorem{intone}[equation]{Intuition}
\newtheorem{notation}[equation]{Notation}
% \newtheorem{queone}[equation]{Question}
% \newtheorem{conjone}[equation]{Conjecture}
\newtheorem{warning}[equation]{Warning}

\title{Differential Cohomology Seminar 2}
\date{06.05.2025 $\&$ 13.05.2025}
\author{Talk by Matthias Frerichs}

\numberwithin{equation}{section}

\begin{document}
\maketitle

In this lecture we want to learn the basics of $\infty$-category theory. For the $\infty$-categorical background, we broadly follow \cite{groth2010inftycats} and a little \cite{lurie2009htt}.

\section{Basics on \texorpdfstring{$(\infty,1)$}{(oo,1)}-categories}

$(\infty,1)$-categories have different models that capture its essence. The first model are \emph{quasi-categories}. 
\begin{definition}
  Given a natural number $n$, let $\langle n\rangle$ denote the linearly ordered set $\{0,\cdots,n\}$. The \emph{simplex category} $\BDelta$ is the category of finite linerly ordered sets $\langle n\rangle$, for every $n$, and monotone functions.
\end{definition}
\begin{definition}
  Given $0\leq i\leq n$, the \emph{$i$-face map} is the unique injective map $\delta^i_n:\langle n-1\rangle\to\langle n\rangle$ missing $i$. The \emph{$i$-degeneracy map} is the unique surjective map $\sigma^i_n:\langle n+1\rangle\to\langle n\rangle$ such that $i$ and $i+1$ have the same image. 
\end{definition}
\begin{theorem}
  As a category, $\BDelta$ is generated from the face and degeneracy maps subject to the \emph{simplicial identities} (see \cite[I.4]{goerssjardine1999simplicialhomotopytheory}). 
\end{theorem}
\begin{definition}
  A \emph{simplicial set} is a contravariant functor $X:\BDelta^{\op} \to \set$. Denote by $\sset$ the category of simplicial sets. The representable simplicial set $\hom_{\BDelta}(-,\langle n\rangle)$ is denoted as $\Delta^n$. The set of \emph{$n$-simplices}, denoted by $X_n$, is the image of $\langle n\rangle$. By Yoneda lemma, we identify a $n$-simplex with the corresponding map of simplicial sets $\Delta^n\to X$. Let $d^n_i:X_{n}\to X_{n-1}$ be the map of sets induced by the $i$-face map $\delta^i_n$, which we also call \emph{$i$-face map}.
\end{definition}
By only representing the face maps, we can depict a simplicial set as follows: 
\begin{equation}
  \begin{tikzcd}
    X_0  & X_1 \arrow[l,shift right]\arrow[l,shift left] & X_2 \arrow[l,shift right=2]\arrow[l]\arrow[l,shift left=2] & \cdots \arrow[l,shift right=3]\arrow[l,shift right=1]\arrow[l,shift left=1]\arrow[l,shift left=3]
  \end{tikzcd}
\end{equation}
$\infty$-categories are then defined in terms of a lifting condition, for which we need to define horns.
\begin{definition}
  Given $i\in\langle n\rangle$, the $i$-horn is the simplicial sub-set $\Lambda^n_i\subseteq\Delta^n$ defined as follows: A monotone map $f:\langle k\rangle\to\langle n\rangle$ is a $k$-simplex of $\Lambda^n_i$ if there is $j\neq i$ that does not belong in the image of $f$. 
\end{definition}
\begin{definition}A simplicial set $X$ is called a \emph{quasi-category} if every inner horn $\Lambda^n_i\to X$, i.e. $0<i<n$, can be extended to a $n$-simplex $\Delta^n\to X$. If every horn $\Lambda^n_i\to X$ admits an extension, $X$ is called a \emph{$\infty$-groupoid}. 
\end{definition}
\begin{remark}
  The nerve of a category $\lC$ is a quasi-category (one can simply check the horn filling condition directly). Moreover, the nerve is a $\infty$-groupoid if and only if $\lC$ is a groupoid. The nerve functor $\mN:\cat\to\sset$ has a left adjoint, the  \emph{homotopy category} functor $\Ho$. If $X$ is a quasi-category, $\Ho(X)$ has an explicit construction where objects are the 0-simplices of $X$ and morphisms are homotopy classes of 1-simplices, see \cite[1.2.5]{land2021introduction} and \cite[Proposition 1.15]{groth2010inftycats}.
\end{remark} 
\begin{definition}
  Let $X$ be a quasi-category, a 1-simplex $f:x\to y$ is an \emph{equivalence} if there are 2-simplices of the following form: 
  \begin{equation}
    \begin{tikzcd}
      & x \arrow[dr,"f"] \arrow[d,Rightarrow] & \\
      y \arrow[ru] \arrow[rr,no head,double line] & {} & y
    \end{tikzcd}\qquad \begin{tikzcd}
      & y \arrow[dr] \arrow[d,Rightarrow] & \\
      x \arrow[ru,"f"] \arrow[rr,no head,double line] & {} & x
    \end{tikzcd}
  \end{equation}where the bottom 1-simplices are the image of $y$ and $x$, respectively, under the function $X_0\to X_1$ induced by $\sigma^0_0$. 
\end{definition}
The following theorem shows that Kan complex have the same relation with quasi-categories, as groupoids have with categories. 
\begin{theorem}
  A quasi-category $X$ is a $\infty$-groupoid if and only if every 1-simplex is an equivalence. 
\end{theorem}
\begin{remark}
  \cite{groth2010inftycats} approaches $\infty$-groupoids in the other direction, namely defining a $\infty$-groupoid as a quasi-category where every 1-simplex is an equivalence, then characterizing $\infty$-groupoids as quasi-categories satisfying the horn filling condition for every horn. 
\end{remark}
A strict model for $\infty$-categories is given by simplicially enriched categories. 
\begin{definition}
  Denote by $\scat$ the category of simplicially enriched categories. 
\end{definition}
\begin{remark}
  Lurie constructs in \cite[1.1.5.1]{lurie2009htt} a functor $\mN_\Delta:\scat\to\sset$, called the \emph{homotopy coherent nerve}. Similarly to the ordinary nerve, it admits a left adjoint, denoted $\kC$ and called \emph{rigidification}, see \cite{duggerspivak2011rigidification}. 
\end{remark}
\begin{theorem}
  $\sset$ and $\scat$ underlie model structures, called \emph{Joyal} and \emph{Dwyer-Kan-Bergner}, respectively, such that fibrant-cofibrant objects are quasi-categories and categories enriched over $\infty$-groupoids, respectively, and the adjunction \begin{equation}
    \begin{tikzcd}
      \scat \arrow[r,shift left,"\mN_\Delta"] & \sset \arrow[l,shift left,"\kC"] 
    \end{tikzcd}
  \end{equation}lifts to a Quillen equivalence. 
\end{theorem}
\begin{definition}
  The category $\Kan$ of $\infty$-groupoids is a self-enriched category, so $\lS:=\mN_\Delta(\Kan)$ is a quasi-category, namely the \emph{quasi-category of $\infty$-groupoids}.
\end{definition}
Let $\Top$ be the category of compactly generated topological spaces.
\begin{theorem}\label{thm:quillen}
  $\sset$ and $\Top$ underlie model structures, both called \emph{Quillen}, such that fibrant-cofibrant are $\infty$-groupoids and retracts of cellular complexes, respectively, and the adjunction \begin{equation}
    \begin{tikzcd}
      \Top \arrow[r,shift left,"\Sing"] & \sset \arrow[l,shift left] 
    \end{tikzcd}
  \end{equation}lifts to a Quillen equivalence. 
\end{theorem}

\section{Basic constructions}

\begin{remark}
  Recall that $\sset$ is self-enriched. More specifically, given simplicial sets $X,Y$, the simplicial set $\hom_{\sset}(X,Y)$ is characterized as follows: A $n$-simplex is a map of simplicial sets $\Delta^n\times X\to Y$.
\end{remark}
\begin{lemma}
  If $X$ is a simplicial set and $\lC$ a $\infty$-category, then $\hom_{\sset}(X,\lC)$ is a $\infty$-category. 
\end{lemma}
\begin{definition}
  Let $\lC$ and $\lD$ be $\infty$-categories, denote by $\Fun(\lC,\lD)$ the \emph{$\infty$-category of functors $\lC\to\lD$}.  
\end{definition}
\nrnote{Definition of opposite $\infty$-category, slice $\infty$-category}

\section{Accessible and Presentable Categories}

In general, a limit, resp. colimit, preserving functor need not have a left, resp. right, adjoint. Here we wish to introduce a rather large class of $\infty$-categories for which the previous statement holds. We begin by recalling the definition of locally presentable 1-categories. Let $\kappa$ denote a regular cardinal. 
\begin{definition}
  A category $\lI$ is \emph{$\kappa$-filtered} if, for every $\lJ$ with $<\kappa$ many morphisms, every diagram $\lJ\to\lI$ has a cocone. A functor $F:\lC \to \lD$ is \emph{$\kappa$-accessible} if it preserves $\kappa$-filtered colimits. Given a category $\lC$, an object $X$ is \emph{$\kappa$-compact} if $\hom_{\lC}(X,-):\lC\to\set$ is $\kappa$-accessible. 
\end{definition}
\begin{definition}
  A category $\lC$ is \emph{$\kappa$-accessible} if there exists a set $S\subseteq\lC_0$ of $\kappa$-compact objects that generate $\lC$ under $\kappa$-filtered colimits. A category is \emph{accessible} if it is $\kappa$-accessible, for some regular cardinal $\kappa$. 
\end{definition}
\begin{definition}
  A category $\lC$ that is accessible and cocomplete is called \emph{locally presentable}.
\end{definition}
\begin{theorem}
  Let $\lC$ be a category, then $\lC$ is locally presentable if and only if there exists a small category $S$ such that the induced functor $\lC\to\lP(S)$ is a fully faithful, accessible right adjoint. 
\end{theorem}
\begin{theorem}
  Let $\lC,\lD$ be locally presentable categories. A functor $F\colon \lC \to \lD$ is a left, resp. right, adjoint if and only if it preserves colimits, resp. it preserves limits and is accessible.
\end{theorem}
\begin{definition}
  Let $X$ be a $\infty$-category. The \emph{$\infty$-category of pre-sheaves of spaces} is defined as $\lP(X):=\hom_{\sset}(X^{\op},\lS)$.
\end{definition}
\begin{theorem}[Yoneda]
  Given a $\infty$-category $X$, there is a fully faithful functor $y:X\to\lP(X)$, called the \emph{Yoneda embedding}, such that: Given a cocomplete $\infty$-category $Y$, pre-composition by $y$ induces an equivalence \begin{equation}
    \begin{tikzcd}
      \Fun^\mL(\lP(X),Y) \arrow[r] & \Fun(X,Y)
    \end{tikzcd}
  \end{equation}where $\Fun^\mL$ denotes the category of colimit preserving functors. 
\end{theorem} The definition of accessible category transfers directly to the $\infty$-categorical setting. 
\begin{theorem}\label{thm:locpresquasi}
  A $\infty$-category $X$ is locally presentable (cocomplete and accessible) if and only if there is a small sub-$\infty$-category $S$ such that $X\to\Fun(S^\op,\lS)$ is a fully faithful, accessible right adjoint. 
\end{theorem}
\begin{remark}
  In view of \cref{thm:locpresquasi}, one can define a category to be \emph{locally presentable} if it is the accessible right localization of a pre-sheaf category for some small $\infty$-category $S$. In particular, every pre-sheaf category is locally presentable. 
\end{remark}
\begin{theorem}
  Let $X,Y$ be presentable $\infty$-categories, then a functor $f:X\to Y$ is a left, resp. right, adjoint if and only if it preserves colimits, resp. it preserves limits and is accessible.
\end{theorem}
\section{Stable \texorpdfstring{$\infty$}{oo}-Categories and Spectra}
We now use the $\infty$-categorical framework to study spectra. The study of spectra originates from the study of \emph{stable phenomena}, i.e. patterns appearing after repeated application of the suspension functor $\Sigma:\Top_*\to\Top_*$.
\begin{example}
  Let $(\Sigma,\Omega):\Top_{*/}\to\Top_{*/}$ be the suspension-loop adjunction on pointed topological spaces. \emph{Freudenthal Suspension Theorem} states that, if $X$ is a $n$-connected space, the adjunction unit $X\to\Omega\Sigma X$ is $2n$-connected. If $X$ is connected, $S^n\wedge X$ is $n$-connected, so $\Sigma^nX\to\Omega\Sigma^{n+1}X$ is $2n$-connected. In particular, $\pi_i(\Sigma^nX)\to\pi_{i+1}(\Sigma^{n+1}X)$ is an isomorphism, for all $i<2n$. The group $\pi_i(\Sigma^nX)$ is denoted $\pi_{n-i}^s(X)$, called the \emph{$(n-i)$-stable homotopy group of $X$}. 
\end{example}
\nrnote{More examples?}
\begin{definition}
  Let $\lC$ be an $\infty$-category, an object $0$ that is both initial and terminal is called 
  \emph{zero object}. A category $\lC$ with a zero object is called a \emph{pointed category}.
\end{definition}
\begin{example}
  Let $1\in\lC$ be a terminal object, then the identity of $1$ is the zero object in the slice category $\lC_{1/}$ of objects under $1$. In particular, the category $\lS_*=\lS_{*/}$ of pointed spaces is pointed. 
\end{example}
\begin{proposition}
  Let $\lD$ be a pointed $\infty$-category. Evaluation at the 0-sphere $S^0$ induces an equivalence \begin{equation}
    \begin{tikzcd}
      \Fun^\mL(\lS_*,\lC) \arrow[r] & \lC
    \end{tikzcd}
  \end{equation}
\end{proposition}
We now introduce the notion of a {triangle}.
\begin{definition}
  Let $\lC$ be a pointed $\infty$-category. A \emph{triangle} in $\lC$ is a commutative diagram of the form \begin{equation}\label{eq:triangle}
    \begin{tikzcd}
      X \arrow[r] \arrow[d] & Y \arrow[d] \\
      0 \arrow[r] & Z
    \end{tikzcd}
  \end{equation}A triangle is \emph{exact}, resp. \emph{coexact}, if it is a pullback, resp. pushout, square. 
\end{definition}
\begin{definition}
  Let $\lC$ be a pointed $\infty$-category. Denote by $\lC^{\Sigma}$, resp. $\lC^\Omega$, the full sub-category of $\Fun(\Delta^1\times\Delta^1,\lC)$ of coexact, resp. exact, triangles of the form 
  \begin{equation}\label{eq:exact}
  \begin{tikzcd}
   X \arrow[r]  \arrow[d] & 0 \arrow[d] \\ 
   0 \arrow[r] & Y
  \end{tikzcd},
\end{equation}
\end{definition}
If $\lC$ is finitely cocomplete, resp. complete, for every object $X$, resp. $Y$, there is a contractible space of coexact, resp. exact, triangles as \cref{eq:exact}. In particular, $\lC\simeq\lC^\Sigma$ and $\lC\simeq\lC^\Omega$.  
\begin{proposition}
  If $\lC$ is a finitely complete and cocomplete pointed $\infty$-category, define the following functors Then the functors
  \begin{equation}
    \begin{tikzcd}
      \Sigma:\lC \arrow[r] & \lC^\Sigma \arrow[r,"\mathrm{ev}_{(1,1)}"] & \lC  & & \Omega:\lC \arrow[r] & \lC^\Omega \arrow[r,"\mathrm{ev}_{(0,0)}"] & \lC
    \end{tikzcd}
  \end{equation}are adjoint ($\Sigma$ is left adjoint to $\Omega$). 
\end{proposition}

\begin{theorem}\label{thm:stable}
  Let $\lC$ be a finitely bicomplete pointed $\infty$-category. The following are equivalent:
  \begin{enumerate}
    \item A triangle is exact if and only if it is coexact. 
    \item $(\Sigma,\Omega)$ is an adjoint equivalence.
    \item A commutative square is a pullback if and only if it is a pushout. 
  \end{enumerate}
\end{theorem}
\begin{definition}
  A finite bicomplete pointed $\infty$-category $\lC$ satisfying any of the equivalent conditions in \cref{thm:stable} is called \emph{stable}.
\end{definition}
If $\bA$ denotes a nice abelian category, there is a \emph{derived $\infty$-category} of $\bA$, denoted $\lD(\bA)$ such that $\Ho\lD(\bA)=D(\bA)$ is the ordinary derived category of $\bA$, i.e. the localization of chain complex at quasi-isomorphisms. From homological algebra, it is known that $D(\bA)$ underlies the structure of a triangulated category, which turns out to be the 1-categorical reflection of $\lD(\bA)$ being stable. 
\begin{proposition}[{\cite[3.11]{lurie2017ha}}]
  If $\lC$ is a stable $\infty$-category, then $\Ho\lC$ has a canonical structure of a triangulated category. 
\end{proposition}To construct the stabilization of a pointed $\infty$-category, there are several approaches, such as reduced excisive functors on $\lS^\mathrm{fin}_*$, the category of pointed, finite spaces, see \cite[1.4.2.8]{lurie2017ha}. Here we consider the more explicit approach using spectrum objects.
\begin{definition}
  Let $\lC$ be a pointed $\infty$-category. A \emph{pre-spectrum object in $C$} consists of a functor $E:\bZ\times\bZ\to\lC$ such that $E(n,m)\simeq0$, for all $n\neq m$. Denote by $\PSp(\lC)$ the category of pre-spectrum objects. The functor $\Omega^{\infty-n}:\PSp(\lC)\to\lC$ is defined as evaluation at $(n,n)$. 
\end{definition}
For every $n$, the diagram \begin{equation}
  \begin{tikzcd}
    E(n,n) \arrow[r]\arrow[d] & 0 \arrow[d] \\
    0 \arrow[r] & E(n+1,n+1) 
  \end{tikzcd}
\end{equation}determines a pair of adjoint morphisms \[\alpha_n:\Sigma E(n,n)\to E(n+1,n+1),\qquad\beta_n:E(n,n)\to\Omega E(n+1,n+1)\]
\begin{definition}
  Let $\lC$ be a pointed $\infty$-category. A \emph{spectrum object in $\lC$} consists of a pre-spectrum object $E$ such that $\beta_n$ is an equivalence, for all $n$. Denote by $\Sp(\lC)\subseteq\PSp(\lC)$ the full sub-category of spectrum objects. 
\end{definition}
\begin{theorem}
  Let $\lC$ be a presentable pointed $\infty$-category, then $\Omega^{\infty-n}:\Sp(\lC)\to\lC$ admits a left adjoint $\Sigma^{\infty-n}:\lC\to\Sp(\lC)$, for every $n$. 
\end{theorem}
In particular, $\Sigma^\infty:\lC\to\Sp(\lC)$ has the following universal property. 
\begin{theorem}
 Let $\lC$ be a presentable pointed $\infty$-category. Given a stable $\infty$-category $\lD$, pre-composition by $\Sigma^\infty$ induces an equivalence 
 \begin{equation}
  \begin{tikzcd}
    \Fun^\mL(\Sp(\lC),\lD) \arrow[r] & \Fun^\mL(\lC,\lD)
  \end{tikzcd}
 \end{equation}In particular, for $\lC=\lS_*$, evaluation at the \emph{sphere spectrum} $\bS=\Sigma^\infty S^0$ induces an equivalence
 \begin{equation}
  \begin{tikzcd}
    \Fun^\mL(\Sp(\lS_*),\lD) \arrow[r] & \lD
  \end{tikzcd}
 \end{equation}
\end{theorem}
\begin{definition}
  The \emph{$\infty$-category of spectra} is the category of spectrum objects in pointed spaces. 
\end{definition}

\section{Generalized Cohomology Theories}

We shall now use the language of $\infty$-categories to reformulate the concept of generalized cohomology theory \` a-l\`a Eilenberg-Steenrod. In this new context, we recall a representability theorem for cohomology theories by spectrum object. 
\begin{remark}
  Denote by $\set^\bZ$ the category of $\bZ$-indexes families of sets. Given an object $S$ and $n\in\bZ$, denote by $\Sigma^n S$ the shifted family $(\Sigma^nS)_i=S_{i-n}$.  
\end{remark}
\begin{definition}[{\cite[1.4.1.6]{lurie2017ha}}]\label{def:cohthe}
  Let $\lC$ be a finitely cocomplete pointed $\infty$-category, $\Sigma_\lC:\lC \to \lC$ the induced suspension functor. A \emph{generalized cohomology theory} is a functor $H:\Ho\lC^{\op}\to\set^\bZ$ together with a natural isomorphism $\partial:\Sigma H\to H\Sigma_\lC$ such that:
  \begin{itemize}
    \item $H$ preserves arbitrary products. In particular, $H^n(0)$ is the one-point set. Given an object $X$, the unique morphism $X\to0$ induces an element $*\simeq H^n(0)\to H^n(X)$, which we denote by $0$. 
    \item Given a coexact triangle $X'\to X\to X''$, if $\eta\in H^n(X)$ has image $0\in H^n(X'')$, then it lies in the image of $H^n(X')\to H^n(X)$. 
  \end{itemize}
\end{definition}
\begin{theorem}[{\cite[1.4.1.10]{lurie2017ha}}]
 Let $\lC$ be a nice $\infty$-category and $(H,\partial)$ a generalized cohomology theory, then, for every $n$, the functor $H^n$ is representable by an object $E(n)$. 
\end{theorem}
The natural isomorphism $\partial$ translates into an equivalence $E(n)\simeq\Omega E(n+1)$, which is then used to construct a spectrum object representing the cohomology theory $H^n$, see \cite[1.4.1.11]{lurie2017ha}.
\begin{remark}
  For $\lC=\lS_*$, the above definition of cohomology theory reduces to the classical Eilenberg-Steenrod definition. Since $\lS_*$ is nice, we thus recover the classical \emph{Brown representability theorem}. 
\end{remark}

{\footnotesize
\bibliographystyle{alpha}
\bibliography{main}
}
\end{document}