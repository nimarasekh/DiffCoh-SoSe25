\documentclass[10pt]{amsart}
\usepackage{amsmath,amsthm,amssymb,amsfonts,mathabx}
\usepackage{eucal}
\usepackage{tikz}
\usepackage{tikz-cd}
\usepackage{enumerate}
\usepackage{enumitem}
\usepackage[colorlinks=true,linkcolor=red, citecolor = blue]{hyperref}
\usepackage[margin=2.5cm]{geometry}
\setlength{\marginparwidth}{2cm}

\usepackage[nameinlink,capitalise,noabbrev]{cleveref}

\usepackage[textwidth=2cm, textsize=small, colorinlistoftodos]{todonotes}

\newcommand{\fA}{\mathbf{A}}
\newcommand{\C}{\mathcal{C}}
\newcommand{\bC}{\mathbb{C}}
\newcommand{\kC}{\mathfrak{C}}
\newcommand{\D}{\mathcal{D}}
\newcommand{\I}{\mathcal{I}}
\newcommand{\mN}{\mathrm{N}}
\newcommand{\mL}{\mathrm{L}}
\newcommand{\J}{\mathcal{J}}
\newcommand{\s}{\mathcal{S}}
\newcommand{\bR}{\mathbb{R}}
\newcommand{\T}{\mathcal{T}}
\newcommand{\bS}{\mathbb{S}}
\newcommand{\bZ}{\mathbb{Z}}


\newcommand{\set}{\mathcal{S}\mathrm{et}}
\newcommand{\Sp}{\mathcal{S}\mathrm{p}}
\newcommand{\PSp}{\mathcal{P}\Sp}
\newcommand{\cat}{\mathcal{C}\mathrm{at}}
\newcommand{\scat}{s\mathcal{C}\mathrm{at}}
\newcommand{\sset}{s\mathcal{S}\mathrm{et}}
\newcommand{\Fun}{\mathcal{F}\mathrm{un}}
\newcommand{\Top}{\mathcal{T}\mathrm{op}}
\newcommand{\Kan}{\mathcal{K}\mathrm{an}}
\newcommand{\Ab}{\mathbf{Ab}}
\newcommand{\Hom}{\mathrm{Hom}}
\newcommand{\Map}{\mathrm{Map}}
\newcommand{\BDelta}{\mathbf{\Delta}} % Bold-face Delta
\newcommand{\im}{\mathrm{Im}} % image
\newcommand{\op}{\mathrm{op}} % opposite
\newcommand{\Ho}{\mathrm{Ho}} % Homotopy category 
\newcommand{\grpd}{\mathcal{G}\mathrm{rpd}} % category of groupoids
\newcommand{\Pshv}{\mathcal{PS}\mathrm{hv}} % category of pre-sheaves


\newcommand{\bbefamily}{\fontencoding{U}\fontfamily{bbold}\selectfont}
\newcommand{\textbbe}[1]{{\bbefamily #1}}
\DeclareMathAlphabet{\mathbbe}{U}{bbold}{m}{n}

\def\DDelta{{\mathbbe{\Delta}}}
\newcommand{\DD}{\DDelta}

\newcommand{\adjun}[4]{
\begin{tikzcd}[row sep=0.5in, column sep=0.5in]
 #1  \arrow[r, shift left=1.8, "#3"] \pgfmatrixnextcell
 #2 \arrow[l, shift left=1.6, "#4", "\bot"'] 
\end{tikzcd}
}

\newcommand{\simpset}[7]{
 \begin{tikzcd}[row sep=0.5in, column sep=0.5in]
   #1 \arrow[r, shorten >=1ex,shorten <=1ex]
   \pgfmatrixnextcell #2 
   \arrow[l, shift left=1.2, "#5"] \arrow[l, shift right=1.2, "#4"'] 
   \arrow[r, shift right, shorten >=1ex,shorten <=1ex ] \arrow[r, shift left, shorten >=1ex,shorten <=1ex] 
   \pgfmatrixnextcell #3 
   \arrow[l] \arrow[l, shift left=2, "#7"] \arrow[l, shift right=2, "#6 "'] 
   \arrow[r, shorten >=1ex,shorten <=1ex] \arrow[r, shift left=2, shorten >=1ex,shorten <=1ex] \arrow[r, shift right=2, 
   shorten >=1ex,shorten <=1ex]
   \pgfmatrixnextcell \cdots 
   \arrow[l, shift right=1] \arrow[l, shift left=1] \arrow[l, shift right=3] \arrow[l, shift left=3] 
 \end{tikzcd}
}


%% N.R. notes
\newcommand{\nrnote}[1]{\todo[color=green!40,linecolor=green!40!black,size=\tiny]{#1}}
\newcommand{\nrmpar}[1]{\todo[noline,color=green!40,linecolor=green!40!black,
  size=\tiny]{#1}}
\newcommand{\nrnoteil}[1]{\ \todo[inline,color=green!40,linecolor=green!40!black,size=\normalsize]{#1}}

\newtheorem{theorem}[equation]{Theorem}
\newtheorem{lemma}[equation]{Lemma}
\newtheorem{proposition}[equation]{Proposition}
\newtheorem{corollary}[equation]{Corollary}
% \newtheorem{statement}[section]{Statement}

\theoremstyle{definition}
\newtheorem{definition}[equation]{Definition}
\newtheorem{example}[equation]{Example}
% \newtheorem{attone}[equation]{Attention}

\theoremstyle{remark}
\newtheorem{remark}[equation]{Remark}
% \newtheorem{intone}[equation]{Intuition}
\newtheorem{notation}[equation]{Notation}
% \newtheorem{queone}[equation]{Question}
% \newtheorem{conjone}[equation]{Conjecture}
\newtheorem{warning}[equation]{Warning}

\title{Differential Cohomology Seminar 2}
\date{06.05.2025 $\&$ 13.05.2025}
\author{Talk by Matthias Frerichs}

\begin{document}
\maketitle

In this lecture we want to learn the basics of $\infty$-category theory. For the $\infty$-categorical background, we broadly follow \cite{groth2010inftycats} and a little \cite{lurie2009htt}.

\section{Basics on \texorpdfstring{$(\infty,1)$}{(oo,1)}-categories}

$(\infty,1)$-categories have different models that capture its essence. The first model are \emph{quasi-categories}. 
\begin{definition}
  Given a natural number $n$, let $\langle n\rangle$ denote the linearly ordered set $\{0,\cdots,n\}$. The \emph{simplex category} $\BDelta$ is the category of finite linerly ordered sets $\langle n\rangle$, for every $n$, and monotone functions.
\end{definition}
\begin{definition}
  Given $0\leq i\leq n$, the \emph{$i$-face map} is the unique injective map $\delta^i_n:\langle n-1\rangle\to\langle n\rangle$ missing $i$. The \emph{$i$-degeneracy map} is the unique surjective map $\sigma^i_n:\langle n+1\rangle\to\langle n\rangle$ such that $i$ and $i+1$ have the same image. 
\end{definition}
\begin{theorem}
  As a category, $\BDelta$ is generated from the face and degeneracy maps subject to the \emph{simplicial identities}, i.e. 
  \begin{equation}
    \delta_{n+1}^i\delta_n^j=\delta_{n+1}^{j+1}\delta_n^i,\qquad i\leq j
  \end{equation}
  \begin{equation}
    \sigma_{n-1}^j\sigma_n^i=\sigma_{n-1}^i\sigma_{n}^{j+1},
    \qquad i\leq j
  \end{equation}
  \begin{equation}
    \sigma_{n}^j\delta_{n+1}^i=\left\{\begin{array}{ll}
      \delta_{n}^i\sigma_{n-1}^{j-1}, & i<j \\ 
      1, & i=j\\ 
      \delta_n^{i-1}\sigma_{n-1}^j, & i>j
    \end{array}\right.
  \end{equation}
\end{theorem}
\begin{proof}
  Omitted.
\end{proof}
\begin{definition}
  A \emph{simplicial set} is a contravariant functor $X:\BDelta^{\op} \to \set$. Denote by $\sset$ the category of simplicial sets. $X_n:=X(\langle n\rangle)$ is the set of \emph{$n$-simplices}. 
\end{definition}
By only representing the face maps, we can depict a simplicial set as follows: 
\begin{equation}
  \begin{tikzcd}
    X_0  & X_1 \arrow[l,shift right]\arrow[l,shift left] & X_2 \arrow[l,shift right=2]\arrow[l]\arrow[l,shift left=2] & \cdots \arrow[l,shift right=3]\arrow[l,shift right=1]\arrow[l,shift left=1]\arrow[l,shift left=3]
  \end{tikzcd}
\end{equation}
$\infty$-categories are then defined in terms of a lifting condition, for which we need to define horns.
\begin{definition}
  Let $\Delta^n$ denote the representable functor associated to $\langle n\rangle$. The face map $\delta^i_n$ induces a map of simplicial sets $d^i_n:\Delta^{n-1}\to\Delta^n$. The image of $d^i_n$ is called the \emph{$i$-face}. The \emph{$i$-horn} $\Lambda^{i,n}$ is the union of all faces, except the $i$-face. 
\end{definition}
\begin{remark}
  Another characterization of $\Lambda^{i,n}$ is the following: A $t$-simplex $f:\langle t\rangle\to\langle n\rangle$ is a $t$-simplex for $\Lambda^{i,n}$ if and only if there is $j\neq i$ not in the image of $f$. 
\end{remark}
\begin{definition}A simplicial set $X$ is a \emph{quasi-category} if, for every $0<i<n$ and solid diagram \begin{equation}
  \begin{tikzcd}
    \Lambda^{i,n} \arrow[d]\arrow[r] & X \\
    \Delta^n \arrow[ru,dashed]
  \end{tikzcd}
\end{equation}
there is a dashed arrow rendering the diagram commutative. If the above condition holds for every $0\leq i\leq n$, we call $X$ a \emph{Kan complex}.
\end{definition}
\begin{example}
  Let $\C$ be a category. The \emph{nerve of $\C$}, denoted $\mN\C$, is the simplicial set where the $n$-simplices are $\Hom_{\cat}(\langle n\rangle,\C)$. This defines a functor $\mN:\cat \to \sset$.
\end{example}
\begin{proposition}\label{prop:nerve}
  $\mN\C$ is a quasi-category, and a Kan complex if and only if $\C$ is a groupoid. 
\end{proposition}
\begin{proof}
Straightforward combinatorics. 
\end{proof}
\begin{remark}
  The nerve is a special case of the following construction: Let $\C$ be a category, $\Gamma:\BDelta\to\C$ a functor, then define $\mN_\Gamma$ as the composition \begin{equation}
    \begin{tikzcd}
      \C \arrow[r] & \Hom_{\cat}(\C^{op},\set) \arrow[r] & \sset
    \end{tikzcd}
  \end{equation}where the first functor is Yoneda, while the second is pre-composition with $\Gamma^\op$. In the case of the nerve, $\Gamma$ is the functor sending $\langle n\rangle$ to the linerly ordered set viewed as a category. On the other hand, assuming $\C$ is cocomplete, we can left Kan extend $\Gamma$ along the Yoneda functor $\BDelta\to\sset$, we denote by $\Ho_\Gamma$ the resulting functor $\sset\to\C$. 
\end{remark}
\begin{proposition}
  The pair $(\Ho_\Gamma,\mN_\Gamma):\C\to\sset$ is an adjoint pair. In the case of $\mN:\cat\to\sset$, the functor is fully faithful. 
\end{proposition}
\begin{proof}
  Abstract nonsense about left Kan extensions. Full faithfulness can be checked directly.  
\end{proof}
\begin{remark}
  $\Ho:\sset\to\cat$ is called the \emph{homotopy category} functor. If $X$ is a quasi-category, $\Ho X$ has $X_0$ as set of objects and homotopy classes of maps as morphism, see \cite[1.2.5]{land2021introduction}. 
\end{remark}
\begin{remark}
  Denote by $\scat$ the category of simplicially enriched categories. In \cite[1.1.5.1]{lurie2009htt}, Lurie constructs a cosimplicial object $\BDelta\to\scat$. The resulting nerve functor $\mN_\Delta:\scat\to\sset$ is called \emph{homotopy coherent nerve}. If $\C$ is a category enriched over $\infty$-groupoids, its homotopy coherent nerve is a $\infty$-category. The induced right adjoint is denoted $\kC$, the adjoint pair $(\kC,\mN_\Delta)$ underlies a Quillen equivalence. 
\end{remark}
Denote by $\Kan$ the category of Kan complexes. One can show that $\Kan$ is self-enriched, which motivates, together with \cref{prop:nerve}, the following definition: 
\begin{definition}
  $\s:=\mN_\Delta(\Kan)$ is called the \emph{quasi-category of $\infty$-groupoids}. 
\end{definition}

\section{Accessible and Presentable Categories}

In general, a limit, resp. colimit, preserving functor need not have a left, resp. right, adjoint. Here we wish to introduce a rather large class of quasi-categories for which the previous statement holds. Let $\kappa$ denote a regular cardinal. 
\begin{definition}
  A category $\I$ is \emph{$\kappa$-filtered} if, for every $\J$ with $<\kappa$ many morphisms, every diagram $\J\to\I$ has a cocone. A functor $F:\C \to \D$ is \emph{$\kappa$-accessible} if it preserves $\kappa$-filtered colimits. Given a category $\C$, an object $X$ is \emph{$\kappa$-compact} if $\Hom_{\C}(X,-):\C\to\set$ is $\kappa$-accessible. 
\end{definition}
\begin{definition}
  A category $\C$ is \emph{$\kappa$-accessible} if there exists a set $S\subseteq\C_0$ of $\kappa$-compact objects that generate $\C$ under $\kappa$-filtered colimits. A category is \emph{accessible} if it is $\kappa$-accessible, for some regular cardinal $\kappa$. 
\end{definition}
\begin{definition}
  A category $\C$ that is accessible and cocomplete is called \emph{locally presentable}.
\end{definition}
\begin{theorem}
  Let $\C$ be a category, then $\C$ is locally presentable if and only if there exists a small category $S$ such that the induced functor $\C\to\Pshv(S)$ is a fully faithful, accessible right adjoint. 
\end{theorem}
\begin{theorem}
  Let $\C,\D$ be locally presentable categories. A functor $F\colon \C \to \D$ is a left, resp. right, adjoint if and only if it preserves colimits, resp. it preserves limits and is accessible.
\end{theorem}
We now generalize this to quasi-categories.
\begin{definition}
  Given a simplicial set $X$, denote by $X^\op$ the simplicial set obtained by reversing the structure maps: $X^\op_n=X_n$, for all $n$, and \[\begin{array}{c}
    (d_i:X^\op_n\to X^\op_{n-1})=(d_{n-i}:X_n\to X_{n-1})\\
    (s_i:X^{\op}_n\to X^\op_{n+1})=(s_{n-i}:X_n\to X_{n+1})
\end{array}\]
\end{definition}
If $X$ is a quasi-category, so is $X^\op$. 
\begin{definition}
  Let $X$ be a quasi-category. The \emph{quasi-category of simplicial presheaves} is defined as $\Fun(X^\op,\s):=\Hom_{\sset}(X^{\op},\s)$.
\end{definition}
\begin{theorem}[Yoneda]
  Given a quasi-category $X$, the mapping space functor $X^\op\times X\to\s$ adjoints to a functor $y:X\to\widehat{X}:=\Fun(X^\op,\s)$, called the \emph{Yoneda embedding}. Given a cocomplete quasi-category $\C$, pre-composition by $y$ induces an equivalence \begin{equation}
    \begin{tikzcd}
      \Fun^\mL(\widehat{X},\C) \arrow[r] & \Fun(X,\C)
    \end{tikzcd}
  \end{equation}where $\Fun^\mL(\widehat{X},\C)$ denotes the category of colimit preserving functors $\widehat{X}\to\C$. 
\end{theorem}
In other words, $\widehat{X}$ is the free cocompletion of $X$. The inverse equivalence if given by taking left Kan extensions. The definition of accessible category transfers directly to the $\infty$-categorical setting. 
\begin{theorem}\label{thm:locpresquasi}
  A quasi-category $X$ is locally presentable (cocomplete and accessible) if and only if there is a small sub-quasi-category $S$ such that $X\to\Fun(S^\op,\s)$ is a fully faithful, accessible right adjoint. 
\end{theorem}
\begin{remark}
  In view of \cref{thm:locpresquasi}, one can define a category to be \emph{locally presentable} if it is the accessible right localization of a pre-sheaf category for some small quasi-category $S$. In particular, every pre-sheaf category is locally presentable. 
\end{remark}
\begin{theorem}
  Let $X,Y$ be presentable $\infty$-categories, then a functor $f:X\to Y$ is a left, resp. right, adjoint if and only if it preserves colimits, resp. it preserves limits and is accessible.
\end{theorem}
\section{Stable \texorpdfstring{$\infty$}{oo}-Categories and Spectra}
We now use the $\infty$-categorical framework to study spectra. The study of spectra originates from the study of \emph{stable phenomena}, i.e. patterns appearing after repeated application of the suspension functor $\Sigma:\Top_*\to\Top_*$.
\begin{example}
  Let $(\Sigma,\Omega):\Top_{*/}\to\Top_{*/}$ be the suspension-loop adjunction on pointed topological spaces. \emph{Freudenthal Suspension Theorem} states that, if $X$ is a $n$-connected space, the adjunction unit $X\to\Omega\Sigma X$ is $2n$-connected. If $X$ is connected, $S^n\wedge X$ is $n$-connected, so $\Sigma^nX\to\Omega\Sigma^{n+1}X$ is $2n$-connected. In particular, $\pi_i(\Sigma^nX)\to\pi_{i+1}(\Sigma^{n+1}X)$ is an isomorphism, for all $i<2n$. The group $\pi_i(\Sigma^nX)$ is denoted $\pi_{n-i}^s(X)$, called the \emph{$(n-i)$-stable homotopy group of $X$}. 
\end{example}
\nrnote{More examples?}
\begin{definition}
  Let $\C$ be an $\infty$-category, an object $0$ that is both initial and terminal is called 
  \emph{zero object}. A category $\C$ with a zero object is called a \emph{pointed category}.
\end{definition}
\begin{example}
  Let $1\in\C$ be a terminal object, then the identity of $1$ is the zero object in the slice category $\C_{1/}$ of objects under $1$. In particular, the category $\s_*=\s_{*/}$ of pointed spaces is pointed. 
\end{example}
\begin{proposition}
  Let $\D$ be a pointed $\infty$-category. Evaluation at the 0-sphere $S^0$ induces an equivalence \begin{equation}
    \begin{tikzcd}
      \Fun^\mL(\s_*,\C) \arrow[r] & \C
    \end{tikzcd}
  \end{equation}where $\Fun^\mL(\s_*,\C)$ denotes the category of colimit preserving functors $\s_*\to\C$. 
\end{proposition}
We now introduce the notion of a {triangle}.
\begin{definition}
  Let $\C$ be a pointed $\infty$-category. A \emph{triangle} in $\C$ is a commutative diagram of the form \begin{equation}\label{eq:triangle}
    \begin{tikzcd}
      X \arrow[r] \arrow[d] & Y \arrow[d] \\
      0 \arrow[r] & Z
    \end{tikzcd}
  \end{equation}A triangle is \emph{exact}, resp. \emph{coexact}, if it is a pullback, resp. pushout, square. 
\end{definition}
\begin{definition}
  Let $\C$ be a pointed $\infty$-category. Denote by $\C^{\Sigma}$, resp. $\C^\Omega$, the full sub-category of $\Fun(\Delta^1\times\Delta^1,\C)$ of coexact, resp. exact, triangles of the form 
  \begin{equation}\label{eq:exact}
  \begin{tikzcd}
   X \arrow[r]  \arrow[d] & 0 \arrow[d] \\ 
   0 \arrow[r] & Y
  \end{tikzcd},
\end{equation}
\end{definition}
If $\C$ is finitely cocomplete, resp. complete, for every object $X$, resp. $Y$, there is a contractible space of coexact, resp. exact, triangles as \cref{eq:exact}. In particular, $\C\simeq\C^\Sigma$ and $\C\simeq\C^\Omega$.  
\begin{proposition}
  If $\C$ is a finitely complete and cocomplete pointed $\infty$-category, define the following functors Then the functors
  \begin{equation}
    \begin{tikzcd}
      \Sigma:\C \arrow[r] & \C^\Sigma \arrow[r,"\mathrm{ev}_{(1,1)}"] & \C  & & \Omega:\C \arrow[r] & \C^\Omega \arrow[r,"\mathrm{ev}_{(0,0)}"] & \C
    \end{tikzcd}
  \end{equation}are adjoint ($\Sigma$ is left adjoint to $\Omega$). 
\end{proposition}

\begin{theorem}\label{thm:stable}
  Let $\C$ be a finitely bicomplete pointed $\infty$-category. The following are equivalent:
  \begin{enumerate}
    \item A triangle is exact if and only if it is coexact. 
    \item $(\Sigma,\Omega)$ is an adjoint equivalence.
    \item A commutative square is a pullback if and only if it is a pushout. 
  \end{enumerate}
\end{theorem}
\begin{definition}
  A finite bicomplete pointed $\infty$-category $\C$ satisfying any of the equivalent conditions in \cref{thm:stable} is called \emph{stable}.
\end{definition}
In homological algebra, the derived category $\D(\fA)$ of an abelian category $\fA$ underlies the structure of a triangulated category. In higher category theory, the derived category $\D_\infty(\fA)$ is constructed as the higher categorical localization of chain complexes at quasi-isomorphisms, then $\Ho\D_\infty(\fA)\simeq\D(\fA)$ and the triangulated structure on $\D(\fA)$ is a reflection of $\D_\infty(\fA)$ being a stable $\infty$-category. 
\begin{proposition}[{\cite[3.11]{lurie2017ha}}]
  If $\C$ is a stable $\infty$-category, then $\Ho\C$ has a canonical structure of a triangulated category. 
\end{proposition}To construct the stabilization of a pointed $\infty$-category, there are several approaches, such as reduced excisive functors on $\s^\mathrm{fin}_*$, the category of pointed, finite spaces, see \cite[1.4.2.8]{lurie2017ha}. Here we consider the more explicit approach using spectrum objects.
\begin{definition}
  Let $\C$ be a pointed $\infty$-category. A \emph{pre-spectrum object in $C$} consists of a functor $E:\bZ\times\bZ\to\C$ such that $E(n,m)\simeq0$, for all $n\neq m$. Denote by $\PSp(\C)$ the category of pre-spectrum objects. The functor $\Omega^{\infty-n}:\PSp(\C)\to\C$ is defined as evaluation at $(n,n)$. 
\end{definition}
For every $n$, the diagram \begin{equation}
  \begin{tikzcd}
    E(n,n) \arrow[r]\arrow[d] & 0 \arrow[d] \\
    0 \arrow[r] & E(n+1,n+1) 
  \end{tikzcd}
\end{equation}determines a pair of adjoint morphisms \[\alpha_n:\Sigma E(n,n)\to E(n+1,n+1),\qquad\beta_n:E(n,n)\to\Omega E(n+1,n+1)\]
\begin{definition}
  Let $\C$ be a pointed $\infty$-category. A \emph{spectrum object in $\C$} consists of a pre-spectrum object $E$ such that $\beta_n$ is an equivalence, for all $n$. Denote by $\Sp(\C)\subseteq\PSp(\C)$ the full sub-category of spectrum objects. 
\end{definition}
\begin{theorem}
  Let $\C$ be a presentable pointed $\infty$-category, then $\Omega^{\infty-n}:\Sp(\C)\to\C$ admits a left adjoint $\Sigma^{\infty-n}:\C\to\Sp(\C)$, for every $n$. 
\end{theorem}
In particular, $\Sigma^\infty:\C\to\Sp(\C)$ has the following universal property. 
\begin{theorem}
 Let $\C$ be a presentable pointed $\infty$-category. Given a stable $\infty$-category $\D$, pre-composition by $\Sigma^\infty$ induces an equivalence 
 \begin{equation}
  \begin{tikzcd}
    \Fun^\mL(\Sp(\C),\D) \arrow[r] & \Fun^\mL(\C,\D)
  \end{tikzcd}
 \end{equation}In particular, for $\C=\s_*$, evaluation at the \emph{sphere spectrum} $\bS=\Sigma^\infty S^0$ induces an equivalence
 \begin{equation}
  \begin{tikzcd}
    \Fun^\mL(\Sp(\s_*),\D) \arrow[r] & \D
  \end{tikzcd}
 \end{equation}
\end{theorem}
\begin{definition}
  The \emph{$\infty$-category of spectra} is the category of spectrum objects in pointed spaces. 
\end{definition}

\section{Generalized Cohomology Theories}

We shall now use the language of $\infty$-categories to reformulate the concept of generalized cohomology theory \` a-l\`a Eilenberg-Steenrod. In this new context, we recall a representability theorem for cohomology theories by spectrum object. 
\begin{remark}
  Denote by $\set^\bZ$ the category of $\bZ$-indexes families of sets. Given an object $S$ and $n\in\bZ$, denote by $\Sigma^n S$ the shifted family $(\Sigma^nS)_i=S_{i-n}$.  
\end{remark}
\begin{definition}[{\cite[1.4.1.6]{lurie2017ha}}]\label{def:cohthe}
  Let $\C$ be a finitely cocomplete pointed $\infty$-category, $\Sigma_\C:\C \to \C$ the induced suspension functor. A \emph{generalized cohomology theory} is a functor $H:\Ho\C^{\op}\to\set^\bZ$ together with a natural isomorphism $\partial:\Sigma H\to H\Sigma_\C$ such that:
  \begin{itemize}
    \item $H$ preserves arbitrary products. In particular, $H^n(0)$ is the one-point set. Given an object $X$, the unique morphism $X\to0$ induces an element $*\simeq H^n(0)\to H^n(X)$, which we denote by $0$. 
    \item Given a coexact triangle $X'\to X\to X''$, if $\eta\in H^n(X)$ has image $0\in H^n(X'')$, then it lies in the image of $H^n(X')\to H^n(X)$. 
  \end{itemize}
\end{definition}
\begin{theorem}[{\cite[1.4.1.10]{lurie2017ha}}]
 Let $\C$ be a nice $\infty$-category and $(H,\partial)$ a generalized cohomology theory, then, for every $n$, the functor $H^n$ is representable by an object $E(n)$. 
\end{theorem}
The natural isomorphism $\partial$ translates into an equivalence $E(n)\simeq\Omega E(n+1)$, which is then used to construct a spectrum object representing the cohomology theory $H^n$, see \cite[1.4.1.11]{lurie2017ha}.
\begin{remark}
  For $\C=\s_*$, the above definition of cohomology theory reduces to the classical Eilenberg-Steenrod definition. Since $\s_*$ is nice, we thus recover the classical \emph{Brown representability theorem}. 
\end{remark}

{\footnotesize
\bibliographystyle{alpha}
\bibliography{main}
}
\end{document}