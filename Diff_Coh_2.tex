\documentclass[10pt]{amsart}
\usepackage{amsmath,amsthm,amssymb,amsfonts,mathabx}
\usepackage{eucal}
\usepackage{tikz}
\usepackage{tikz-cd}
\usepackage{enumerate}
\usepackage{enumitem}
\usepackage[colorlinks=true,linkcolor=red, citecolor = blue]{hyperref}
\usepackage[margin=2.5cm]{geometry}
\setlength{\marginparwidth}{2cm}

\usepackage[nameinlink,capitalise,noabbrev]{cleveref}

\usepackage[textwidth=2cm, textsize=small, colorinlistoftodos]{todonotes}

\newcommand{\C}{\mathcal{C}}
\newcommand{\bC}{\mathbb{C}}
\newcommand{\kC}{\mathfrak{C}}
\newcommand{\D}{\mathcal{D}}
\newcommand{\I}{\mathcal{I}}
\newcommand{\mN}{\mathrm{N}}
\newcommand{\J}{\mathcal{J}}
\newcommand{\s}{\mathcal{S}}
\newcommand{\bR}{\mathbb{R}}
\newcommand{\bS}{\mathbb{S}}
\newcommand{\bZ}{\mathbb{Z}}


\newcommand{\set}{\mathcal{S}\mathrm{et}}
\newcommand{\Sp}{\mathcal{S}\mathrm{p}}
\newcommand{\cat}{\mathcal{C}\mathrm{at}}
\newcommand{\scat}{s\mathcal{C}\mathrm{at}}
\newcommand{\sset}{s\mathcal{S}\mathrm{et}}
\newcommand{\Fun}{\mathrm{Fun}}
\newcommand{\Top}{\mathcal{T}\mathrm{op}}
\newcommand{\Kan}{\mathcal{K}\mathrm{an}}
\newcommand{\Ab}{\mathcal{A}\mathrm{b}}
\newcommand{\Hom}{\mathrm{Hom}}
\newcommand{\Map}{\mathrm{Map}}
\newcommand{\BDelta}{\mathbf{\Delta}} % Bold-face Delta
\newcommand{\im}{\mathrm{Im}} % image
\newcommand{\op}{\mathrm{op}} % opposite
\newcommand{\Ho}{\mathrm{Ho}} % Homotopy category 
\newcommand{\grpd}{\mathcal{G}\mathrm{rpd}}
\newcommand{\Pshv}{\mathcal{PS}\mathrm{hv}}


\newcommand{\bbefamily}{\fontencoding{U}\fontfamily{bbold}\selectfont}
\newcommand{\textbbe}[1]{{\bbefamily #1}}
\DeclareMathAlphabet{\mathbbe}{U}{bbold}{m}{n}

\def\DDelta{{\mathbbe{\Delta}}}
\newcommand{\DD}{\DDelta}

\newcommand{\adjun}[4]{
\begin{tikzcd}[row sep=0.5in, column sep=0.5in]
 #1  \arrow[r, shift left=1.8, "#3"] \pgfmatrixnextcell
 #2 \arrow[l, shift left=1.6, "#4", "\bot"'] 
\end{tikzcd}
}

\newcommand{\simpset}[7]{
 \begin{tikzcd}[row sep=0.5in, column sep=0.5in]
   #1 \arrow[r, shorten >=1ex,shorten <=1ex]
   \pgfmatrixnextcell #2 
   \arrow[l, shift left=1.2, "#5"] \arrow[l, shift right=1.2, "#4"'] 
   \arrow[r, shift right, shorten >=1ex,shorten <=1ex ] \arrow[r, shift left, shorten >=1ex,shorten <=1ex] 
   \pgfmatrixnextcell #3 
   \arrow[l] \arrow[l, shift left=2, "#7"] \arrow[l, shift right=2, "#6 "'] 
   \arrow[r, shorten >=1ex,shorten <=1ex] \arrow[r, shift left=2, shorten >=1ex,shorten <=1ex] \arrow[r, shift right=2, 
   shorten >=1ex,shorten <=1ex]
   \pgfmatrixnextcell \cdots 
   \arrow[l, shift right=1] \arrow[l, shift left=1] \arrow[l, shift right=3] \arrow[l, shift left=3] 
 \end{tikzcd}
}


%% N.R. notes
\newcommand{\nrnote}[1]{\todo[color=green!40,linecolor=green!40!black,size=\tiny]{#1}}
\newcommand{\nrmpar}[1]{\todo[noline,color=green!40,linecolor=green!40!black,
  size=\tiny]{#1}}
\newcommand{\nrnoteil}[1]{\ \todo[inline,color=green!40,linecolor=green!40!black,size=\normalsize]{#1}}

\newtheorem{theorem}[equation]{Theorem}
\newtheorem{lemma}[equation]{Lemma}
\newtheorem{proposition}[equation]{Proposition}
\newtheorem{corollary}[equation]{Corollary}
% \newtheorem{statement}[section]{Statement}

\theoremstyle{definition}
\newtheorem{definition}[equation]{Definition}
\newtheorem{example}[equation]{Example}
% \newtheorem{attone}[equation]{Attention}

\theoremstyle{remark}
\newtheorem{remark}[equation]{Remark}
% \newtheorem{intone}[equation]{Intuition}
\newtheorem{notation}[equation]{Notation}
% \newtheorem{queone}[equation]{Question}
% \newtheorem{conjone}[equation]{Conjecture}
\newtheorem{warning}[equation]{Warning}

\title{Differential Cohomology Seminar 2}
\date{06.05.2025 $\&$ 13.05.2025}
\author{Talk by Matthias Frerichs}

\begin{document}
\maketitle

In this lecture we want to learn the basics of $\infty$-category theory. For the $\infty$-categorical background, we broadly follow \cite{groth2010inftycats} and a little \cite{lurie2009htt}.

\section{Basics on \texorpdfstring{$(\infty,1)$}{(oo,1)}-categories}

$(\infty,1)$-categories have different models that capture its essence. The first model are \emph{quasi-categories}. 
\begin{definition}
  Given a natural number $n$, let $\langle n\rangle$ denote the linearly ordered set $\{0,\cdots,n\}$. The \emph{simplex category} $\BDelta$ is the category of finite linerly ordered sets $\langle n\rangle$, for every $n$, and monotone functions.
\end{definition}
\begin{definition}
  Given $0\leq i\leq n$, the \emph{$i$-face map} is the unique injective map $\delta^i_n:\langle n-1\rangle\to\langle n\rangle$ missing $i$. The \emph{$i$-degeneracy map} is the unique surjective map $\sigma^i_n:\langle n+1\rangle\to\langle n\rangle$ such that $i$ and $i+1$ have the same image. 
\end{definition}
\begin{theorem}
  As a category, $\BDelta$ is generated from the face and degeneracy maps subject to the \emph{simplicial identities}, i.e. 
  \begin{equation}
    \delta_{n+1}^i\delta_n^j=\delta_{n+1}^{j+1}\delta_n^i,\qquad i\leq j
  \end{equation}
  \begin{equation}
    \sigma_{n-1}^j\sigma_n^i=\sigma_{n-1}^i\sigma_{n}^{j+1},
    \qquad i\leq j
  \end{equation}
  \begin{equation}
    \sigma_{n}^j\delta_{n+1}^i=\left\{\begin{array}{ll}
      \delta_{n}^i\sigma_{n-1}^{j-1}, & i<j \\ 
      1, & i=j\\ 
      \delta_n^{i-1}\sigma_{n-1}^j, & i>j
    \end{array}\right.
  \end{equation}
\end{theorem}
\begin{proof}
  Omitted.
\end{proof}
\begin{definition}
  A \emph{simplicial set} is a contravariant functor $X:\BDelta^{\op} \to \set$. Denote by $\sset$ the category of simplicial sets. $X_n:=X(\langle n\rangle)$ is the set of \emph{$n$-simplices}. 
\end{definition}
By only representing the face maps, we can depict a simplicial set as follows: 
\begin{equation}
  \begin{tikzcd}
    X_0  & X_1 \arrow[l,shift right]\arrow[l,shift left] & X_2 \arrow[l,shift right=2]\arrow[l]\arrow[l,shift left=2] & \cdots \arrow[l,shift right=3]\arrow[l,shift right=1]\arrow[l,shift left=1]\arrow[l,shift left=3]
  \end{tikzcd}
\end{equation}
$\infty$-categories are then defined in terms of a lifting condition, for which we need to define horns.
\begin{definition}
  Let $\Delta^n$ denote the representable functor associated to $\langle n\rangle$. The face map $\delta^i_n$ induces a map of simplicial sets $d^i_n:\Delta^{n-1}\to\Delta^n$. The image of $d^i_n$ is called the \emph{$i$-face}. The \emph{$i$-horn} $\Lambda^{i,n}$ is the union of all faces, except the $i$-face. 
\end{definition}
\begin{remark}
  Another characterization of $\Lambda^{i,n}$ is the following: A $t$-simplex $f:\langle t\rangle\to\langle n\rangle$ is a $t$-simplex for $\Lambda^{i,n}$ if and only if there is $j\neq i$ not in the image of $f$. 
\end{remark}
\begin{definition}A simplicial set $X$ is a \emph{quasi-category} if, for every $0<i<n$ and solid diagram \begin{equation}
  \begin{tikzcd}
    \Lambda^{i,n} \arrow[d]\arrow[r] & X \\
    \Delta^n \arrow[ru,dashed]
  \end{tikzcd}
\end{equation}
there is a dashed arrow rendering the diagram commutative. If the above condition holds for every $0\leq i\leq n$, we call $X$ a \emph{Kan complex}.
\end{definition}
\begin{example}
  Let $\C$ be a category. The \emph{nerve of $\C$}, denoted $\mN\C$, is the simplicial set where the $n$-simplices are $\Hom_{\cat}(\langle n\rangle,\C)$. This defines a functor $\mN:\cat \to \sset$.
\end{example}
\begin{proposition}\label{prop:nerve}
  $\mN\C$ is a quasi-category, and a Kan complex if and only if $\C$ is a groupoid. 
\end{proposition}
\begin{proof}
Straightforward combinatorics. 
\end{proof}
\begin{remark}
  The nerve is a special case of the following construction: Let $\C$ be a category, $\Gamma:\BDelta\to\C$ a functor, then define $\mN_\Gamma$ as the composition \begin{equation}
    \begin{tikzcd}
      \C \arrow[r] & \Hom_{\cat}(\C^{op},\set) \arrow[r] & \sset
    \end{tikzcd}
  \end{equation}where the first functor is Yoneda, while the second is pre-composition with $\Gamma^\op$. In the case of the nerve, $\Gamma$ is the functor sending $\langle n\rangle$ to the linerly ordered set viewed as a category. On the other hand, assuming $\C$ is cocomplete, we can left Kan extend $\Gamma$ along the Yoneda functor $\BDelta\to\sset$, we denote by $\Ho_\Gamma$ the resulting functor $\sset\to\C$. 
\end{remark}
\begin{proposition}
  The pair $(\Ho_\Gamma,\mN_\Gamma):\C\to\sset$ is an adjoint pair. In the case of $\mN:\cat\to\sset$, the functor is fully faithful. 
\end{proposition}
\begin{proof}
  Abstract nonsense about left Kan extensions. Full faithfulness can be checked directly.  
\end{proof}
\begin{remark}
  $\Ho:\sset\to\cat$ is called the \emph{homotopy category} functor. If $X$ is a quasi-category, $\Ho X$ has $X_0$ as set of objects and homotopy classes of maps as morphism, see \cite[1.2.5]{land2021introduction}. 
\end{remark}
\begin{remark}
  Denote by $\scat$ the category of simplicially enriched categories. In \cite[1.1.5.1]{lurie2009htt}, Lurie constructs a cosimplicial object $\BDelta\to\scat$. The resulting nerve functor $\mN_\Delta:\scat\to\sset$ is called \emph{homotopy coherent nerve}. If $\C$ is a category enriched over $\infty$-groupoids, its homotopy coherent nerve is a $\infty$-category. The induced right adjoint is denoted $\kC$, the adjoint pair $(\kC,\mN_\Delta)$ underlies a Quillen equivalence. 
\end{remark}
Denote by $\Kan$ the category of Kan complexes. One can show that $\Kan$ is self-enriched, which motivates, together with \cref{prop:nerve}, the following definition: 
\begin{definition}
  $\s:=\mN_\Delta(\Kan)$ is called the \emph{quasi-category of $\infty$-groupoids}. 
\end{definition}

\section{Accessible and Presentable Categories}

In general, a limit, resp. colimit, preserving functor need not have a left, resp. right, adjoint. Here we wish to introduce a rather large class of quasi-categories for which the previous statement holds. Let $\kappa$ denote a regular cardinal. 
\begin{definition}
  A category $\I$ is \emph{$\kappa$-filtered} if, for every $\J$ with $<\kappa$ many morphisms, every diagram $\J\to\I$ has a cocone. A functor $F:\C \to \D$ is \emph{$\kappa$-accessible} if it preserves $\kappa$-filtered colimits. Given a category $\C$, an object $X$ is \emph{$\kappa$-compact} if $\Hom_{\C}(X,-):\C\to\set$ is $\kappa$-accessible. 
\end{definition}
\begin{definition}
  A category $\C$ is \emph{$\kappa$-accessible} if there exists a set $S\subseteq\C_0$ of $\kappa$-compact objects that generate $\C$ under $\kappa$-filtered colimits. A category is \emph{accessible} if it is $\kappa$-accessible, for some regular cardinal $\kappa$. 
\end{definition}
\begin{definition}
  A category $\C$ that is accessible and cocomplete is called \emph{locally presentable}.
\end{definition}
\begin{theorem}
  Let $\C$ be a category, then $\C$ is locally presentable if and only if there exists a small category $S$ such that the induced functor $\C\to\Pshv(S)$ is fully faithful, accessible, and a right adjoint. 
\end{theorem}
\begin{theorem}
  Let $\C,\D$ be locally presentable categories. A functor \item $F\colon \C \to \D$ is a left, resp. right, adjoint if and only if it preserves colimits, resp. limits and is accessible.
\end{theorem}
We now generalize this to quasi-categories.
\begin{definition}
  Given a simplicial set $X$, denote by $X^\op$ the simplicial set obtained by reversing the structure maps: $X^\op_n=X_n$, for all $n$, and \[\begin{array}{c}
    (d_i:X^\op_n\to X^\op_{n-1})=(d_{n-i}:X_n\to X_{n-1})\\
    (s_i:X^{\op}_n\to X^\op_{n+1})=(s_{n-i}:X_n\to X_{n+1})
\end{array}\]
\end{definition}
If $X$ is a quasi-category, so is $X^\op$. 
\begin{definition}
  Let $X$ be a quasi-category. The \emph{quasi-category of simplicial presheaves} is defined as $\Pshv(X):=\Hom_{\sset}(X^{\op},\s)$.
\end{definition}
\begin{theorem}[Yoneda]
  Given a quasi-category $X$, there is a fully faithful functor $X\to\Pshv(X)$. The functor $X\to\Pshv(X)$ presents the category of pre-sheaves as the free cocompletion of $X$. 
\end{theorem}
%--------------TO BE CONTINUED-------------
\begin{theorem}
  Let $K$ be a quasi-category. 
  \begin{enumerate}
    \item There exists a set of $\kappa$-compact objects $\C^0$ in $K$, such that every object in $K$ is a $\kappa$-filtered colimit of objects in $\C^0$.
    \item There exists a small category $\C^0$ and a fully faithful accessible right adjoint $\C \to \Fun((\C^0)^{op},\set)$.
  \end{enumerate}
\end{theorem}

In those cases we say $K$ is presentable. We now again have the adjoint functor theorem. 

\begin{theorem}
  Let $\C,\D$ be presentable $\infty$-categories.
  \begin{enumerate}
    \item $F\colon \C \to \D$ is a left adjoint if and only if it preserves colimits.
    \item $F\colon \C \to \D$ is a right adjoint if and only if it preserves limits and is accessible.
  \end{enumerate}
\end{theorem}

Note in particular the $\infty$-category of sheaves is a presentable $\infty$-category.

\section{Stable \texorpdfstring{$\infty$}{oo}-Categories and Spectra}
We now use the $\infty$-categorical framework to study spectra. Let us recall some facts about spectra, to motivate the story. The \emph{Freudenthal suspension theorem} states that the suspension functor $\Sigma \colon \Top \to \Top$ stabilizes the homotopy type. More explicitly, the map
\[
 \pi_k(X) \to \pi_{k+1}( \Sigma X) \to \pi_{k+2}(\Sigma^2X) \to ... 
\]
stabilizes for $k$ large enough, if $X$ satisfies some connectivity condition. This defined the stable homotopy groups $\pi_n^S(X)$ as the stabilization of this sequence.

There is significant interest in computing these stable homotopy groups, in particular in the case where $X$ is a sphere, given that it helps us understand many phenomena in algebraic topology.

We now want a setting where these stable homotopy groups naturally live and can be studied. We know that $(\Sigma,\Omega)$ induces an adjunction on the category of pointed topological spaces. What we want is an adjustment of this definition such that the adjunction $(\Sigma,\Omega)$  is an equivalence. 

We now take a $\infty$-categorical perspective on this and use it to study such stable phenomena. 

\begin{definition}
  Let $\C$ be an $\infty$-category with initial and terminal object. $\C$ has a $0$-object if they are equivalent.
\end{definition}

\begin{example}
  Let $\C$ be a $1$-category. Then $\C$ is pointed as a $1$-category if and only if it is pointed as an $\infty$-category. 
\end{example}

\begin{example}
  Notice $\s$ is not pointed, we hence can define $\s_*$ as the slices under the terminal space, i.e. $\s_* = \s_{*/}$. This $\infty$-category is then pointed by construction.
\end{example}

Note $\s_*$ is not just some pointed $\infty$-category, it is in some sense the universal one.
\begin{proposition}
  Let $\D$ be a pointed $\infty$-category. Then the functor 
  \[\mathrm{ev}_{S^0}\colon\Fun^L(\s_*,\D) \xrightarrow{ \ \simeq \ } \D \]
  that evaluates at $S^0$ is an equivalence.
\end{proposition}

We now generalize from there and define triangles in $\s_*$. 

\begin{definition}
  Let $\C$ be a pointed $\infty$-category. A \emph{triangle} in $\C$ is a diagram of the form
  \[
  \begin{tikzcd}
   X \arrow[r]  \arrow[d] & Y \arrow[d] \\ 
   0 \arrow[r] & Z
  \end{tikzcd}
  \]
  where $X$, $X$, and $Z$ are objects in $\C$.
\end{definition}

\begin{definition}
  We say a triangle is a \emph{exact} if it is a pullback square and \emph{coexact} if it is a pushout square.
\end{definition}

\begin{definition}
  Let $\C$ be a pointed $\infty$-category. Let $\C^{\Sigma}$ be the full subcategory of $\Fun([1] \times [1],\C)$ with objects coexact triangles of the form 
  \[
  \begin{tikzcd}
   X \arrow[r]  \arrow[d] & 0 \arrow[d] \\ 
   0 \arrow[r] & Y
  \end{tikzcd},
  \]
  meaning $Y$ is the suspension of $X$.
\end{definition}

\begin{definition}
  Let $\C$ be a pointed $\infty$-category. Let $\C^{\Omega}$ be the full subcategory of $\Fun([1] \times [1],\C)$ with objects exact triangles of the form 
  \[
  \begin{tikzcd}
   Y \arrow[r]  \arrow[d] & 0 \arrow[d] \\ 
   0 \arrow[r] & X
  \end{tikzcd},
  \]
  meaning $Y$ is the loop object of $X$.
\end{definition}

\begin{proposition}
  If $\C$ is a pointed $\infty$-category with finite (co)limits. Then there exists functors 
  \[ \Sigma \colon \C \to \C^{\Sigma} \to \C\]
  \[ \Omega \colon \C \to \C^{\Omega} \to \C\]
\end{proposition}

\begin{theorem}
  Let $\C$ be a pointed $\infty$-category with finite (co)limits. The following are equivalent:
  \begin{enumerate}
    \item A triangle is exact if and only if it is coexact. 
    \item The functors $\Sigma$ and $\Omega$ are equivalences and the inverses of each other.
    \item A square is a pullback square if and only if it is a pushout square.
  \end{enumerate}
   
\end{theorem}

\begin{definition}
  A pointed $\infty$-category $\C$ is \emph{stable} if it satisfies one of the three equivalent conditions above.
\end{definition}

Recall that before the rise of $\infty$-categories, \emph{triangulated categories} were used to study stable homotopy theory. Hence, it is unsurprising that we can relate stable $\infty$-categories to triangulated categories.

\begin{proposition}
  If $\C$ is a stable $\infty$-category, then the homotopy category $h\C$ is a triangulated category.
\end{proposition}

Of course arbitrary pointed $\infty$-categories will not be stable. We hence want a procedure that stabilizes them. There are several approaches. One approach, that is powerful from a theoretical perspective, is given via reduced $1$-excisive functors out of finite pointed spaces. Here, we focus on explicit spectrum objects, as there are characterized more explicitly. For a comparison of these two approaches see \cite{lurie2017ha}.

\begin{definition}
  Let $\C$ be a pointed $\infty$-category. A \emph{pre-spectrum object} in $\C$, is a functor $X \colon \bZ \times \bZ \to \C$ such that $X(i,j) = 0$ for $i \neq j$ and all squares are pushout squares. Let $PSp(\C)$ be the $\infty$-category of pre-spectrum objects in $\C$. 
\end{definition}

For a given pre-spectrum object $X$, let $\alpha_{m-1}\colon \Sigma X_{m-1} \to X_m$ and $\beta_m \colon X_m \to \Omega X_{m+1} = \Omega \Sigma X_m$. 

\begin{definition}
  Let $\C$ be a pointed $\infty$-category. A \emph{spectrum object} in $\C$ is a pre-spectrum object in $X$, such that $\alpha_{m-1}$ and $\beta_m$ are equivalences for all $m$. Let $\Sp(\C)$ be the $\infty$-category of spectrum objects in $\C$.
\end{definition}

\begin{definition}
  Let $\C$ be a pointed $\infty$-category. The stabilization of $\C$ is the $\infty$-category $\Sp(\C)$ of spectrum objects in $\C$. 
\end{definition}
Of course $\C$ and $\Sp(\C)$ are suitably related.

\begin{theorem}
  For a given pointed $\infty$-category $\C$, there is an adjunction
  \[
  \begin{tikzcd}
    \C \arrow[r, shift left=1.8, "\Sigma"] \arrow[r, shift right=1.8, "\Omega"', "\bot", leftarrow] &  \Sp(\C) 
  \end{tikzcd}
  \]
\end{theorem}

Moreover, $\Sp(\C)$ is in some sense the universal stabilization of $\C$.

\begin{theorem}
 Let $\C$ be a pointed $\infty$-category and $\D$ a stable $\infty$-category. Then $\Sigma^\infty$ induces an equivalence of $\infty$-categories
 \[(\Sigma^\infty)^* \colon\Fun^L(\Sp(\C),\D) \simeq \Fun^L(\C,\D)\]
\end{theorem}

Let us now focus on the case $\C = \s_*$.

\begin{example}
  The stabilization of $\s_*$ is the $\infty$-category of spectra, denoted $\Sp$.
\end{example}

Similar to $\s_*$, $\Sp$ is also the universal stable $\infty$-category, as a special instance of the result above.

\begin{theorem}
 If $\D$ is a stable $\infty$-category. Then the functor 
  \[\mathrm{ev}_{\bS}\colon\Fun^L(\Sp,\D) \xrightarrow{ \ \simeq \ } \D \]
  that evaluates at $\bS$ is an equivalence. 
\end{theorem}


\section{Generalized Cohomology Theories}
Cohomology theories were traditionally defined in the context of topological spaces. However, now that we have the tools of $\infty$-categories and stable $\infty$-categories. We can significantly generalize those definitions. This last result follows work in \cite{lurie2017ha}.
\begin{definition}
  Let $\C$ be a pointed $\infty$-category with pushouts, and $\Sigma\colon\C \to \C$ and suspension functor. A \emph{generalized cohomology theory} is a functor $H \colon h\C^{op}\to \Ab_\bZ$, such that the following conditions hold:
  \begin{itemize}
    \item There is a natural isomorphism $H^\bullet \to H^{\bullet +1} \Sigma$ 
    \item Coexact sequences maps to exact sequences. 
    \item Arbitrary coproducts map to products.
  \end{itemize}
\end{definition}

We now have the following major result that significantly generalizes the classical Brown representability theorem.
\begin{theorem}
 Let $\C$ be a nice $\infty$-category and $(H^\bullet,\delta)$ be a generalized cohomology theory. Then there exists a spectrum object $E$ in $\C$, such that $H^\bullet(X) \cong \Hom_{h\C}(X,E^\bullet)$, where $\delta = (\beta_\bullet)_*$.
\end{theorem}

\begin{example}
  Unsurprisingly, $\s_*$ satisfies the niceness conditions, and so we can conclude that every generalized cohomology on $\s_*$ is given by a spectrum, recovering the original Brown representability theorem.
\end{example}


{\footnotesize
\bibliographystyle{alpha}
\bibliography{main}
}
\end{document}