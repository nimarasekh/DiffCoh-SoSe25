\documentclass[10pt]{amsart}
\usepackage{amsmath,amsthm,amssymb,amsfonts}
\usepackage[mathscr]{euscript}
\usepackage{tikz}
\usepackage{tikz-cd}
\usepackage{enumerate}
\usepackage{enumitem}
\usepackage{mathtools}
\usepackage[colorlinks=true, linkcolor=red, citecolor = blue]{hyperref}
\usepackage[margin=2.5cm]{geometry}
\setlength{\marginparwidth}{2cm}

\usepackage[nameinlink,capitalise,noabbrev]{cleveref}

\usepackage[textwidth=2cm, textsize=small, colorinlistoftodos]{todonotes}

\newcommand{\A}{\mathscr{A}}
\newcommand{\cA}{\mathcal{A}}
\newcommand{\B}{\mathscr{B}}
\newcommand{\cB}{\mathcal{B}}
\newcommand{\sC}{\mathscr{C}}
\newcommand{\sD}{\mathscr{D}}
\newcommand{\sE}{\mathscr{E}}
\newcommand{\cF}{\mathcal{F}}
\newcommand{\sG}{\mathscr{G}}
\newcommand{\cG}{\mathcal{G}}
\newcommand{\cH}{\mathcal{H}}
\newcommand{\cL}{\mathcal{L}}
\newcommand{\sS}{\mathscr{S}}
\newcommand{\bS}{\mathbb{S}}
\newcommand{\bR}{\mathbb{R}}
\newcommand{\bZ}{\mathbb{Z}}
\newcommand{\bC}{\mathbb{C}}
\newcommand{\fO}{\mathbf{O}}
\newcommand{\rO}{\mathscr{O}}
\newcommand{\M}{\mathcal{M}}
\newcommand{\bQ}{\mathbb{Q}}

\newcommand{\aff}{\mathrm{Aff}}
\newcommand{\Cyc}{\mathrm{Cyc}}
\newcommand{\tmf}{\mathrm{tmf}}
\newcommand{\TMF}{\mathrm{TMF}}
\newcommand{\Tmf}{\mathrm{Tmf}}
\newcommand{\Def}{\mathrm{Def}}
\newcommand{\curv}{\mathrm{curv}}
\newcommand{\CS}{\mathrm{CS}}
\newcommand{\ch}{\mathrm{ch}}
\newcommand{\dKU}{\smash{\widehat{ku}}}
\newcommand{\dKUnabla}{\smash{\widehat{ku}^\nabla}}
\renewcommand{\sp}{\mathrm{sp}}

\DeclareMathOperator{\tr}{tr}
\newcommand{\Map}{\mathrm{Map}}
\newcommand{\uMap}{\underline{\mathrm{Map}}}
\newcommand{\Hom}{\mathrm{Hom}}
\newcommand{\Ho}{\mathrm{ho}}
\newcommand{\set}{\mathscr{S}\mathrm{et}}
\newcommand{\CMon}{\mathrm{CMon}}
\newcommand{\CGrp}{{\normalfont\texttt{CMon}}}
\newcommand{\fib}{{\normalfont\texttt{fib}}}
\newcommand{\cofib}{{\normalfont\texttt{cofib}}}
\newcommand{\Bun}{{\normalfont\texttt{Bun}}}
\newcommand{\Sp}{\mathscr{S}\mathrm{p}}
\newcommand{\Ch}{\mathscr{C}\mathrm{h}}
\newcommand{\cat}{\mathscr{C}\mathrm{at}}
\newcommand{\scat}{s\mathscr{C}\mathrm{at}}
\newcommand{\sset}{s\mathscr{S}\mathrm{et}}
\newcommand{\Line}{\mathscr{L}\mathrm{ine}}
\newcommand{\Fun}{\mathrm{Fun}}
\newcommand{\Nat}{\mathrm{Nat}}
\newcommand{\colim}{\mathrm{colim}}
\newcommand{\Top}{\mathscr{T}\mathrm{op}}
\newcommand{\Grpd}{\mathscr{G}\mathrm{rpd}}
\newcommand{\Grp}{\mathscr{G}\mathrm{rp}}
\newcommand{\Euc}{\mathscr{E}\mathrm{uc}}
\newcommand{\Mfd}{\mathscr{M}\mathrm{fd}}
\newcommand{\Kan}{\mathscr{K}\mathrm{an}}
\newcommand{\Vect}{\mathscr{V}\mathrm{ect}}
\newcommand{\Mod}{\mathscr{M}\mathrm{od}}
\newcommand{\Proj}{\mathscr{P}\mathrm{roj}}
\newcommand{\Ab}{\mathscr{A}\mathrm{b}}
\newcommand{\Shv}{\mathscr{S}\mathrm{hv}}
\newcommand{\Yon}{\mathscr{Y}\mathrm{on}}
\newcommand{\Open}{\mathscr{O}\mathrm{pen}}
\newcommand{\PSh}{\mathscr{P}\mathscr{S}\mathrm{h}}
\newcommand{\Pic}{\mathscr{P}\mathrm{ic}}
\newcommand{\triv}{\mathscr{T}\mathrm{riv}}
\newcommand{\aut}{\mathrm{Aut}}
\newcommand{\Th}{\mathrm{Th}}
\newcommand{\fr}{\mathrm{Fr}}
\newcommand{\Arr}{\mathrm{Arr}}
\newcommand{\ev}{\mathrm{ev}}
\newcommand{\dis}{\mathrm{dis}}
\newcommand{\Mon}{\mathrm{Mon}}
\newcommand{\Rep}{\mathrm{Rep}}
\newcommand{\PT}{\mathrm{PT}}
\newcommand{\Fred}{\mathcal{F}\mathrm{red}}
\newcommand{\inv}{\mathrm{inv}}
\newcommand{\const}{\mathrm{const}}
\newcommand{\spin}{\mathrm{spin}}
\newcommand{\scal}{\mathrm{scal}}
\newcommand{\HS}{\mathcal{H}\mathcal{S}}
\newcommand{\PU}{\mathcal{P}\mathcal{U}}

\newcommand{\bbefamily}{\fontencoding{U}\fontfamily{bbold}\selectfont}
\newcommand{\textbbe}[1]{{\bbefamily #1}}
\DeclareMathAlphabet{\mathbbe}{U}{bbold}{m}{n}

\def\DDelta{{\mathbbe{\Delta}}}
\newcommand{\DD}{\DDelta}


% New commands Matthias Frerichs
\newcommand{\Disc}{\mathrm{Disc}}
\newcommand{\op}{\mathrm{op}}
\newcommand{\D}{\mathsf{d}}


%% N.R. notes
\newcommand{\nrnote}[1]{\todo[color=green!40,linecolor=green!40!black,size=\tiny]{#1}}
\newcommand{\nrmpar}[1]{\todo[noline,color=green!40,linecolor=green!40!black,
  size=\tiny]{#1}}
\newcommand{\nrnoteil}[1]{\ \todo[inline,color=green!40,linecolor=green!40!black,size=\normalsize]{#1}}

\newtheorem{theorem}[equation]{Theorem}
\newtheorem{lemma}[equation]{Lemma}
\newtheorem{proposition}[equation]{Proposition}
\newtheorem{corollary}[equation]{Corollary}
% \newtheorem{statement}[section]{Statement}

\theoremstyle{definition}
\newtheorem{definition}[equation]{Definition}
\newtheorem{example}[equation]{Example}
% \newtheorem{attone}[equation]{Attention}

\theoremstyle{remark}
\newtheorem{remark}[equation]{Remark}
% \newtheorem{intone}[equation]{Intuition}
\newtheorem{notation}[equation]{Notation}
% \newtheorem{queone}[equation]{Question}
% \newtheorem{conjone}[equation]{Conjecture}
\newtheorem{warning}[equation]{Warning}

\numberwithin{equation}{section}

\title{Differential Cohomology Seminar 11}
\date{?}
\author{Talk by Matthias Frerichs}

\begin{document}
\maketitle

\section{Differential Cohomology in a Cohesive $\infty$-topos}

\subsection{Sheaves on manifolds are cohesive}

We consider the essentially small category of manifolds with corners,
denoted $\Mfd$. This category is endowed with the Grothendieck
topology generated by surjective submersions with discrete fibers.
This is also a dense subsite of the the site of all smooth manifolds
with jointly surjective open covers.

\begin{definition}
  Let $\mathcal{C}$ be a $(\infty,1)$-category. The category of
  presheaves on the site of manifolds with values in $\mathcal{C}$ is
  denoted by $\Fun(\Mfd^\mathrm{op}, \mathcal{C})$.
\end{definition}

We will only consider presentable $(\infty, 1)$-categories here, even
if \cite{bunkenikolausvoelkl2016diffcoh} treat the more general case too.

\begin{definition}
  (\cite[Def. 2.3.]{bunkenikolausvoelkl2016diffcoh}, Sheaf on Manifolds)
  A functor $F\in \Fun(\Mfd^\mathrm{op}, \mathcal{C})$ is a sheaf if
  for any manifold $M$ and covering $U\rightarrow M$ the canoncial map
  \begin{align*}
    F(M) \rightarrow \lim_{\Delta}F(U^\bullet)
  \end{align*}
  is an equivalence. Here $U^\bullet$ is the {\v C}ech nerve of the
  covering and $F(U^\bullet)$ is the simplical object in
  $\Fun(\Delta^\mathrm{op}, \mathcal{C})$ by applying $F$ to the {\v
    C}ech nerve.

  Denote the full subcategory of sheaves by $\Shv_{\Mfd}(\mathcal{C})
  \subset \Fun(\Delta^\mathrm{op}, \mathcal{C})$ which is a reflecive
  localization given by
  \begin{align*}
    L \colon \Fun(\Delta^\mathrm{op}, \mathcal{C})
    \rightleftarrows \Shv_{\Mfd}(\mathcal{C}) \colon \mathrm{inclusion}.
  \end{align*}
\end{definition}

\begin{remark}
  (\cite[Def. 2.3.]{bunkenikolausvoelkl2016diffcoh},
  Homotopy-invariance)
  A presheaf $F \in \Fun(\Delta^\mathrm{op}, \mathcal{C})$ is called
  homotopy invariant, if for all manifolds $M\in \Mfd$ the canonical map
  $F(M)\rightarrow F(\Delta^1\times M)$ induced by the projection is an
  equivalence.

  There is an adjunction
  \begin{align*}
    \mathcal{H}^{\mathrm{pre}} \colon \Fun(\Delta^\mathrm{op},
    \mathcal{C}) \rightleftarrows \Fun^h(\Delta^\mathrm{op},
    \mathcal{C}) \colon \mathrm{inclusion}
  \end{align*}
  where $\mathcal{H}^{\mathrm{pre}}$ is called homotopification.

  This functor commutes with sheaffication and restricts to sheaves
  yielding
  \begin{align*}
    \mathcal{H} \colon \Shv_{\Mfd}(\mathcal{C})
    \rightleftarrows \Shv_{\Mfd}^h(\mathcal{C}) \colon \mathrm{inclusion}.
  \end{align*}
\end{remark}

By \cite[Prop. 2.6.]{bunkenikolausvoelkl2016diffcoh} this functor
preserves finite products for the special case that $\mathcal{C}$ is
stable or the category of spaces. In our case we will only consider
the category of spaces.

The category $\Shv_{\Mfd}^h(\mathcal{C}) $ is equivalent to
$\mathcal{C}$ by sheaffication of the constant presheaf and evaluation
at the point
\begin{align*}
   \mathrm{const} \colon \mathcal{C}
    \rightleftarrows \Shv_{\Mfd}^h(\mathcal{C}) \colon \ev_{\ast}.
\end{align*}

In conclusion we have the following quadruple of adjoint functors.

\begin{remark}
  (\cite[Rmk. 2.7.]{bunkenikolausvoelkl2016diffcoh},
  Coherent Topos)
  There is an adjoint quadruple of functors
  \begin{align*}
    (\ev_\ast \circ \mathcal{H} \dashv \mathrm{const} \dashv \ev_*
    \dashv \mathcal{G}) \colon \Shv_{\Mfd}(\mathcal{C})
    \mathrel{\substack{
    \textstyle\overset{\ev_\ast \circ \mathcal{H}}\longrightarrow\\[-0.2ex]
    \textstyle\overset{\mathrm{const}}\longleftarrow \\[-0.2ex]
    \textstyle\overset{\ev_\ast}\longrightarrow \\[-0.2ex]
    \textstyle\overset{G}\longleftarrow}
    }
    \mathcal{C}
  \end{align*}
  where $ G \colon  \mathcal{C}
  \rightarrow \Shv_{\Mfd}(\mathcal{C}) $ is defined by $G(F)(M) \coloneqq
  \const(F)(\ast)^{M_\mathrm{disc}} = F^{M_\mathrm{disc} }$. Here $\ev_\ast \circ \mathcal{H}$
  preserves products
  (\cite[Prop. 2.6.]{bunkenikolausvoelkl2016diffcoh}) and $\const$ and
  $G$ are fully faithful.
\end{remark}

We will only consider cohesive topoi over the $\infty$-category of
spaces.

\begin{definition}
  (\cite[Rmk. 3.4.2.]{schreiber2013cohesive})
  A $\infty$-topos $\mathcal{X}$ is called cohesive if it admits a
  quadruple adjoint
   \begin{align*}
    (\Pi \dashv \mathrm{Disc} \dashv \Gamma
    \dashv \mathrm{coDisc}) \colon \mathcal{X}
    \mathrel{\substack{
     \textstyle\overset{\Pi}\longrightarrow\\[-0.2ex]
    \textstyle\overset{\mathrm{Disc}}\longleftarrow \\[-0.2ex]
    \textstyle\overset{\Gamma}\longrightarrow \\[-0.2ex]
    \textstyle\overset{\mathrm{coDisc}}\longleftarrow}
    }
    \mathcal{C}
   \end{align*}
   such that $\Pi$ preserves finite products and $\mathrm{Disc}$ and
   $\mathrm{coDisc}$ are fully faithfull.
\end{definition}

Now we can state the definition of the intrinsic non-abelian
cohomology in a cohesive $\infty$-topos as in
\cite[Def. 3.6.134.]{schreiber2013cohesive}.

\begin{definition}
  (\cite[Def. 3.6.134.]{schreiber2013cohesive}, Cohomolgy in an
  $\infty$-topos)
  For two objects $X,A \in \mathcal{X}$ we define the cohomology set
  of $X$ with coefficients in $A$ by
  \begin{align*}
    H(X,A) \coloneqq \pi_0\mathcal{X}(X,A)
  \end{align*}
  where $\mathcal{X}(X,A)$ is the $\infty$-groupoid of morphisms
  between $X$ and $A$. Generally, if $A$ has a $p$-fold delooping for
  $p\in \mathbb{N}$ we define
  \begin{align*}
    H^p(X,A) \coloneqq \pi_0\mathcal{X}(X,\mathbf{B}^pA)
  \end{align*}
\end{definition}

The goal is now to write down a differential refinement of the
cohomolgy in a cohesive $\infty$-topos. For this we need to define a
version of De Rham cohomolgy that will later correspond to a truncated
complex of differential form as used in the classical case.

\subsection{Cohesive De-Rham complex}


\begin{definition}
  \label{def:cohesive_de_rham}
  (\cite[Def. 3.9.12]{schreiber2013cohesive})
  Let $\mathcal{X}$ be a cohesive $\infty$-topos. For $X\in
  \mathcal{X}$ we define the cohesive de Rham homotopy type
  $\Pi_{\mathrm{dR}}X$ of $X$ as the pushout
  \begin{equation}
    \begin{tikzcd}[nodes in empty cells, column sep= 1.5cm, row sep=1.5cm]
      X
      \ar[r]
      \ar[d,"\eta"]
      &
      \ast
      \ar[d]
      \\
      \mathrm{Disc}\circ \Pi (X)
      \ar[r]
      &
      \Pi_{\mathrm{dR}}X
    \end{tikzcd}
  \end{equation}
  where $\eta$ is the counit of the $\Pi \dashv \mathrm{Disc}$.

  For a pointed object $\ast \rightarrow A$ in $\mathcal{H}$ we define
  the de Rham coefficient object $\flat_{\mathrm{dR}}A$ by the pullback
  \begin{equation}
    \begin{tikzcd}[nodes in empty cells, column sep= 1.5cm, row sep=1.5cm]
      \flat_{\mathrm{dR}}A
      \ar[r]
      \ar[d]
      &
      \mathrm{Disc} \circ \Gamma (A)
      \ar[d, "\epsilon"]
      \\
      \ast
      \ar[r]
      &
      A
    \end{tikzcd}
  \end{equation}
  where $\epsilon$ is the counit of the adjunction
  $\mathrm{Disc}\dashv \Gamma$.
\end{definition}

This also yields an adjunction
\begin{proposition}
  (\cite[Prop. 3.9.13, Def 3.9.15]{schreiber2013cohesive})
  There is an adjunction
  \begin{align*}
    (\Pi_{\mathrm{dR}} \dashv \flat_{\mathrm{dR}}) \colon
    \ast/\mathcal{X}
    \mathrel{\substack{
    \textstyle\xlongleftarrow{\Pi_{\mathrm{dR}}} \\[-0.2ex]
    \textstyle\xlongrightarrow[\flat_{\mathrm{dR}}]{}}
    }
    \mathcal{X}
  \end{align*}
  where $\ast/\mathcal{X}$ is the category of pointed objects in $\mathcal{X}$.
  We write
  \begin{align*}
    H_{\mathrm{dR}}(X,A) \coloneqq H(\Pi_{\mathrm{dR}}X,A) \cong H(X,\flat_{\mathrm{dR}}A)
  \end{align*}
  for the de Rham cohomology of $X$ with coefficients in $A$.
\end{proposition}


Now it is sensible to check wether this yield the well known de Rham
cohomolgy in the case that $\mathcal{X}$ is a sheaf on a smooth
manifold $M$.

\begin{definition}
  Define the site of cartesian spaces $\mathsf{CartSp}$ as the category of cartesian
  spaces $\mathbb{R}^n$ for $n\in \mathbb{N}$ with continuous smooth
  functions between them.

  Consider the category of simplicial presheaves
  $[\mathsf{CartSp}^\op, \mathsf{sSet}]$


  Now we endow the category of simplical sets with the quillen model
  structure and localize at the maps
  \begin{align*}
    \check{C}(\{U_i\rightarrow U\}) \rightarrow U
  \end{align*}
  where $\{U_i \rightarrow U\}$ is a covering family of $U$, $\check{C}$
  denotes the simplicial manifold defined by the \v{C}ech nerve which
  gives a simplical presheaf by Yoneda embedding. Here $U$ is
  considered as a simplicial presheaf by inserting the constant
  presheaf $U$ in each degree, see \cite[Section
  2]{dugger-2000-univers}. We then obtain the category of
  $\infty$-sheaves $\Shv_\Mfd$ valued in spaces, see
  \cite[Prop. 6.5.2.14.]{lurie2009htt} and \cite[Thm. 2.2.15.]{schreiber2013cohesive}.
\end{definition}

\begin{proposition}
  (\cite[Prop. 4.4.22., Def. 4.4.21.]{schreiber2013cohesive})
  The simplical presheaf
  \begin{align*}
    U(1)[n] \coloneqq
    [
    \cdots \rightarrow 0 \rightarrow \mathcal{C}^\infty (-, U(1))
    \rightarrow 0\rightarrow \cdots \rightarrow 0
    ]
  \end{align*}
  concentrated in degree $n$ is a fibrant representative in
  $\Shv_{\Mfd}$ of $\mathbf{B}^nU(1)$ under the Quillen equivalence of
  chain complexes with projective model structure and the Quillen
  model structure on simplicial sets, see see \cite[Prop. 2.2.31.]{schreiber2013cohesive}.
\end{proposition}

\begin{proposition}
  (\cite[Prop. 4.4.49.]{schreiber2013cohesive})
  A fibrant representative in $\Shv_{\Mfd}$ of the de Rham coefficient
  object $\flat_{\mathrm{dR}}\mathbf{B}^nU(1)$ is given by the
  truncated ordinary de Rham complex of smooth differential forms
  \begin{align*}
    \flat_{\mathrm{dR}}\mathbf{B}^nU(1)_{\mathrm{chn}} \coloneqq
    \Psi
    [
    \Omega^1(-) \xrightarrow{\D_{\mathrm{dR}}}
    \Omega^2(-) \xrightarrow{\D_{\mathrm{dR}}}
    \cdots
    \xrightarrow{\D_{\mathrm{dR}}}
    \Omega^{n-1}(-) \xrightarrow{\D_{\mathrm{dR}}}
    \Omega^n_{\mathrm{cl}}(-)
    ]
  \end{align*}
  where $\Psi$ denotes the Quillen equivalence between simplical sets
  with the Quillen model structure and the projective model structure
  on chain complexes, see \cite[Prop. 2.2.31.]{schreiber2013cohesive}.
\end{proposition}

\begin{proposition}
  (\cite[Prop. 4.4.50.]{schreiber2013cohesive})
  There is are isomorphisms for $n\in \mathbb{N}$
  \begin{align*}
    H_{\mathrm{dR}}^n(X,\mathbf{B}^nU(1)) \coloneqq
    \pi_0 \Shv_{\Mfd} (M, \flat_{\mathrm{dR}}\mathbf{B}^nU(1)) \cong
    \begin{cases}
      H_{\mathrm{dR}}^n(M) & n \geq 2 \\
      \Omega_{\mathrm{cl}}^1(M) & n = 1\\
      0 & n = 0
    \end{cases}
  \end{align*}
  where a manifold considered as a constant sheaf.
\end{proposition}

\subsection{Differential cohomology}

Now we still need to define a characteristic map for the differential
refinement we want to construct.

\begin{definition}
  (\cite[Def. 3.9.29.]{schreiber2013cohesive})
  For a group object $G \in \mathcal{X}$ in the cohesive
  $\infty$-topos $\mathcal{X}$ define the Maurer-Cartan Form
  \begin{align*}
    \theta \colon G \rightarrow \flat_{\mathrm{dR}}\mathbf{B}G
  \end{align*}
  via the pullback diagrams:
  \begin{equation}
    \begin{tikzcd}[nodes in empty cells, column sep= 0.8cm, row sep=0.8cm]
      G
      \ar[r]
      \ar[d,"\theta"]
      &
      \ast
      \ar[d]
      \\
      \flat_{\mathrm{dR}}\mathbf{B}G
      \ar[r]
      \ar[d]
      &
      \mathrm{Disc}\circ \Gamma( \mathbf{B}G)
      \ar[d,"\epsilon"]
      \\
      \ast
      \ar[r]
      &
      \mathbf{B}G
    \end{tikzcd}
  \end{equation}
  where, again $\epsilon$ is the counit of the adjunction.
\end{definition}

Now the general definition of differential cohomology with
coefficients in a braided $\infty$-group, that means the double
delooping $\mathbf{B}^2G$ exists, is given as follows:

\begin{definition}
  (\cite[Def. 3.9.32., Rmk. 3.9.33.]{schreiber2013cohesive}, General
  differential cohomology in a cohesive topos)

  The differential cohomology with coefficients in $\mathbf{B}G$ is
  defined as the cohomology of the slice topos
  \begin{align*}
     \mathcal{X}/\flat_{\mathrm{dR} \mathbf{B}^2 G}.
  \end{align*}
  A domain object in this is an object $X\in \mathcal{X}$ with a de
  Rham cocycle $F\colon X\rightarrow \flat_{\mathrm{dR}}\mathbf{B}^2
  G$.
  An element in the corresponding cohomology group
  $\mathcal{X}/\flat_{\mathrm{dR} \mathbf{B}^2 G}((X,F),
  \mathrm{curv}_{G} \coloneqq \theta \colon \mathbf{B}G
  \rightarrow \flat_{\mathrm{dR}}\mathbf{B}^2G)$ is then given by a
  tranformation
  \begin{equation}
    \begin{tikzcd}[nodes in empty cells, column sep= 1cm, row sep=1.5cm]
      X
      \ar[rr, "g"{name=g}]
      \ar[dr, swap,"F"{name=F}]
      \arrow[Rightarrow, shorten = 4pt, from=g, to=F , "\nabla"]
      &
      &
      \mathbf{B}G
      \ar[dl, "\mathrm{curv}_{G}"]
      \\
      &
      \flat_{\mathrm{dR}}\mathbf{B}^2G
      &
    \end{tikzcd}
  \end{equation}
  This is:
  \begin{itemize}
    \item a cocycle $g\colon X \rightarrow \mathbf{B}G$ for a
      $G$-principal bundle over $X$ (in the cohesive sense \cite[Def 3.6.152.]{schreiber2013cohesive})
    \item an equivalence $g^* \mathrm{curv}_{G} \rightarrow F$
      interpreted as a connection on the $G$-principal bundle.
  \end{itemize}
\end{definition}

Restricting to the special case that $G$ is an Eilenberg-MacLane
object, i.e. $\mathbf{B}G \cong \mathbf{B}^n A$ for a $0$-truncated
abelian group object $A$, see \cite[Def. 5.5.6.1.]{lurie2009htt}, we
finally look at a more familiar diagram of a differential refinement.

\begin{definition}
  For $X\in \mathcal{X}$ and $n\geq 1$ write
  \begin{align*}
    \mathcal{X}_{\mathrm{diff}}(X,\mathbf{B}^nA) \coloneqq
    \mathcal{X}(X,\mathbf{B}^nA )
    \prod_{\mathcal{X}(X,\flat_{\mathrm{dR}}\mathbf{B}^nA)} H_{\mathrm{dR}}^{n+1}(X,A)
  \end{align*}
  for the pullback of the diagram
   \begin{equation}
    \begin{tikzcd}[nodes in empty cells, column sep= 1cm, row sep=1.5cm]
      \mathcal{X}_{\mathrm{diff}}(X,\mathbf{B}^nA)
      \ar[d]
      \ar[r]
      &
      H_{\mathrm{dR}}^{n+1}(X,A)
      \ar[d]
      \\
      \mathcal{X}(X,\mathbf{B}^nA )
      \ar[r,"\mathrm{curv}_\ast"]
      &
      \mathcal{X}(X,\flat_{\mathrm{dR}}\mathbf{B}^nA)
    \end{tikzcd}
  \end{equation}
  where $\curv \colon \mathbf{B}^{n} A \rightarrow
  \flat_{\mathrm{dR}}\mathbf{B}^{n+1}A$ is the Maurer-Cartan Form, in
  this case called the curvature characteristic morphism and the right
  vertical map is the unique, up to equivalence morphism from
  $H_\mathrm{dR}^{n+1}(X,A)$ seen as the $0$-truncation of
  $\mathcal{X}(X,\flat_{\mathrm{dR}}\mathbf{B}^nA)$.

  We call
  \begin{align*}
    H_{\mathrm{diff}}^n(X,A) \coloneqq \pi_0 \mathcal{X}_{\mathrm{diff}}(X,\mathbf{B}^nA)
  \end{align*}
  the degree-$n$ differential cohomology of $X$ with coefficients in $A$.
\end{definition}

One (not me) can also see that in the case of $A = U(1) \cong K(\mathbb{Z},1)$
this definition gives the classical definition of Hopkins and Singer,
see \cite{hopkinssinger2005diffcoh}.

\begin{theorem}
  (\cite[Thm. 4.4.87.]{schreiber2013cohesive})
  For a smooth manifold $M$ seen as a sheaf on $\mathsf{CartSp}$ we
  have that $ H_{\mathrm{diff}}^n(M,U(1))$ is given by the subset of
  Deligne cocycles that picks for each de Rham cohomology class of $M$
  a curvature form representative.
\end{theorem}

% \section{Snippets}

% \begin{definition}
%   (\cite[Def 6.1.0.2]{lurie2009htt}, $\infty$-Topos) Let $\mathcal{X}$ be an
%   $\infty$-category. We will say that $\mathcal{X}$ is an
%   $\infty$-topos if there exists a small $\infty$-category
%   $\mathcal{C}$ and an accessible left exact localization functor
%   $\PSh(\mathcal{C})\to\mathcal{X}$.

%   Where $\PSh(\mathcal{C})\coloneqq \Fun(\mathcal{C}^\mathrm{op},
%   \Sp)$ is the presheaf category and $\Sp$ is the $\infty$-category of
%   spaces.

% \end{definition}


% There is also a notion of cohomolgy in $\infty$-topoi by Lurie,
% further called ordinary cohomology:

% \begin{definition}
%   (\cite[Def. 7.2.2.14.]{lurie2009htt}, Ordinary cohomology)
%   Let $\mathcal{X}$ be an $\infty$-topos, $n\geq 0$ an integer, and $A$ an abelian group object of the topos $\Disc(\mathcal{X})$. We define
%   \begin{align*}
%     H^{n}(\mathcal{X},A)= \pi_{0}\mathcal{X}(1,K(A,n));
%   \end{align*}
%   we refer to $H^{n}(\mathcal{X},A)$ as the $n$th cohomology group of $\mathcal{X}$ with coefficients in $A$.
% \end{definition}

% The category of discrete objects of an $(\infty,1)$-category is
% defined by Lurie as:

% \begin{definition}
%   (\cite[Def. 5.5.6.1.]{lurie2009htt})
%   \begin{itemize}
%   \item A space/Kan-complex $X$ is called $n$-truncated if for every point
%     $x \in X$ the homotopy groups $\pi_i(X,x)$ vanish for all $i >n$.
%   \item A space/Kan-complex $X$ is called $n$-connective if for every point
%     $x \in X$ the homotopy groups $\pi_i(X,x)$ vanish for all $i < n$.
%   \item An object $C$ of an $(\infty,1)$-category $\mathcal{C}$ is
%     $k$-truncated for $k \geq -1$ if the space $\mathcal{C}(D,C)$
%     is $k$-truncated for every $D\in \mathcal{C}$.
%     Denote the full subcategory of $k$-truncated objects by
%     $\tau_{\leq k}\mathcal{C}$.
%   \item For an object $C\in \mathcal{C}$ in an $(\infty,1)$-category
%     $C$ is called discrete if it is $0$-truncated. Denote the
%     full subcategory of discrete objects by $\Disc(\mathcal{C})$.
%   \end{itemize}
% \end{definition}

% One can see that the category of discrete objects,
% i.e. $\Disc(\mathcal{C})$ or $\tau_{\leq 0}\mathcal{C}$ is
% equivalent to the nerve of its homotopy category
% $N(\pi_0(\mathcal{C}))$. This follows from
% \cite[Prop. 2.3.4.18]{lurie2009htt}, which states that the full
% subcategories of $k$-truncated objects are equivalent to
% $k+1$-categories and \cite[Prop. 2.3.4.5.]{lurie2009htt} which states
% that a simplical set which is a $1$-category, is also equivalent
% to the nerve of its homotopy category.

% \begin{definition}
%   (\cite[Def. 7.2.2.1.]{lurie2009htt}, Pointed objects)
%   A pointed object $X\in \mathcal{X}$ is a morphism
%   \begin{align*}
%     X_\ast \colon 1 \rightarrow X
%   \end{align*}
%   where $1$ is a final object of $\mathcal{X}$.
%   The full subcategory of $\Fun(\Delta^1,\mathcal{X})$ spanned by the
%   pointed objects is denoted $\mathcal{X}_\ast$
% \end{definition}




	{\footnotesize
 \bibliographystyle{alpha}
 \bibliography{main}
 }
\end{document}

%%% Local Variables:
%%% mode: LaTeX
%%% TeX-master: t
%%% End:
