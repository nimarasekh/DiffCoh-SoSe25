\documentclass[10pt]{amsart}
\usepackage{amsmath,amsthm,amssymb,amsfonts}
\usepackage[mathscr]{euscript}
\usepackage{tikz}
\usepackage{tikz-cd}
\usepackage{enumerate}
\usepackage{enumitem}
\usepackage{mathtools}
\usepackage[colorlinks=true, linkcolor=red, citecolor = blue]{hyperref}
\usepackage[margin=2.5cm]{geometry}
\setlength{\marginparwidth}{2cm}

\usepackage[nameinlink,capitalise,noabbrev]{cleveref}

\usepackage[textwidth=2cm, textsize=small, colorinlistoftodos]{todonotes}

\newcommand{\lA}{\mathcal{A}}
\newcommand{\bA}{\mathbb{A}}
\newcommand{\lB}{\mathcal{B}}
\newcommand{\bB}{\mathbb{B}}
\newcommand{\lC}{\mathcal{C}}
\newcommand{\rC}{\mathscr{C}}
\newcommand{\bC}{\mathbb{C}}
\newcommand{\lD}{\mathcal{D}}
\newcommand{\bE}{\mathbb{E}}
\newcommand{\lE}{\mathcal{E}}
\newcommand{\lF}{\mathcal{F}}
\newcommand{\lG}{\mathcal{G}}
\newcommand{\lH}{\mathcal{H}}
\newcommand{\mH}{\mathrm{H}}
\newcommand{\lI}{\mathcal{I}}
\newcommand{\lJ}{\mathcal{J}}
\newcommand{\lK}{\mathcal{K}}
\newcommand{\lL}{\mathcal{L}}
\newcommand{\lM}{\mathcal{M}}
\newcommand{\bN}{\mathbb{N}}
\newcommand{\mN}{\mathrm{N}}
\newcommand{\lN}{\mathcal{N}}
\newcommand{\fO}{\mathbf{O}}
\newcommand{\rO}{\mathscr{O}}
\newcommand{\lO}{\mathcal{O}}
\newcommand{\lP}{\mathcal{P}}
\newcommand{\bQ}{\mathbb{Q}}
\newcommand{\bR}{\mathbb{R}}
\newcommand{\lR}{\mathcal{R}}
\newcommand{\lS}{\mathcal{S}}
\newcommand{\bS}{\mathbb{S}}
\newcommand{\lT}{\mathcal{T}}
\newcommand{\lU}{\mathcal{U}}
\newcommand{\lV}{\mathcal{V}}
\newcommand{\lX}{\mathcal{X}}
\newcommand{\lY}{\mathcal{Y}}
\newcommand{\bZ}{\mathbb{Z}}

\newcommand{\op}{\mathrm{op}} % opposite
\newcommand{\ch}{\mathrm{ch}}
\newcommand{\Hom}{\lH\mathrm{om}} % enriched hom-spaces, hom-spectrum
\renewcommand{\hom}{\mH\mathrm{om}} % hom-set, hom-space
\newcommand{\Ho}{\mathrm{Ho}} % homotopy category
\newcommand{\set}{\lS\mathrm{et}} % category of sets
\newcommand{\Mon}{\lM\mathrm{on}} % category of monoid objects
\newcommand{\CMon}{\lC\Mon} % category of commutative monoid objects
\newcommand{\CGrp}{\lC\lG\mathrm{on}} % category of commutative group objects
\newcommand{\Bdl}{\lB\mathrm{dl}} % bundles
\newcommand{\Sp}{\lS\mathrm{p}} % category of spectra
\newcommand{\Ch}{\lC\mathrm{h}} % category of chain complexes
\newcommand{\cat}{\lC\mathrm{at}} % category of categories
\newcommand{\scat}{\mathrm{s}\cat} % category of simplicial categories
\newcommand{\sset}{\mathrm{s}\set} % category of simplicial sets
\newcommand{\Line}{\lL\mathrm{ine}} % category of lines
\newcommand{\LineBdl}{\Line\Bdl} % category of line bundles
\newcommand{\Fun}{\lF\mathrm{un}} % category of functors
\newcommand{\Top}{\lT\op} % category of topological spaces 
\newcommand{\Grpd}{\lG\mathrm{rpd}} % category of groupoids
\newcommand{\Grp}{\lG\mathrm{rp}} % category of groups
\newcommand{\Euc}{\lE\mathrm{uc}} % site of Euclidean manifolds
\newcommand{\Mfd}{\lM\mathrm{fd}} % site of smooth manifolds
\newcommand{\Kan}{\lK\mathrm{an}} % category of Kan complexes
\newcommand{\Vect}{\lV\mathrm{ect}} % category of vector spaces
\newcommand{\Mod}{\lM\mathrm{od}} % category of modules
\newcommand{\Ab}{\lA\mathrm{b}} % category of abelian groups
\newcommand{\Shv}{\lS\mathrm{hv}} % category of sheaves
\newcommand{\Open}{\lO\mathrm{pen}} % category of open subspaces
\newcommand{\Pic}{\lP\mathrm{ic}} % Picard category
\newcommand{\PT}{\lP\lT} % Pontryagin-Thom
\newcommand{\Fred}{\lF\mathrm{red}} % Fredholm
\newcommand{\fl}{\mathrm{fl}} % flat
\newcommand{\GrbBdl}{\lG\mathrm{rb}\Bdl} % category of bundle gerbes
\newcommand{\cl}{\mathrm{cl}} % closed forms
\newcommand{\BDelta}{\mathbf{\Delta}} % simplex category
\newcommand{\Sing}{\mathrm{Sing}} % singular simplicial set


\DeclareMathOperator*\colim{colim} % colimit



%% N.R. notes
\newcommand{\nrnote}[1]{\todo[color=green!40,linecolor=green!40!black,size=\tiny]{#1}}
\newcommand{\nrmpar}[1]{\todo[noline,color=green!40,linecolor=green!40!black,
  size=\tiny]{#1}}
\newcommand{\nrnoteil}[1]{\ \todo[inline,color=green!40,linecolor=green!40!black,size=\normalsize]{#1}}

\newtheorem{theorem}[equation]{Theorem}
\newtheorem{lemma}[equation]{Lemma}
\newtheorem{proposition}[equation]{Proposition}
\newtheorem{corollary}[equation]{Corollary}
% \newtheorem{statement}[section]{Statement}

\theoremstyle{definition}
\newtheorem{definition}[equation]{Definition}
\newtheorem{example}[equation]{Example}
% \newtheorem{attone}[equation]{Attention}

\theoremstyle{remark}
\newtheorem{remark}[equation]{Remark}
% \newtheorem{intone}[equation]{Intuition}
\newtheorem{notation}[equation]{Notation}
% \newtheorem{queone}[equation]{Question}
% \newtheorem{conjone}[equation]{Conjecture}
\newtheorem{warning}[equation]{Warning}

\numberwithin{equation}{section}

\title{Differential Cohomology Seminar 8}
\date{12.11.2025 \& 26.11.2025}
\author{Talk by Hannes Berkenhagen}

\begin{document}
\maketitle

The aim of this talk is to review twisted cohomology theory with the aim of later discussing twisted differential cohomology theories \cite{bunkegepner2021differential}. For these talks the main source is \cite{abghr2014infty,abg2018thom}.

\section{Twisted Cohomology}
Let $R$ be a ring spectrum, meaning a monoid object in the $\infty$-category of spectra $\Sp$. From this we get a presentable stable $\infty$-category $\Mod_R$ of left $R$-module spectra. Objects therein are morphisms of the form $R \wedge M \to M$ satisfying the usual associativity and unit conditions up to coherent homotopies.
\begin{remark}
	Every stable category $\lC$ is enriched over spectra. Given two objects $C,D$, we will denote by $\Hom_\lC(C,D)$ the hom-spectrum and by $\hom_\lC(C,D)=\Omega^\infty\Hom_\lC(C,D)$ the underlying hom-space.  
\end{remark}
Note we have an adjunction 
\[
\begin{tikzcd}[column sep=2cm]
\Sp \arrow[r, shift left, "R \wedge -"] & \Mod_R \arrow[l, shift left, "{\Hom_{\Mod_R}(R,-)}"]
\end{tikzcd}, 
\]
where the right adjoint is equivalent to the forgetful functor.
\begin{definition}
	Let $R$ be a ring spectrum. An \emph{$R$-line} is an $R$-module $L$ such that $L \simeq R$.
\end{definition}
\begin{definition}
	Let $\Line_R$ be the full sub-$\infty$-groupoid of $\Mod_R$ spanned by the $R$-lines.
\end{definition}
By construction, $\Line_R$ is equivalent to the category with a single object $R$ and hom-space $GL_1R$, the $\infty$-group of $R$-linear automorphisms of $R$. Notice that $GL_1(R)\subseteq\hom_{\Mod_R}(R,R)\simeq\hom_{\Sp}(\bS,R)=\Omega^\infty R$. 
\begin{lemma}\label{lem:units}
	$GL_1(R)$ fits into the pullback square
	\begin{center}
		\begin{tikzcd}
			GL_1(R)\arrow[r]\arrow[d] & \Omega^\infty R\arrow[d]\\
			\pi_0(R)^\times \arrow[r] & \pi_0(R)
		\end{tikzcd}
	\end{center}In particular, the inclusion $GL_1(R)\to\Omega^\infty R$ induces an isomorphism on $n$-homotopy groups, for all $n\geq1$. 
\end{lemma}
\begin{proof}
	$\pi_0(R)\simeq\hom_{\Ho\Mod_R}(R,R)$, where $\Ho\Mod_R$ is the homotopy category of $R$-modules, the right-vertical arrow corresponds to $\hom_{\Mod_R}(R,R)\to\hom_{\Ho\Mod_R}(R,R)$ mapping a morphism to its homotopy class, and $\pi_0(R)^\times\subseteq\hom_{\Ho\Mod_R}(R,R)$ is the set of isomorphisms. Finally, a morphism in a $\infty$-category $\lC$ is an equivalence if and only if its homotopy class is an isomorphism in $\Ho\lC$. 
\end{proof}
\begin{definition}
	Let $X$ be a space. Denote by $\Mod_R(X)$ the $\infty$-category of $R$-module spectra parametrized over $X$, i.e. the functor category $\Fun(X^{\op},\Mod_R)$.
\end{definition}

\begin{definition}
	Let $X$ be a space. Denote by $\Line_R(X)$ the $\infty$-groupoid of $R$-line spectra parametrized over $X$, i.e. the functor category $\Fun(X^{\op},\Line_R)$. 
\end{definition}
Since $\Line_R\simeq BGL_1(R)$, functors $X^{\op}\to\Line_R$ classify $GL_1(R)$-principal bundles over $X$. In this sense, elements of $\Line_R(X)$ are $\infty$-local systems. 
\begin{example}
	Let $R_X\colon X^{\op} \to*\to \Line_R$ be the constant functor with value $R$. 
\end{example}
We now proceed to the Thom construction. We introduce it to full generality. 
\begin{definition}[{Thom spectrum}] \label{def:thom}
	The \emph{Thom $R$-module spectrum} is the functor
	\[
	R \colon \lS/\Mod_R \to \Mod_R,
	\]
	which sends $f\colon X^{\op} \to \Mod_R$ to the $R$-module spectrum $R^f$ given by colimit of $f$. 
\end{definition}
\begin{remark}The colimit of a functor $F:\lC\to\lX$ is the left Kan extension of $F$ along the terminal functor $p:\lC\to*$. If $\lX$ is cocomplete, a functor $\pi:\lC\to\lD$ induces a pullback functor $\pi^*:\Fun(\lD,\lX)\to\Fun(\lC,\lX)$, which is cocontinuous, therefore using left Kan extension along $\pi$ we can construct a left adjoint $\pi_!$.  
\end{remark}
Let us have a look at some examples of Thom spectra. 
\begin{example}[Trivial twist] We begin by identifying the Thom spectrum of $R_X$. By the universal property of colimits, the Thom spectrum functor is left adjoint to the functor $\Mod_R\to\lS/\Mod_R$ sending a module $E$ to the corresponding functor $*\to\Mod_R$. In particular, the Thom spectrum functor is cocontinuous. Next, consider a $R$-module $E$ and the functor $\lS\to\lS/\Mod_R$ sending a space $X$ to the constant functor $E_X:X^\op\to\Mod_R$ with value $E$. This functor is also cocontinuous, by the way colimits are calculated in $\lS/\Mod_R$. Therefore, the composition \begin{equation}
	\begin{tikzcd}
		\lS \arrow[r] & \lS/\Mod_R \arrow[r] & \Mod_R
	\end{tikzcd}
\end{equation}is cocontinuous, and so completely determined by the value of the one-point space $*$, which is $E$. On the other hand, the following composition \begin{equation}
	\begin{tikzcd}
		\lS \arrow[r,"\Sigma^\infty_+"] & \Sp \arrow[r,"E\wedge-"] & \Mod_R
	\end{tikzcd}
\end{equation}is also cocontinuous and sends $*$ to $E$. We conclude that the colimit of $E_X$, for every space $X$, is equivalent to $E\wedge\Sigma^\infty_+X$. 
\end{example}
For the next example, we require some preliminary lemmas. 
\begin{definition}
	Given a vector bundle $V\to B$, the \emph{Thom space} of $V$, denoted by $S^V$, is the homotopy cofiber of $V_0\subseteq V$, where $V_0$ is the complement of the zero section. 
\end{definition}
\begin{example}
	Let $B=*$ and $V=\bR^n$, then $S^V$ is homotopy equivalent to the $n$-sphere $S^n$. 
\end{example}
\begin{lemma}
	Let $\lC$ be a (small) category and $y:\lC\to\Fun(\lC^\op,\lS)$ the $\infty$-categorical Yoneda embedding. The colimit of $y$ is the terminal pre-sheaf, i.e. the pre-sheaf with constant value $*$.
\end{lemma}
\begin{proof}
	Apply the density theorem to the terminal pre-sheaf. 
\end{proof}

%--------------TO BE CONTINUED-----------------
We now proceed to define twisted cohomology theories.

\begin{definition}[{Twisted cohomology}] \label{def:twistedcohomology}
	Let $R$ be a ring spectrum, $X$ be a space, $p\colon X \to *$ the terminal functor, and $f\colon X^{\op} \to \Line_R$ a $R$-line bundle over $X$.  The \emph{$f$-twisted $R$-cohomology of $X$} is defined as the mapping spectrum 
	\[ R_f(X)	\coloneqq \Hom_{\Mod_R}(Mf, R) \simeq \Hom_{\Mod_R(X)}(f,p^*R) \simeq \Hom_{\Mod_R(X)}(f,R_X) .\]
	Similarly, the \emph{$f$-twisted $R$-homology of $X$} is defined as
	\[ R^f(X) \coloneqq \Hom_{\Mod_R}(R,Mf) \simeq Mf. \]
	The $f$-twisted $R$-cohomology groups of $X$ are defined as the homotopy groups of $R_f(X)$, i.e.
	\[ R^n_f(X) \coloneqq \pi_0(\Hom_{\Mod_R}(Mf, \Sigma^n R)) \cong \pi_{-n}(\Hom_{\Mod_R(X)}(f,R_X)). \]Similarly, the $f$-twisted $R$-homology groups of $X$ are defined as the homotopy groups of $R^f(X)$, i.e. 
	\[ R_n^f(X) \coloneqq \pi_0(\Hom_{\Mod_R}(\Sigma^n R, Mf)) \cong \pi_n(Mf). \]
\end{definition}

\begin{example}[Trivial twist] If $f:X\to \Line_R$ factors through $*$, then $f$ factors as $X^{\op}\to*\to\lS$, the constant factor with value $*$, and $R\wedge\Sigma^\infty_+(-):\lS\to\Mod_R$. The latter functor commutes with colimits, being a left adjoint, while the colimit of the latter is $X$ itself, then $Mf\simeq R\wedge\Sigma^\infty_+X$. In particular, $f$-twisted $R$-cohomology and $R$-homology of $X$ reduce to ordinary (untwisted) $R$-cohomology and $R$-homology of $X$. 
\end{example}


\section{Examples of Twisted Cohomology}
We now proceed to analyze several examples of twisted cohomology	theories. This requires some preliminary lemmas.

\begin{lemma}
	Consider a space $X$ and the $\infty$-categorical Yoneda's embedding $y:X\to\Fun(X^{\op},\lS)$. The colimit of $y$ is the terminal pre-sheaf on $X$, i.e. the pre-sheaf with constant value the one-point space. 
\end{lemma}

\begin{proof}
	Let $S$ be a pre-sheaf on $X$, consider then the slice category $X_{/S}$ of pairs $(x,\phi)$, where $x$ is an object of $X$ and $\phi:y(x)\to S$. The density theorem for $\infty$-categories states that $S$ is equivalent to the colimit of $X_{/S}\to X\xrightarrow{y}\Fun(X^{\op},\lS)$, the first map being the canonical projection. Take $S=*$, then $X_{/*}\to X$ is an equivalence, hence the claim. 
\end{proof}
Let $G$ be a topological group and $BG$ the $\infty$-groupoid with a single object $*$ and hom-space $G$. The category $\lS_G\coloneqq\Fun(BG,\lS)$ is equivalent to the category of $G$-spaces.
\begin{lemma}
	Consider $X=BG$, a $G$-space $f:X\to\lS$ and its left Kan extension $f_!:\lS_G\to\lS$, then $f_!\simeq(-\times E)/G$, where $E=f(*)$. 
\end{lemma}
\begin{proof}
	Evaluate at $*$, then $f_!(y(*))=E$, by definition, and $y(*)\simeq G$, as $G$-spaces, hence $(y(*)\times E)/G\simeq(G\times E)/G\simeq E$. Since $f_!$ and $(-\times E)/G$ agree on representables and are colimit-preserving, they are equivalent. 
\end{proof}
\begin{example}\label{ex:bundle}Take the space $BO(n)$ and $f_n:BO(n)\to\lS_*$ the $n$-sphere $S^n$ with $O(n)$-action coming from the one-point compactification of the regular action on $\bR^n$. Let $\alpha_n=\Sigma^{\infty-n}f_n:BO(n)\to\Sp$, then $\alpha_n(*)=\Sigma^{\infty-n}f_n(*)=\Sigma^{\infty-n}S^n\simeq\bS$, so $\alpha_n$ factors through $\Line_\bS$. Let $X=BO(n)^{\op}$ and $p:X\to*$ the terminal functor, then \[M\alpha_n=p_!\Sigma^{\infty-n}f_n\simeq\Sigma^{\infty-n}p_!(f_n)_!y\simeq\Sigma^{\infty-n}(f_n)_!\underbrace{p_!(y)}_{\simeq*}\simeq\Sigma^{\infty-n}(*\times S^n)/O(n)\]
Let $P=EO(n)$ be the universal $O(n)$-bundle and $M$ a $O(n)$-space, then $*\times_{O(n)} M$ is modelled by the \emph{strict} quotient $(P\times M)/O(n)$, then \[S^n/O(n)=\mathrm{cofib}(\bR^n_0\subseteq\bR^n)/O(n)\simeq\mathrm{cofib}(\underbrace{*\times_{O(n)}\bR^n_0}_{\simeq E^n_0}\subseteq\underbrace{ *\times_{O(n)}\bR^n}_{\simeq E^n})=\mathrm{Th}(E^n) \]
where $E^n=P\times_{O(n)}\bR^n\to BO(n)$ is the universal $n$-dimensional vector bundle, hence $M\alpha_n\simeq\Sigma^{\infty-n}\mathrm{Th}(E^n)$. 
\end{example}
The functor $BO(n)\to\Line_\bS$ induces a $\infty$-group homomorphism $j_n:O(n)\to GL_1(\bS)$, mapping $\phi$ to $\Sigma^{\infty-n}\mathrm{Th}(\phi)$. Consider the suspension morphism $s_n=\bR\oplus-:O(n)\to O(1+n)$, then $$j_n(\bR\oplus\phi)=\Sigma^{\infty-n-1}\mathrm{Th}(\bR\oplus\phi)\simeq\Sigma^{\infty-n-1}\underbrace{\mathrm{Th}(\bR)}_{\simeq S^1}\wedge\mathrm{Th}(\phi)\simeq\Sigma^{\infty-n}\mathrm{Th}(\phi)=j_n$$Recall that the colimit over the suspension morphisms $s_n$ is the stable orthogonal group $O$.
\begin{definition}
	Denote by $j$ the induced group homomorphism $O\to GL_1(\bS)$, called the \emph{$J$-homomorphism}.  
\end{definition} 
\begin{example}
	Let $X=O^{\op}$ and take $Bj:BO\to\Line_\bS$, then $Mj$ is denoted $MO$ and called the \emph{real bordism spectrum}. 
\end{example}
Denote by $M$ the extended Thom spectrum functor $\Grp^{\op}_\infty/\Mod_R\to\Mod_R$, this is a left adjoint to the functor sending a $R$-module to the corresponding functor $*\to\Mod_R$. In particular, $M$ preserves colimits and  $Bj\simeq\colim_nBj_n$, therefore we have the following:
\begin{theorem}
	$MO\simeq\colim_nMO(n)=\colim_n\Sigma^{\infty-n}\mathrm{Th}(E^n)$. 
\end{theorem}
\begin{example}\label{ex:bordismspectra}
	A group homomorphism $\xi:G\to O$ induces a functor $f:BG\to\Line_\bS$. The Thom spectrum $Mf$ is denoted $MG$ or $M\xi$, and called \emph{$G$-bordism spectrum}. For $G=U,SO,Spin$, and $String$, we obtain the \emph{complex, oriented, spin}, and \emph{string bordism spectra}. 
\end{example}
\begin{remark}
	In \cref{ex:bordismspectra} we might take $G=\{*\}$, the one-point group, then $MG\simeq\bS$, which, if it didn't have a name, might be called \emph{framed bordism spectrum}, following the naming convention in \cref{ex:bordismspectra} and in line with the theorem that $\pi_*(\bS)\simeq\Omega^\mathrm{fr}_*$, the bordism ring of framed (trivialized tangent bundle) smooth manifolds.  
\end{remark}
Let $R$ be a ring in sets, then $R$ is a $A_\infty$-ring spectrum (actually, $E_\infty$), $\Omega^\infty R$ is equivalent to $R$ with discrete topology ($\pi_0(\Omega^\infty R)\simeq R$, as sets, and every other homotopy group vanish). In particular, $GL_1(R)$ is simply $R^\times$ with discrete topology. Consider then the fiber sequence $SO\to O\to \bZ^\times\simeq GL_1(\bZ)$.
\begin{example}
	$X=BO^{\op}$ and $\alpha=w_1:BO\to\Line_\bZ$, the 1st Stiefel-Whitney class (delooping of the determinant $O\to GL_1(\bZ)$), then $Mw_1$ is a $\bZ$-module spectra. Let $i:SO\subseteq O$, then $w_1i$ factors through the point, so $M(w_1i)\simeq\bZ\wedge\Sigma^\infty_+SO$. 
\end{example}
Given $f:X^{\op}\to\Line_R$ and a sequence $F\xrightarrow{i} Y\xrightarrow{\pi} X$, there is an induced sequence of Thom $R$-module spectra $MF\to MY\to MX$. If $\pi i$ factors through the point, $MF\simeq R\wedge\Sigma^\infty_+F$. 
\begin{lemma}\label{lem:adj}
	Let $R$ be a ring spectrum and $X$ a connected monoidal $\infty$-groupoid, then \[\hom_{\Mon(\Sp)}(\Sigma^{\infty}_+X,R)\simeq\hom_{\Mon(\lS)}(X,GL_1(R))\] 
\end{lemma}
\begin{proof}
	Since $X$ is connected, the space of homomorphisms $X\to GL_1(R)$ is equivalent to the space of homomorphisms $X\to\Omega^\infty R$, then use that $(\Sigma^\infty,\Omega^\infty)$ is a monoidal adjunction (The monoidal structure on spectra is such that $\Sigma^\infty$ is strong monoidal). 
\end{proof}
\begin{remark}
	Notice that we can weaken the result. Namely, if $X$ is 1-connected (pointed and connected), then the space of functors (of $\infty$-groupoids) $X\to GL_1(R)$ is equivalent to the space of functors $X\to\Omega^\infty R$ such that $*\to X\to\Omega^\infty R$ is an equivalence. This last space is equivalent, via the $(\Sigma^\infty_+,\Omega^\infty)$ adjunction, to the space of morphisms of spectra $\Sigma^\infty_+X\to R$, such that $\bS\to\Sigma^\infty_+X\to R$ represents a unit in $\pi_0(R)$.
\end{remark}
\begin{remark}\label{rmk:strengthen}
	Notice that we can also strengthen the result. Namely, if $X$ is a connected, commutative monoid object and $R$ is a commutative ring spectrum, then $\Omega^\infty R$ and $GL_1(R)$ are also commutative monoid objects. Using the same argument, together with the fact that $(\Sigma^\infty,\Omega^\infty)$ is actually a \emph{symmetric} monoidal adjunction, we conclude that \[\hom_{\CMon(\Sp)}(\Sigma^\infty_+X,R)\simeq\hom_{\CMon(\lS)}(X,GL_1(R))\] 
\end{remark}
\begin{remark}\label{rmk:moncat}
	Let $\lD$ be a monoidal $\infty$-category. Consider $\cat_\infty/\lD$, the $\infty$-category of functors into $\lD$, with monoidal structure given by \begin{center}
		\begin{tikzcd}
			(F:\lA\to\lD,G:\lB\to\lD) \arrow[r,|->] & (\lA\times\lB\xrightarrow{F\times G}\lD\times\lD\xrightarrow{\otimes}\lD)
		\end{tikzcd}
	\end{center} The monoidal unit is the functor $*\to\lD$ picking out the monoidal unit of $\lD$. If $\lD$ is symmetric monoidal, then so is $\cat_\infty/\lD$. A (commutative) monoid object in $\cat_\infty/\lD$ is given by a (symmetric) monoidal category $\lC$ and a (symmetric) monoidal functor $F:\lC\to\lD$. 
\end{remark}
In view of \cref{rmk:moncat}, let $R$ be commutative ring spectrum, then $\Mod_R$ is a symmetric monoidal $\infty$-category and $\Line_R$ is a symmetric monoidal $\infty$-groupoid. The category $\Grpd^{\op}_\infty/\Line_R$ is then symmetric monoidal and (commutative) monoid objects are given by (symmetric) monoidal $\infty$-groupoids $X^{\op}$ a (symmetric) monoidal functors $X^{\op}\to\Line_R$. One can then check that $M$ is a symmetric monoidal functor, so that (commutative) monoid objects are sent to (commutative) monoid objects in $\Mod_R$, i.e. (commutative) $R$-algebras. 
\begin{example}\label{ex:tmf}
	Let $\mathrm{Tmf}$ be the commutative ring spectrum of topological modular forms (see \cref{rmk:names}) and $\sigma:MString\to\mathrm{tmf}$ the $String$-orientation of $\mathrm{tmf}$. In the sequence \begin{center}
		\begin{tikzcd}
			BString \arrow[r] & BO \arrow[r,"Bj"] & \Line_\bS
		\end{tikzcd}
	\end{center}all functors are symmetric monoidal, so that $MString$ is a commutative $\bS$-algebra, i.e. a commutative ring. The $String$-orientation of $\mathrm{tmf}$ is also a commutative ring homomorphism. In the fiber sequence $K(\bZ,3)\to BString\to BSO$, the fiber map $i:K(\bZ,3)\to BString$ is also a symmetric monoidal, so the composition 
	\begin{center}
		\begin{tikzcd}
			\Sigma^\infty_+K(\bZ,3)\arrow[r,"Mi"] & MString \arrow[r,"\sigma"] & \mathrm{tmf}
		\end{tikzcd}
	\end{center}is a commutative ring homomorphism. Using \cref{lem:adj}, we conclude that the induced homomorphism $K(\bZ,3)\to\Omega^\infty\mathrm{tmf}$ (which is a homomorphism of commutative monoid objects, given \cref{rmk:strengthen}) factors through $GL_1(\mathrm{tmf})$, and so it induces a (symmetric monoidal) functor $K(\bZ,4)\to\Line_\mathrm{tmf}$, i.e. \emph{2-bundle gerbes twist $\mathrm{tmf}$}.
\end{example}
\begin{remark}\label{rmk:names}The spectrum of topological modular forms comes in three main flavors, namely:
\begin{enumerate}
	\item $\mathrm{TMF}$, i.e. the global sections of the spectral structure sheaf $\rO^{top}:(\mathrm{Aff}/\lM_{ell})^{\op}\to\CMon(\Sp)$ on the (\' etale site of the) moduli stack of elliptic curves.
	\item $\mathrm{Tmf}$, i.e. the global sections of the spectral structure sheaf $\bar{\rO}^{top}:(\mathrm{Aff}/\bar{\lM}_{ell})^{\op}\to\CMon(\Sp)$ on the (\' etale site of the) \emph{compactified} moduli stack of elliptic curves. The inclusion $\lM_{ell}\hookrightarrow\bar{\lM}_{ell}$ induces a commutative ring homomorphism $\mathrm{Tmf}\to\mathrm{TMF}$.
	\item $\mathrm{tmf}$, i.e. the connective cover of $\mathrm{Tmf}$. By definition, there is a commutative ring homomorphism $\mathrm{tmf}\to\mathrm{Tmf}$. 
\end{enumerate}In \cite{ando2010multiplicative}, $\mathrm{tmf}$ is used to denote our $\mathrm{TMF}$, in \cite{goerss2009topologicalmodularformsaftern}, $\mathrm{tmf}$ is used to denoted our $\mathrm{Tmf}$, and \cite{douglas2014topological} has the same notation as us. In \cref{ex:tmf}, we use $\mathrm{tmf}$ to mean the connective cover of $\mathrm{Tmf}$. 
\end{remark}
Let us go down one step in the chromatic ladder. 
\begin{example}\label{ex:ktheory}
	Recall the fiber sequences for $Spin$ and $Spin^c$: \[\bZ_2\to Spin\to SO,\qquad S^1\to Spin^c\to SO\] All the spaces involved are commutative groups. Applying the Thom spectrum functor to the delooped sequences, we get\[\Sigma^\infty_+K(\bZ_2,1)\to MSpin\to MSO,\qquad\Sigma^\infty_+K(\bZ,2)\to MSpin^c\to MSO \]
	Let $\sigma:MSpin\to KO$ and $\sigma^c:MSpin^c\to KU$ be the Atiyah-Bott-Shapiro orientation of real and complex $K$-theory (see \cite{atiyahbottshapiro1964clifford}).
	Similar to \cref{ex:tmf}, we get homomorphisms 
	\begin{center}
		\begin{tikzcd}
			\Sigma^\infty_+K(\bZ_2,1)\arrow[r] & MSpin \arrow[r,"\sigma"] & KO, & \Sigma^\infty_+K(\bZ,2)\arrow[r] & MSpin^c \arrow[r,"\sigma^c"] & KU
		\end{tikzcd}
	\end{center}
	Using \cref{lem:adj} again and delooping, we obtain functors $K(\bZ_2,2)\to\Line_{KO}$ and $K(\bZ,3)\to \Line_{KU}$, i.e. \emph{real, resp. complex, bundle gerbes twist real, resp. complex, $K$-theory}. 
\end{example}

\section{Twists via Picard Groupoids and Grading}
This section requires some some further details. Up until now we defined everything via $\Line_R$, however for many applications we need to work with $\Pic_R$ instead.
\begin{definition}
	Given a monoidal $\infty$-category $(\lC,\otimes,1)$, an object $M$ is \emph{invertible} if there is an object $N$ such that $N\otimes M\simeq M\otimes N\simeq 1$. The \emph{Picard $\infty$-groupoid} of $\lC$ is the sub-$\infty$-groupoid generated by invertible modules. 
\end{definition}
\begin{remark}\label{rmk:dualinver}
	A monoidal category $(\lC,\otimes,1)$ is \emph{closed} if, for every object $M$, the \emph{left tensoring with $M$} functor $M^\otimes:\lC\to\lC$ admits a right adjoint $F(M,-)$. If $M$ is invertible, with inverse $N$, then $M^\otimes$ is an equivalence with $N^\otimes$ as inverse. In particular, we can promote $(M^\otimes,N^\otimes)$ to an adjoint equivalence. By uniqueness of adjoint functors $F(M,-)\simeq N^\otimes$, and so \[N\simeq N\otimes1= N^\otimes(1)\simeq F(M,1)=:DM\] 
\end{remark}
\begin{definition}
	Given a closed monoidal category $(\lC,\otimes,1)$, the functor $D:=F(-,1):\lC^{\op}\to\lC$ will be called \emph{duality}. Given an object $M$, the object $DM$ is called the \emph{dual of $M$}. 
\end{definition}
\begin{definition}
If $R$ is a ring spectrum, $\Mod_R$ is closed monoidal. Denote by $\Pic_R$ the Picard $\infty$-groupoid of $\Mod_R$. 
\end{definition}
\begin{remark}
	$\Sigma^nR$ is invertible, with inverse $\Sigma^{-n}R$. In particular, there is a map $\bZ\times\Line_R\to\Pic_R$. However, this map need not be neither injective (if $R$ is $n$-periodic), nor surjective (see \cite{hill2011tmf}). 
\end{remark}
As mentioned in, the Thom spectrum functor makes sense for every functor $f:X^{\op}\to\Mod_R$. However, all examples of twists encountered so far came from functors into $\Line_R$. An example of twist that is not the result of a $R$-line bundle is the \emph{degree shift}. 
\begin{definition}
	Denote by $M$ the \emph{Thom $R$-module spectrum} functor \[\Grpd^{\op}_\infty/\Mod_R\to\Mod_R\]sending a functor $f:X^{\op}\to\Mod_R$ to its colimit.  
\end{definition}
\begin{example}
	Let $f:X^{\op}\to\Line_R$ be a twist. Denote by $\Sigma^nf$ the composition of $f$ with the shift functor $\Sigma^n:\Line_R\to\Pic_R$. Since $\Sigma^n$ is an equivalence, it commutes with colimits, so $$M\Sigma^nf\simeq \Sigma^nMf$$If $f=R_X$, then $M\Sigma^nf\simeq \Sigma^nR\wedge\Sigma^\infty_+X$, so  $\Sigma^nf$-twisted $R$-cohomology and $R$-homology correspond to normal $R$-cohomology and $R$-homology with a degree shift by $n$. 
\end{example}

\section{Umkehr Map}
We now proceed to the construction of the umkehr map in twisted cohomology theories. Here we follow \cite{abg2018thom}. Let $R$ be a ring spectrum, denote by $D_R$ the duality of $\Mod_R$. Given a invertible $R$-twist $\alpha:X^{\op}\to\Pic_R$, denote by $D_R\alpha$ the post-composition of $\alpha^{\op}$ with $D_R$.

\begin{remark} Depending on which cohomological degrees we want, we sometimes use the following alternative definition of	twisted cohomology:
	\[ R^{\alpha}(X) =\pi_0\hom_{\Mod_R}(M(D_R\alpha), R) \]
 meaning we use the dual/inverse of $\alpha$. To justify the use of $D_R\alpha$, consider the following: Let $\beta=D_R\alpha$, then
 \[\hom_{\Mod_R}(M\beta,R)\simeq\hom_{\Mod_{R}(X)}(D_R\alpha,R_X)\simeq\hom_{\Mod_R(X)}(R_X,\alpha\otimes_{R_X}R_X)\simeq\hom_{\Mod_R(X)}(R_X,\alpha)\]The second equivalence is a consequence of the fact that, if $M$ be an invertible $R$-module, left tensoring with $M$ is left adjoint to tensoring with $D_RM$, and $D_RD_RM\simeq M$, see \cref{rmk:dualinver}. If we think of $\alpha$ as a bundle of invertible $R$-modules over $X$, then $\hom_{\Mod_R(X)}(R_X,\alpha)$ is the spectrum of global sections of $\alpha$, which aligns with the idea that {twisted cohomology is the homotopy groups of the global sections of a bundle of spectra}. 
\end{remark}
If $R=\bS$, we denote the duality $D_R$ by simply $D$. 
\begin{definition}
	The \emph{Spanier-Whitehead dual} of a space $X$ is the dual spectrum of $\Sigma^\infty_+X$ with respect to the sphere spectrum.
\end{definition}
\begin{remark}\label{rmk:monunit}
	Recall that $\Sp$ is a closed symmetric monoidal category, with $\Hom_{\Sp}(E,-)$ being the right adjoint to the functor given by smash product with $E$. Let $(\lC,\otimes,1)$ a closed monoidal category, where $F(X,-)$ is the right adjoint to the functor given by left tensoring with $X$, then \[\\hom_\lC(-,F(1,E))\simeq\hom_\lC(1\otimes-,E)\simeq\hom_\lC(-,E)\]Using Yoneda's lemma, we conclude that $E\simeq F(1,E)$, for all $E$. 
\end{remark}
\begin{example}
	By \cref{rmk:monunit}, the Spanier-Whitehead dual to $*$ is the sphere spectrum, since \[D*=\Hom(\Sigma^\infty_+*,\bS)=\Hom(\Sigma^\infty S^0,\bS)=\Hom(\bS,\bS)\simeq\bS\]
\end{example}
\begin{definition}
	Denote by $\phi$ the functor $\lS^{\op}\to\bS/\Sp$ from spaces to the category of spectra under $\bS$, mapping $X$ to the map of spectra \begin{center}
		\begin{tikzcd}
			\phi(X):\bS\simeq D*\arrow[r,"Dp"] & DX
		\end{tikzcd}
	\end{center}where $p$ is the terminal map $X\to *$.
\end{definition} 
\begin{definition}
	Denote by $\widehat{\Mfd}_\infty$ the topological groupoid of closed smooth manifolds and diffeomorphisms. The set of diffeomorphisms are topologized by the weak $C^\infty$ topology, see \cite{hirsch1994differentialtopology}. Let $\Mfd_\infty$ be homotopy coherent nerve of $\widehat{\Mfd}_\infty$. 
\end{definition}
Let $B$ be a connected compact space and $\pi\colon X \to B$ a continuous function such that, for every $b\in B$, the fiber $\pi^{-1}(b)=:X_b$ is equipped with the structure of a closed smooth manifold, which varies continuously in $B$, in the sense of being classified by a functor
\[f:B \to \Mfd_\infty\]
Given such a classifying map $f$, consider the following composition \begin{center}
	\begin{tikzcd}
		B^{\op}\arrow[r] & \Mfd_\infty^{\op} \arrow[r] & \lS^{\op} \arrow[r,"\phi"] & \bS/\Sp
	\end{tikzcd}
\end{center}where the middle functor is the one forgetting the smooth structure. The above composition is then an object of
\[ \Fun(B^{\op},\bS/\Sp) \simeq \bS_B/\Fun(B^{\op}, \Sp) \]i.e. it is a natural transformation \[\phi_{X/B}:\bS_B\to D_B(f)\]where $D_B$ denotes the Spanier-Whithead duality applied to (the opposite of) $f:B\to\lS$ point-wise. Let $p:B\to *$ be the terminal functor and $p_!:\Fun(B^{\op},\Sp)\to\Sp$ the left Kan extension along $p$, i.e. the colimit functor. Applying $p_!$ to $\phi_{X/B}$, we obtain a morphism of spectra \begin{equation}\label{eq:thom}
	\begin{tikzcd}
		\Sigma_+^\infty B\simeq p_!p^*\bS=p_!\bS_B \arrow[rr,"p_!(\phi_{X/B})"] & & p_!D_B(f)
	\end{tikzcd}
\end{equation}
Let $T_{X/B}\to X$ be the vector bundle of fiber-wise tangent vectors, classified by $X^{\op}\to BO(n)$, where $n$ is the manifold dimension of the fibers\footnote{$B$ is connected, so the fibers have constant dimension.}. Denote by $\alpha$ the composition \begin{equation}\label{eq:alpha}
	\begin{tikzcd}
		X^{\op}\arrow[r,"T_{X/B}"] & BO(n) \arrow[r] & \lS_* \arrow[r,"\Sigma^\infty"] & \Sp
	\end{tikzcd}
\end{equation}the middle arrow being the $n$-sphere $S^n$ with $O(n)$-action coming from the one-point compactification of the regular action on $\bR^n$. The above diagram takes values in the Picard $\infty$-groupoid of the sphere spectrum. 
\begin{theorem}\label{thm:pontry}
	$p_!D_B(f)\simeq M(-\alpha)=:X^{-T_{X/B}}$.
\end{theorem}
\begin{definition}
	The morphism 
	\begin{center}
		\begin{tikzcd}
		\PT(f):\Sigma^{\infty}_+B\arrow[r] & X^{-T_{X/B}}	
		\end{tikzcd}
	\end{center}
	is called \emph{Pontryagin-Thom transfer map}. 
\end{definition}

Consider now a general situation. Let $R$ be a ring spectrum and $\alpha:X^{\op}\to\Line_R$ a $R$-line bundle. 
\begin{definition}
	An \emph{orientation} of $M\alpha$ is lift 
	\begin{center}
		\begin{tikzcd}
			& *\arrow[d]\\
			X^{\op}\arrow[r,"\alpha"']\arrow[ru,dashed] & \Line_R
		\end{tikzcd}
	\end{center}Explicitly, it is an equivalence of $R_X$-modules $\alpha\to R_X$. 
\end{definition}
Applying $p_!$, we see that an equivalence $t:\alpha\to R_X$ induces an equivalence
\begin{center}
	\begin{tikzcd}
		M\alpha\arrow[r,"p_!(t)"] & R\wedge\Sigma^{\infty}_+X
	\end{tikzcd}
\end{center}In our case, $\alpha$ in \cref{eq:alpha} is not valued in $\Line_\bS$, but $\Sigma^{-n}\alpha$ is. An orientation of $\Sigma^{-n}\alpha$ induces an equivalence \begin{equation}\label{eq:thomiso}
	\begin{tikzcd}
		\Sigma^{-n}M\alpha\arrow[r,"\simeq"] & R\wedge\Sigma^{\infty}_+X
	\end{tikzcd}
\end{equation}
\begin{definition}
	The isomorphism in cohomology induced by \cref{eq:thomiso} is called \emph{Thom isomorphism}. 
\end{definition}
Finally, we have the following definition:
\begin{definition}
	Assuming $\Sigma^{-n}\alpha:X^{\op}\to\Line_R$ is orientable, the \emph{Umkehr map} is the map: 
	\begin{center}
		\begin{tikzcd}
			R^*(\Sigma^\infty_+X)\arrow[rr,"\text{Thom iso.}"] & & R^{*-n}(X^{-T_{X/B}}) \arrow[rr,"\PT(f)^{*-n}"] & & R^{*-n}(\Sigma^{\infty}_+B)
		\end{tikzcd}
	\end{center}
\end{definition}
\begin{remark}
	The name \emph{Umkehr} comes from the map going in the opposite direction to $R^*(\Sigma^\infty_+X)\to R^*(\Sigma^\infty_+B)$, given by pulling back along $\pi$. 
\end{remark}
We now introduce the \emph{twisted Umkehr map}. Given a twist $\beta\colon B^{\op} \to \Pic_R$, we can smash to get a map 
\begin{center}
	\begin{tikzcd}
		\beta\simeq\bS_B\wedge_B\beta \arrow[r] & D_B(f)\wedge_B\beta
	\end{tikzcd}
\end{center}
Applying the functor $p_!$ once again, we obtain a morphism \begin{center}
	\begin{tikzcd}
		M\beta \arrow[r] & p_!(D_B(f)\wedge_B\beta)\simeq X^{-T_f+\beta\pi}
	\end{tikzcd}
\end{center}
where $\beta\pi:X^{\op}\to\Pic_R$ is the composition of $\pi$ and the twist $\beta$.
\begin{definition}
	The \emph{twisted Pontryagin-Thom transfer map} is the morphism:
	\begin{center}
		\begin{tikzcd}
			\PT(f,\beta):M\beta\arrow[r] & X^{-T_f+\beta\pi} 
		\end{tikzcd}
	\end{center}The \emph{twisted Umkehr map} is the map of twisted cohomology groups induced by $\PT(f,\beta)$. 
\end{definition}
\begin{example}
	Assume that $X^{-T_{X/B}+\beta\pi}$ is orientable, i.e. the following diagram commutes \begin{center} 
		\begin{tikzcd}
			X \arrow[r, "-T_f"] \arrow[d] & \Pic_\bS \arrow[d] \\ 	B \arrow[r, "\alpha"] & \Pic_R 
		\end{tikzcd}
	\end{center}then $X^{-T_{X/B}+\beta\pi}\simeq R\wedge\Sigma^\infty_+X$ and the twisted Umkehr map becomes:
	\begin{center}
		\begin{tikzcd}
			R^*(\Sigma^\infty_+X)\arrow[r] & R^{*-\beta}(B)
		\end{tikzcd}
	\end{center}
\end{example}

{\footnotesize
 \bibliographystyle{alpha}
 \bibliography{main}
 }
\end{document}