\documentclass[10pt]{amsart}
\usepackage{amsmath,amsthm,amssymb,amsfonts}
\usepackage[mathscr]{euscript}
\usepackage{tikz}
\usepackage{tikz-cd}
\usepackage{enumerate}
\usepackage{enumitem}
\usepackage{mathtools}
\usepackage[colorlinks=true, linkcolor=red, citecolor = blue]{hyperref}
\usepackage[margin=2.5cm]{geometry}
\setlength{\marginparwidth}{2cm}

\usepackage[nameinlink,capitalise,noabbrev]{cleveref}

\usepackage[textwidth=2cm, textsize=small, colorinlistoftodos]{todonotes}

\newcommand{\cF}{\mathcal{F}}
\newcommand{\sC}{\mathscr{C}}
\newcommand{\sS}{\mathscr{S}}
\newcommand{\bS}{\mathbb{S}}
\newcommand{\bR}{\mathbb{R}}
\newcommand{\bZ}{\mathbb{Z}}
\newcommand{\bC}{\mathbb{C}}
\newcommand{\fO}{\mathbf{O}}

\newcommand{\Cyc}{\mathrm{Cyc}}
\newcommand{\Def}{\mathrm{Def}}
\newcommand{\curv}{\mathrm{curv}}
\newcommand{\CS}{\mathrm{CS}}
\newcommand{\ch}{\mathrm{ch}}
\newcommand{\dKU}{\smash{\widehat{ku}}}
\newcommand{\dKUnabla}{\smash{\widehat{ku}^\nabla}}
\renewcommand{\sp}{\mathrm{sp}}

\DeclareMathOperator{\tr}{tr}
\newcommand{\Map}{\mathrm{Map}}
\newcommand{\Hom}{\mathrm{Hom}}
\newcommand{\Ho}{\mathrm{Ho}}
\newcommand{\set}{\mathscr{S}\mathrm{et}}
\newcommand{\CMon}{{\normalfont\texttt{CMon}}}
\newcommand{\CGrp}{{\normalfont\texttt{CMon}}}
\newcommand{\fib}{{\normalfont\texttt{fib}}}
\newcommand{\cofib}{{\normalfont\texttt{cofib}}}
\newcommand{\Bun}{{\normalfont\texttt{Bun}}}
\newcommand{\Sp}{\mathscr{S}\mathrm{p}}
\newcommand{\Ch}{\mathscr{C}\mathrm{h}}
\newcommand{\cat}{\mathscr{C}\mathrm{at}}
\newcommand{\scat}{s\mathscr{C}\mathrm{at}}
\newcommand{\sset}{s\mathscr{S}\mathrm{et}}
\newcommand{\Line}{\mathscr{L}\mathrm{ine}}
\newcommand{\Fun}{\mathrm{Fun}}
\newcommand{\Nat}{\mathrm{Nat}}
\newcommand{\colim}{\mathrm{colim}}
\newcommand{\Top}{\mathscr{T}\mathrm{op}}
\newcommand{\Grpd}{\mathscr{G}\mathrm{rpd}}
\newcommand{\Grp}{\mathscr{G}\mathrm{rp}}
\newcommand{\Euc}{\mathscr{E}\mathrm{uc}}
\newcommand{\Mfd}{\mathscr{M}\mathrm{fd}}
\newcommand{\Kan}{\mathscr{K}\mathrm{an}}
\newcommand{\Vect}{\mathscr{V}\mathrm{ect}}
\newcommand{\Mod}{\mathscr{M}\mathrm{od}}
\newcommand{\Proj}{\mathscr{P}\mathrm{roj}}
\newcommand{\Ab}{\mathscr{A}\mathrm{b}}
\newcommand{\Shv}{\mathscr{S}\mathrm{hv}}
\newcommand{\Yon}{\mathscr{Y}\mathrm{on}}
\newcommand{\Open}{\mathscr{O}\mathrm{pen}}
\newcommand{\PSh}{\mathscr{P}\mathscr{S}\mathrm{h}}
\newcommand{\Pic}{\mathscr{P}\mathrm{ic}}
\newcommand{\triv}{\mathscr{T}\mathrm{riv}}
\newcommand{\aut}{\mathrm{Aut}}
\newcommand{\Th}{\mathrm{Th}}
\newcommand{\fr}{\mathrm{Fr}}
\newcommand{\Arr}{\mathrm{Arr}}
\newcommand{\ev}{\mathrm{ev}}

\newcommand{\bbefamily}{\fontencoding{U}\fontfamily{bbold}\selectfont}
\newcommand{\textbbe}[1]{{\bbefamily #1}}
\DeclareMathAlphabet{\mathbbe}{U}{bbold}{m}{n}

\def\DDelta{{\mathbbe{\Delta}}}
\newcommand{\DD}{\DDelta}



%% N.R. notes
\newcommand{\nrnote}[1]{\todo[color=green!40,linecolor=green!40!black,size=\tiny]{#1}}
\newcommand{\nrmpar}[1]{\todo[noline,color=green!40,linecolor=green!40!black,
  size=\tiny]{#1}}
\newcommand{\nrnoteil}[1]{\ \todo[inline,color=green!40,linecolor=green!40!black,size=\normalsize]{#1}}

\newtheorem{theorem}[equation]{Theorem}
\newtheorem{lemma}[equation]{Lemma}
\newtheorem{proposition}[equation]{Proposition}
\newtheorem{corollary}[equation]{Corollary}
% \newtheorem{statement}[section]{Statement}

\theoremstyle{definition}
\newtheorem{definition}[equation]{Definition}
\newtheorem{example}[equation]{Example}
% \newtheorem{attone}[equation]{Attention}

\theoremstyle{remark}
\newtheorem{remark}[equation]{Remark}
% \newtheorem{intone}[equation]{Intuition}
\newtheorem{notation}[equation]{Notation}
% \newtheorem{queone}[equation]{Question}
% \newtheorem{conjone}[equation]{Conjecture}
\newtheorem{warning}[equation]{Warning}

\numberwithin{equation}{section}

\title{Differential Cohomology Seminar 8}
\date{12.11.2025 \& 26.11.2025}
\author{Talk by Hannes Berkenhagen}

\begin{document}
\maketitle

The aim of this talk is to review twisted cohomology theory with the aim of later discussing twisted differential cohomology theories \cite{bunkegepner2021differential}. For these talks the main source is \cite{abghr2014infty,abg2018thom}.

\section{Twisted Cohomology}
Let $R$ be a ring spectrum, meaning a monoid object in the $\infty$-category of spectra $\Sp$. From this we get a presentable stable $\infty$-category $\Mod_R$ of left $R$-module spectra. Objects therein are morphisms of the form $R \wedge M \to M$ satisfying the usual associativity and unit conditions up to coherent homotopies.

Note we have an adjunction diagram 
\[
\begin{tikzcd}[column sep = 2cm]
\Sp \arrow[r, shift left=1.8, "R \wedge -", "\bot"'] & \Mod_R \arrow[l, shift left=1.8, "{\Hom_R(R,-)}"]
\end{tikzcd}, 
\]
where the right adjoint is in fact the forgetful functor. This in particular means $\Mod_R$  has a distinguished object $R$ i.e. the free $R$-module of rank $1$. We now refine these constructions. 

\begin{definition}
	Let $R$ be a ring spectrum. An \emph{$R$-line} is an $R$-module $L$ such that $L \simeq R$.
\end{definition}

\begin{definition}
	Let $\Line_R$ be the full sub-$\infty$-groupoid of $\Mod_R$ spanned by the $R$-lines.
\end{definition}
Note by construction this is equivalent to $BGL_1(R)$, the $\infty$-group of $R$-linear automorphisms of $R$.
\begin{definition}
	Let $X$ be a space. Then we denote by $\Mod_R(X)$ the $\infty$-category of $R$-module spectra parametrized over $X$, i.e. the functor category $\Fun(X^{op},\Mod_R)$.
\end{definition}

\begin{definition}
	Let $X$ be a space. Then we denote by $\Line_R(X)$ the full sub-$\infty$-groupoid of $\Mod_R(X)$ spanned by those functors $L \colon X^{op} \to \Line_R$, meaning it is $\Fun(X^{op},\Line_R)$.
\end{definition}

Recall this generalization of local systems of abelian groups, which map out of groupoids instead of $\infty$-groupoids.

\begin{example}
	Let $R_X\colon X^{op} \to \Line_R$ be the constant functor, then this is an object in $\Line_R(X)$.
\end{example}

We now proceed to the Thom construction.

\begin{definition}[{Thom spectrum}] \label{def:thom}
	The \emph{Thom $R$-module spectrum} is the functor
	\[
	M \colon \Grpd_{\infty}^{op}/\Line_R \to \Mod_R,
	\]
	which sends $f\colon X^{op} \to \Line_R$ to the $R$-module spectrum $\colim(X^{op} \xrightarrow{f} \Line_R \to \Mod_R)$.
\end{definition}

Let us note an alternative characterization that will be important later.
\begin{remark}
	For a given map $f\colon X \to Y$, we get a map $f^*\colon \Mod_R(Y) \to \Mod_R(X)$ by precomposition with $f^{op}$. By the adjoint functor theorem,	this functor admits a left adjoint $f_! \colon \Mod_R(X) \to \Mod_R(Y)$ and a right adjoint $f_* \colon \Mod_R(X) \to \Mod_R(Y)$. Now for $p\colon X \to *$, we have \[Mf \simeq p_!(i \circ f).\]
\end{remark}
\begin{remark}[{\cite[3.6]{andoblumberggepner2010twistedktmf}}]
	Let $\triv_R$ be the slice groupoid $\Line_R/R$, together with the canonical projection $\pi:\triv_R\to\Line_R$. An object of $\triv_R$ is a pair $(L,\phi)$ of a $R$-line $L$ and an equivalence $\phi:L\to R$. Consider then the following: \begin{enumerate}
		\item $\triv_R$ is a slice $\infty$-groupoid, hence contractible, and $\pi$ is a Kan fibration. 
		\item $GL_1(R)$ is equivalent to the fiber of $\pi$ over $R\in\Line_R$ and acts freely on the fibers of $\pi$. 
\end{enumerate}These observations imply $\Line_R$ is equivalent to $BGL_1(R)$, the classifying space for $R$-line bundles. 
\end{remark}
We now proceed to define twisted cohomology theories.

\begin{definition}[{Twisted cohomology}] \label{def:twistedcohomology}
	Let $R$ be a ring spectrum, $X$ be a space and let $p\colon X \to *$ be the projection map, and let $f\colon X^{op} \to \Line_R$ be an $R$-line bundle over $X$.  The \emph{$f$-twisted $R$-cohomology of $X$} is defined as the mapping spectrum 
	\[ R_f(X)	\coloneqq \Map_{\Mod_R}(Mf, R) \simeq \Map_{\Mod_R(X)}(f,p^*R) \simeq \Map_{\Mod_R(X)}(f,R_X) .\]
	Similarly, we have \emph{$f$-twisted $R$-homology of $X$} defined as
	\[ R^f(X) \coloneqq \Map_{\Mod_R}(R,Mf) \simeq Mf. \]
\end{definition}

In particular we can define the twisted cohomology groups of a space $X$ with twist given by a map $f\colon X \to \Line_R$, by 
\[ R^n_f(X) \coloneqq \pi_0(\Map_{\Mod_R}(Mf, \Sigma^n R)) \cong \pi_{-n}(\Map_{\Mod_R(X)}(f,R_X)). \]
Similarly we define the twisted homology groups by
\[ R_n^f(X) \coloneqq \pi_0(\Map_{\Mod_R}(\Sigma^n R, Mf)) \cong \pi_n(Mf). \]

\begin{remark}
	 Note that if $f$ is the constant map with value $R$, then we recover ordinary $R$-cohomology and $R$-homology. Concretely, we have 
	\[ Mf \simeq R \wedge \Sigma^{\infty}_+ X\]
	and so it follows that 
	
	\[ R_f(X) \simeq \Map_{\Mod_R}(R \wedge \Sigma^{\infty}_+ X, R) \simeq \Map_{\Sp}(\Sigma^{\infty}_+ X, R) \]
	Applying $\pi_{-n}$ this recovers the regular $R$-cohomology groups of $X$, and analogous for homology.
\end{remark}

Let us	look at some examples of twisted cohomology theories.
\begin{example}Let $BO(n)$ be the topological groupoid of $n$-dimensional, inner product spaces and orthogonal morphisms, viewed as a $\infty$-groupoid. Consider the composition
\begin{center}
	\begin{tikzcd}
		f_n:BO(n) \arrow[r,"\Th_n"] & \sS_* \arrow[r,"\Sigma^{\infty-n}"] & \Sp
	\end{tikzcd}
\end{center}	
where the first functor maps a inner product space $V$ to its one-point compactification. Choosing a orthonormal basis for $V$, we get an isomorphism $\bR^n\simeq V$, then $\Sigma^{\infty-n}\Th(V)\simeq\Sigma^{\infty-n}\Th(\bR^n)\simeq\Sigma^{\infty-n}S^n\simeq\bS$, so that $f_n$ factors through $\Line_\bS$. Let $p:BO(n)\to*$, then $$Mf_n\simeq p_!\Sigma^{\infty-n}\Th_n\simeq\Sigma^{\infty-n}p_!\Th_n\simeq\Sigma^{\infty-n}\Th(E_n)$$
where $E_n\to BO(n)$ is the universal $n$-dimensional vector bundle and $\Th(E_n)$ is the one-point compactification of the total space $E_n$. To prove the last equivalence, consider the following facts:
\begin{enumerate}
	\item Given a inner product space $V$, the Thom space of $V$ is homotopy equivalent to the homotopy cofiber of the inclusion $V\setminus\{0\}\subseteq V$. 
	\item $\Th_n$ factors as the cofiber functor $C:\Arr(\sS)\to \sS_*$ following the functor $F:BO(n)\to\Arr(\sS)$ mapping $V$ to the inclusion $V\setminus\{0\}\subseteq V$. Here, $\Arr(\sS)$ denotes the $\infty$-category of arrows in $\sS$.  
	\item $p_!$ commutes with $C$, since this last functor is left adjoint to the inclusion $\sS_*\subseteq\Arr(\sS)$. 
	\item By straightening-unstraightening, a functor $f:BO(n)\to\sS$ is equivalent to a Kan fibration $E^f=BO(n)\times_{\sS}\sS_*\to BO(n)$ and $p_!(f)\simeq E^f$. 
	\item Let $F_1$, resp. $F_0$, be the target and source components of $F$, then $E^{F_1}\simeq E_n$ and $E^{F_0}\simeq E_n\setminus\zeta_n$, where $\zeta_n$ is the zero section.
	\item Finally, $p_!\Th_n$ is equivalent to the cofiber of $E_n\setminus\zeta_n\subseteq E_n$, which is equivalent to the one-point compactification of the total space $E_n$. 
\end{enumerate}
\end{example}
\begin{example}
	Let $O$ be the stable orthogonal group. The map $J_n$ induced by $f_n$ from the automorphisms of $
	\bR$ to $\aut(\bS)$ sends $\phi:\bR^n\to\bR^n$ to $\Sigma^{\infty-n}\Th(\phi):\bS\to\bS$. Consider $\bR\oplus-:O(n)\to O(n+1)$, then $\Th(\bR\oplus\phi)=\Th(\bR)\wedge\Th(\phi)=\Sigma\Th(\phi)$, therefore the diagram 
	\begin{center}
		\begin{tikzcd}
			O(n)\arrow[r,"\bR\oplus-"] \arrow[dr,"J_n"'] & O(n+1) \arrow[d,"J_{n+1}"]\\
			& GL_1(\bS)
		\end{tikzcd}
	\end{center}commutes, and the maps $J_n$ induced a group homomorphism $J:O\to GL_1(\bS)$, called \emph{J-homomorphism}. The delooped map $BJ:BO\to\Line_\bS$ is equivalent to the colimit of the maps $f_n$. The Thom spectrum of $BJ$ is denoted $MO$, called the \emph{real bordism spectrum}. 
\end{example}
\begin{example}\label{ex:bordismspectra}
	The decomplexification map $U\to O$ induces a twisting over $X=BU$ by post-composition with $BJ$. The Thom spectrum of $BU\to BO\to\Line_\bS$ is denoted $MU$, called the \emph{complex bordism spectrum}. In general, a group homomorphism $\xi:G\to O$ induces a twisting over $X=BG$ by post-composing with $BJ$. By taking $G=SO,Spin$ or $String$, we obtain the \emph{oriented, spin} and \emph{string bordism spectra}. 
\end{example}
\begin{example}
Recall that we have a diagram characterizing $BString$
\[ K(\bZ,3) \to BString \to BSpin \to BGL_1(\bS) = \Line_{\bS}\] 
then applying the Thom construction to this diagram we get 
\[\Sigma^\infty_+ K(\bZ,3) \to MString \to MSpin .\] 
Here we use the fact that the map $K(\bZ,3) \to \Line_{\bS}$ factors through the point, and hence its Thom spectrum is $\Sigma^\infty_+ K(\bZ,3)$.

As a next step we have a string orientation map $MString \to \mathrm{tmf}$, where $\mathrm{tmf}$ is the spectrum of topological modular forms, following the computation in \cite{ahr2010tmforientation}. Using the fact that $\Sigma^\infty_+$ is monoidal, we know that the units of $K(\bZ,3)$ include $K(\bZ,3)$ itself, meaning we get a map $K(\bZ,3) \to GL_1(\mathrm{tmf})$. Applying $B$ gives us a map $f\colon K(\bZ,4) \to BGL_1(\mathrm{tmf}) = \Line_{\mathrm{tmf}}$, classifying the twist of $\mathrm{tmf}$-cohomology. Further details can be found in \cite{andoblumberggepner2010twistedktmf}.
\end{example}

Let us try a more feasible example.

\begin{example}
	Let 
	\[K(1,\bZ/2\bZ) \to BSpin \to BSO\] 
	be the fiber sequence and let 
	\[K(\bZ,2) \to BSpin^c \to BSO  \to BGL_1\bS = \Line_\bS.\]
	Applying the Thom construction	to the second sequence and again using the fact that $K(\bZ,2) \to \Line_\bS$ factors through the point, we get the sequence 
	\[ \Sigma^\infty_+ K(\bZ,2) \to MSpin^c \to MSO .\]
	Now, using Atiyah--Bott--Shapiro orientation $MSpin^c \to KU$ \cite{atiyahbottshapiro1964clifford}, we get a map $K(\bZ,2) \to GL_1 KU$, which induces a map $K(\bZ,3) \to \Line_{KU}$. This gives us twisted $KU$-cohomology of $K(\bZ,3)$. As $K(\bZ,3)$ classifies bundle gerbes, this is the twist of $KU$-theory by bundle gerbes. 
\end{example}

\section{Twists via Picard Groupoids and Grading}
This section requires some some further details. Up until now we defined everything via $\Line_R$, however for many applications we need to work with $\Pic_R$ instead.

\begin{definition}
	Let $R$ be a ring spectrum. The \emph{Picard $\infty$-groupoid} $\Pic_R$ is the full sub-$\infty$-groupoid of $\Mod_R$ spanned by the invertible $R$-modules, i.e. those $R$-modules $M$ such that there exists an $R$-module $N$ with $M \wedge_R N \simeq R$.
\end{definition}

\begin{remark}
	It is a common fact that the $\infty$-groupoid	$\Pic_R$ is equivalent to $\pi_0(R) \times \Line_R$, meaning that the invertible $R$-modules are given by the $R$-lines together with a shift by an element in $\pi_0(R)$. Here it is important to note that this equivalence does not respect the monoidal structure is only an equivalence of underlying $\infty$-groupoids.
\end{remark}

We can now generalize \cref{def:thom} to $\Pic_R$.

\begin{definition}[{Thom spectrum}] 
	The \emph{Thom $R$-module spectrum} is the functor
	\[
	M \colon \Grpd_{\infty}^{op}/\Pic_R \to \Mod_R,
	\]
	which sends $f\colon X^{op} \to \Pic_R$ to the $R$-module spectrum $\colim(X^{op} \xrightarrow{f} \Pic_R \to \Mod_R)$.
\end{definition}

We can now generalize \cref{def:twistedcohomology}	to $\Pic_R$.

\begin{definition}[{Twisted cohomology}] 
	Let $R$ be a ring spectrum, $X$ be a space and let $p\colon X \to *$ be the projection map, and let $f\colon X^{op} \to \Pic_R$.  The \emph{$f$-twisted $R$-cohomology of $X$} is defined as the mapping spectrum 
	\[ R_f(X)	\coloneqq \Map_{\Mod_R}(Mf, R) \simeq \Map_{\Mod_R(X)}(f,p^*R) \simeq \Map_{\Mod_R(X)}(f,R_x) .\]
	Similarly, we have \emph{$f$-twisted $R$-homology of $X$} defined as
	\[ R^f(X) \coloneqq \Map_{\Mod_R}(R,Mf) \simeq Mf. \]
\end{definition}

Let us see how this additional generality helps us. Let $f \colon X \to \Line_R$, then we get a map $\Sigma^n f \colon \Sigma^n X \to \Pic_R$, which induces a cohomology $R^{* + n}(X)$. Now we have  
\[ R_{\Sigma^nf}(X) \simeq \Map(M\Sigma^nf,R) \simeq \Map(\Sigma^nf,R_x) \simeq \Map(f,\Omega^n R_x)\simeq \Omega^n \Map(f,R_x) \simeq \Omega^n R_f(X) \]
This equivalence can only work via $\Pic_R$. Indeed even though $R$ is in $\Line_R$, $\Sigma^n R$ is not in $\Line_R$. However, it is in $\Pic_R$, with inverse $\Sigma^{-n} R$.

This show that gradings can be recovered via twisted cohomology theories using $\Pic_R$ as twists.

{\footnotesize
 \bibliographystyle{alpha}
 \bibliography{main}
 }
\end{document}