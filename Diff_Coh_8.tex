\documentclass[10pt]{amsart}
\usepackage{amsmath,amsthm,amssymb,amsfonts}
\usepackage[mathscr]{euscript}
\usepackage{tikz}
\usepackage{tikz-cd}
\usepackage{enumerate}
\usepackage{enumitem}
\usepackage{mathtools}
\usepackage[colorlinks=true, linkcolor=red, citecolor = blue]{hyperref}
\usepackage[margin=2.5cm]{geometry}
\setlength{\marginparwidth}{2cm}

\usepackage[nameinlink,capitalise,noabbrev]{cleveref}

\usepackage[textwidth=2cm, textsize=small, colorinlistoftodos]{todonotes}

\newcommand{\A}{\mathscr{A}}
\newcommand{\B}{\mathscr{B}}
\newcommand{\cF}{\mathcal{F}}
\newcommand{\D}{\mathscr{D}}
\newcommand{\sC}{\mathscr{C}}
\newcommand{\sS}{\mathscr{S}}
\newcommand{\bS}{\mathbb{S}}
\newcommand{\bR}{\mathbb{R}}
\newcommand{\bZ}{\mathbb{Z}}
\newcommand{\bC}{\mathbb{C}}
\newcommand{\fO}{\mathbf{O}}
\newcommand{\rO}{\mathscr{O}}
\newcommand{\M}{\mathcal{M}}
\newcommand{\bQ}{\mathbb{Q}}

\newcommand{\aff}{\mathrm{Aff}}
\newcommand{\Cyc}{\mathrm{Cyc}}
\newcommand{\tmf}{\mathrm{tmf}}
\newcommand{\TMF}{\mathrm{TMF}}
\newcommand{\Tmf}{\mathrm{Tmf}}
\newcommand{\Def}{\mathrm{Def}}
\newcommand{\curv}{\mathrm{curv}}
\newcommand{\CS}{\mathrm{CS}}
\newcommand{\ch}{\mathrm{ch}}
\newcommand{\dKU}{\smash{\widehat{ku}}}
\newcommand{\dKUnabla}{\smash{\widehat{ku}^\nabla}}
\renewcommand{\sp}{\mathrm{sp}}

\DeclareMathOperator{\tr}{tr}
\newcommand{\Map}{\mathrm{Map}}
\newcommand{\Hom}{\mathrm{Hom}}
\newcommand{\Ho}{\mathrm{ho}}
\newcommand{\set}{\mathscr{S}\mathrm{et}}
\newcommand{\CMon}{\mathrm{CMon}}
\newcommand{\CGrp}{{\normalfont\texttt{CMon}}}
\newcommand{\fib}{{\normalfont\texttt{fib}}}
\newcommand{\cofib}{{\normalfont\texttt{cofib}}}
\newcommand{\Bun}{{\normalfont\texttt{Bun}}}
\newcommand{\Sp}{\mathscr{S}\mathrm{p}}
\newcommand{\Ch}{\mathscr{C}\mathrm{h}}
\newcommand{\cat}{\mathscr{C}\mathrm{at}}
\newcommand{\scat}{s\mathscr{C}\mathrm{at}}
\newcommand{\sset}{s\mathscr{S}\mathrm{et}}
\newcommand{\Line}{\mathscr{L}\mathrm{ine}}
\newcommand{\Fun}{\mathrm{Fun}}
\newcommand{\Nat}{\mathrm{Nat}}
\newcommand{\colim}{\mathrm{colim}}
\newcommand{\Top}{\mathscr{T}\mathrm{op}}
\newcommand{\Grpd}{\mathscr{G}\mathrm{rpd}}
\newcommand{\Grp}{\mathscr{G}\mathrm{rp}}
\newcommand{\Euc}{\mathscr{E}\mathrm{uc}}
\newcommand{\Mfd}{\mathscr{M}\mathrm{fd}}
\newcommand{\Kan}{\mathscr{K}\mathrm{an}}
\newcommand{\Vect}{\mathscr{V}\mathrm{ect}}
\newcommand{\Mod}{\mathscr{M}\mathrm{od}}
\newcommand{\Proj}{\mathscr{P}\mathrm{roj}}
\newcommand{\Ab}{\mathscr{A}\mathrm{b}}
\newcommand{\Shv}{\mathscr{S}\mathrm{hv}}
\newcommand{\Yon}{\mathscr{Y}\mathrm{on}}
\newcommand{\Open}{\mathscr{O}\mathrm{pen}}
\newcommand{\PSh}{\mathscr{P}\mathscr{S}\mathrm{h}}
\newcommand{\Pic}{\mathscr{P}\mathrm{ic}}
\newcommand{\triv}{\mathscr{T}\mathrm{riv}}
\newcommand{\aut}{\mathrm{Aut}}
\newcommand{\Th}{\mathrm{Th}}
\newcommand{\fr}{\mathrm{Fr}}
\newcommand{\Arr}{\mathrm{Arr}}
\newcommand{\ev}{\mathrm{ev}}
\newcommand{\dis}{\mathrm{dis}}
\newcommand{\Mon}{\mathrm{Mon}}

\newcommand{\bbefamily}{\fontencoding{U}\fontfamily{bbold}\selectfont}
\newcommand{\textbbe}[1]{{\bbefamily #1}}
\DeclareMathAlphabet{\mathbbe}{U}{bbold}{m}{n}

\def\DDelta{{\mathbbe{\Delta}}}
\newcommand{\DD}{\DDelta}



%% N.R. notes
\newcommand{\nrnote}[1]{\todo[color=green!40,linecolor=green!40!black,size=\tiny]{#1}}
\newcommand{\nrmpar}[1]{\todo[noline,color=green!40,linecolor=green!40!black,
  size=\tiny]{#1}}
\newcommand{\nrnoteil}[1]{\ \todo[inline,color=green!40,linecolor=green!40!black,size=\normalsize]{#1}}

\newtheorem{theorem}[equation]{Theorem}
\newtheorem{lemma}[equation]{Lemma}
\newtheorem{proposition}[equation]{Proposition}
\newtheorem{corollary}[equation]{Corollary}
% \newtheorem{statement}[section]{Statement}

\theoremstyle{definition}
\newtheorem{definition}[equation]{Definition}
\newtheorem{example}[equation]{Example}
% \newtheorem{attone}[equation]{Attention}

\theoremstyle{remark}
\newtheorem{remark}[equation]{Remark}
% \newtheorem{intone}[equation]{Intuition}
\newtheorem{notation}[equation]{Notation}
% \newtheorem{queone}[equation]{Question}
% \newtheorem{conjone}[equation]{Conjecture}
\newtheorem{warning}[equation]{Warning}

\numberwithin{equation}{section}

\title{Differential Cohomology Seminar 8}
\date{12.11.2025 \& 26.11.2025}
\author{Talk by Hannes Berkenhagen}

\begin{document}
\maketitle

The aim of this talk is to review twisted cohomology theory with the aim of later discussing twisted differential cohomology theories \cite{bunkegepner2021differential}. For these talks the main source is \cite{abghr2014infty,abg2018thom}.

\section{Twisted Cohomology}
Let $R$ be a ring spectrum, meaning a monoid object in the $\infty$-category of spectra $\Sp$. From this we get a presentable stable $\infty$-category $\Mod_R$ of left $R$-module spectra. Objects therein are morphisms of the form $R \wedge M \to M$ satisfying the usual associativity and unit conditions up to coherent homotopies.

Note we have an adjunction diagram 
\[
\begin{tikzcd}[column sep = 2cm]
\Sp \arrow[r, shift left=1.8, "R \wedge -", "\bot"'] & \Mod_R \arrow[l, shift left=1.8, "{\Hom_R(R,-)}"]
\end{tikzcd}, 
\]
where the right adjoint is in fact the forgetful functor. This in particular means $\Mod_R$  has a distinguished object $R$ i.e. the free $R$-module of rank $1$. We now refine these constructions. 

\begin{definition}
	Let $R$ be a ring spectrum. An \emph{$R$-line} is an $R$-module $L$ such that $L \simeq R$.
\end{definition}
\begin{definition}
	Let $\Line_R$ be the full sub-$\infty$-groupoid of $\Mod_R$ spanned by the $R$-lines.
\end{definition}
By construction, $\Line_R$ is equivalent to the category with a single object $R$ and hom-space $GL_1R$, the $\infty$-group of $R$-linear automorphisms of $R$. Notice that $GL_1(R)\subseteq\Hom_{\Mod_R}(R,R)\simeq\Hom(\bS,R)=\Omega^\infty R$. 
\begin{lemma}\label{lem:units}
	$GL_1(R)$ fits into the pullback square
	\begin{center}
		\begin{tikzcd}
			GL_1(R)\arrow[r]\arrow[d] & \Omega^\infty R\arrow[d]\\
			\pi_0(R)^\times \arrow[r] & \pi_0(R)
		\end{tikzcd}
	\end{center}In particular, the inclusion $GL_1(R)\to\Omega^\infty R$ induces an isomorphism on $n$-homotopy groups, for all $n\geq1$. 
\end{lemma}
\begin{proof}
	$\pi_0(R)\simeq\Hom_{\Ho\Mod_R}(R,R)$, where $\Ho\Mod_R$ is the homotopy category of $R$-modules, the right-vertical arrow corresponds to $\Hom_{\Mod_R}(R,R)\to\Hom_{\Ho\Mod_R}(R,R)$ mapping a morphism to its homotopy class, and $\pi_0(R)^\times\subseteq\Hom_{\Ho\Mod_R}(R,R)$ is the set of isomorphisms. Finally, a morphism in a $\infty$-category $\sC$ is an equivalence if and only if its homotopy class is an isomorphism in $\Ho\sC$. 
\end{proof}
\begin{definition}
	Let $X$ be a space. Denote by $\Mod_R(X)$ the $\infty$-category of $R$-module spectra parametrized over $X$, i.e. the functor category $\Fun(X^{op},\Mod_R)$.
\end{definition}

\begin{definition}
	Let $X$ be a space. Denote by $\Line_R(X)$ the $\infty$-groupoid of $R$-line spectra parametrized over $X$, i.e. the $\Fun(X^{op},\Line_R)$. 
\end{definition}
Since $\Line_R\simeq BGL_1(R)$, functors $X^{op}\to\Line_R$ are generalization of local systems. 
\begin{example}
	Let $R_X\colon X^{op} \to*\to \Line_R$ be the constant functor with value $R$. 
\end{example}

We now proceed to the Thom construction.

\begin{definition}[{Thom spectrum}] \label{def:thom}
	The \emph{Thom $R$-module spectrum} is the functor
	\[
	M \colon \Grpd_{\infty}^{op}/\Line_R \to \Mod_R,
	\]
	which sends $f\colon X^{op} \to \Line_R$ to the $R$-module spectrum $\colim(X^{op} \xrightarrow{f} \Line_R \xrightarrow{i} \Mod_R)$.
\end{definition}
\begin{remark}\label{rmk:extthomspc}
	Notice that the definition of Thom $R$-module spectrum make sense for any functor $X^{op}\to R\Mod$. 
\end{remark}
Let us note an alternative characterization that will be important later.
\begin{remark}
	For a given map $f\colon X \to Y$, we get a map $f^*\colon \Mod_R(Y) \to \Mod_R(X)$ by precomposition with $f^{op}$. Since $f^*$ preserves both limits and colimits, we construct a left adjoint $f_! \colon \Mod_R(X) \to \Mod_R(Y)$ and a right adjoint $f_* \colon \Mod_R(X) \to \Mod_R(Y)$, using left and right Kan extension along $f^{op}$. Let $f=p:X\to*$ be the terminal functor, then left Kan extension along $p^{op}$ is exactly taking colimit, therefore
	\[Mf \simeq p_!(i \circ f).\]
\end{remark}

\begin{remark}[{\cite[3.6]{andoblumberggepner2010twistedktmf}}]
	Let $\triv_R$ be the slice groupoid $\Line_R/R$, together with the canonical projection $\pi:\triv_R\to\Line_R$, then: \begin{enumerate}
		\item $\triv_R$ is a slice $\infty$-groupoid, hence contractible, and $\pi$ is a Kan fibration. 
		\item $GL_1(R)$ is equivalent to the fiber of $\pi$ over $R\in\Line_R$ and acts freely on the fibers of $\pi$. 
\end{enumerate}These observations imply $\Line_R$ is the classifying space for $GL_1(R)$-bundles. Here we use the term $GL_1(R)$-bundle to mean a parametrized family of $GL_1(R)$-spaces with a free and transitive action. 
\end{remark}
We now proceed to define twisted cohomology theories.

\begin{definition}[{Twisted cohomology}] \label{def:twistedcohomology}
	Let $R$ be a ring spectrum, $X$ be a space, $p\colon X \to *$ the terminal functor, and $f\colon X^{op} \to \Line_R$ a $R$-line bundle over $X$.  The \emph{$f$-twisted $R$-cohomology of $X$} is defined as the mapping spectrum 
	\[ R_f(X)	\coloneqq \Map_{\Mod_R}(Mf, R) \simeq \Map_{\Mod_R(X)}(f,p^*R) \simeq \Map_{\Mod_R(X)}(f,R_X) .\]
	Similarly, the \emph{$f$-twisted $R$-homology of $X$} is defined as
	\[ R^f(X) \coloneqq \Map_{\Mod_R}(R,Mf) \simeq Mf. \]
	The $f$-twisted $R$-cohomology groups of $X$ are defined as the homotopy groups of $R_f(X)$, i.e.
	\[ R^n_f(X) \coloneqq \pi_0(\Map_{\Mod_R}(Mf, \Sigma^n R)) \cong \pi_{-n}(\Map_{\Mod_R(X)}(f,R_X)). \]Similarly, the $f$-twisted $R$-homology groups of $X$ are defined as the homotopy groups of $R^f(X)$, i.e. 
	\[ R_n^f(X) \coloneqq \pi_0(\Map_{\Mod_R}(\Sigma^n R, Mf)) \cong \pi_n(Mf). \]
\end{definition}
\begin{example}[Trivial twist] If $f:X\to \Line_R$ factors through $*$, then $f$ factors as the $X^{op}\to*\to\sS$, the constant factor with value $*$, and $R\wedge\Sigma^\infty_+(-):\sS\to\Mod_R$. The latter functor commutes with colimits, being a left adjoint, while the colimit of the latter is $X$ itself, then $Mf\simeq R\wedge\Sigma^\infty_+X$. In particular, $f$-twisted $R$-cohomology and $R$-homology of $X$ reduce to ordinary (untwisted) $R$-cohomology and $R$-homology of $X$. 
\end{example}
\begin{definition}
	Given a vector bundle $\pi:E\to B$, define the \emph{Thom space} of $\pi$, denoted $\Th(E)$, to be the homotopy cofiber of $E_0\subseteq E$, where $E_0$ is the complement of the zero section. 
\end{definition}
\begin{lemma}
	Consider a space $X$ and the $\infty$-categorical Yoneda's embedding $y:X\to\Fun(X^{op},\sS)$. The colimit of $y$ is the terminal pre-sheaf on $X$, i.e. the pre-sheaf with constant value the one-point space. 
\end{lemma}
\begin{proof}
	Let $S$ be a pre-sheaf on $X$, consider then the slice category $X_{/S}$ of pairs $(x,\phi)$, where $x$ is an object of $X$ and $\phi:y(x)\to S$. The density theorem for $\infty$-categories states that $S$ is equivalent to the colimit of $X_{/S}\to X\xrightarrow{y}\Fun(X^{op},\sS)$, the first map being the canonical projection. Take $S=*$, then $X_{/*}\to X$ is an equivalence, hence the claim. 
\end{proof}
Let $G$ be a topological group and $BG$ the $\infty$-groupoid with a single object $1$ and hom-space $G$. The category $\sS_G:=\Fun(BG,\sS)$ is equivalent to the category of $G$-spaces.
\begin{lemma}
	Consider $X=BG$, a $G$-space $f:X\to\sS$ and its left Kan extension $f_!:\sS_G\to\sS$, then $f_!\simeq(-\times E)/G$, where $E=f(1)$. 
\end{lemma}
\begin{proof}
	Evaluate at $1$, then $f_!(y(1))=E$, by definition, and $y(1)\simeq G$, as $G$-spaces, hence $(y(1)\times E)/G\simeq(G\times E)/G\simeq E$. Since $f_!$ and $(-\times E)/G$ agree on representables and are colimit-preserving, they are equivalent. 
\end{proof}
\begin{example}Take the space $BO(n)$ and $f_n:BO(n)\to\sS_*$ the $n$-sphere $S^n$ with $O(n)$-action coming from the one-point compactification of the regular action on $\bR^n$. Let $\alpha_n=\Sigma^{\infty-n}f_n:BO(n)\to\Sp$, then $\alpha_n(1)=\Sigma^{\infty-n}f_n(1)=\Sigma^{\infty-n}S^n\simeq\bS$, so $\alpha_n$ factors through $\Line_\bS$. Let $X=BO(n)^{op}$ and $p:X\to*$ the terminal functor, then \[M\alpha_n=p_!\Sigma^{\infty-n}f_n\simeq\Sigma^{\infty-n}p_!(f_n)_!y\simeq\Sigma^{\infty-n}(f_n)_!\underbrace{p_!(y)}_{\simeq*}\simeq\Sigma^{\infty-n}(*\times S^n)/O(n)\]
Let $P=EO(n)$ be the universal $O(n)$-bundle and $M$ a $O(n)$-space, then $*\times_{O(n)} M$ is modelled by the \emph{strict} quotient $(P\times M)/O(n)$, then \[S^n/O(n)=\mathrm{cofib}(\bR^n_0\subseteq\bR^n)/O(n)\simeq\mathrm{cofib}(\underbrace{*\times_{O(n)}\bR^n_0}_{\simeq E^n_0}\subseteq\underbrace{ *\times_{O(n)}\bR^n}_{\simeq E^n})=\Th(E^n) \]
where $E^n=P\times_{O(n)}\bR^n\to BO(n)$ is the universal $n$-dimensional vector bundle, hence $M\alpha_n\simeq\Sigma^{\infty-n}\Th(E^n)$. 
\end{example}
The functor $BO(n)\to\Line_\bS$ induces a $\infty$-group homomorphism $j_n:O(n)\to GL_1(\bS)$, mapping $\phi$ to $\Sigma^{\infty-n}\Th(\phi)$. Consider the suspension morphism $s_n=\bR\oplus-:O(n)\to O(1+n)$, then $$j_n(\bR\oplus\phi)=\Sigma^{\infty-n-1}\Th(\bR\oplus\phi)\simeq\Sigma^{\infty-n-1}\underbrace{\Th(\bR)}_{\simeq S^1}\wedge\Th(\phi)\simeq\Sigma^{\infty-n}\Th(\phi)=j_n$$Recall that the colimit over the suspension morphisms $s_n$ is the stable orthogonal group $O$.
\begin{definition}
	Denote by $j$ the induced group homomorphism $O\to GL_1(\bS)$, called the \emph{$J$-homomorphism}.  
\end{definition} 
\begin{example}
	Let $X=O^{op}$ and take $Bj:BO\to\Line_\bS$, then $Mj$ is denoted $MO$ and called the \emph{real bordism spectrum}. 
\end{example}
Denote by $M$ the extended Thom spectrum functor $\Grp^{op}_\infty/\Mod_R\to\Mod_R$, this is a left adjoint to the functor $\rO$ sending a $R$-module to the functor $*\to\Mod_R$ picking out $M$. In particular, $M$ preserves colimits and  $Bj\simeq\colim_nBj_n$, therefore we have the following:
\begin{theorem}
	$MO\simeq\colim_nMO(n)=\colim_n\Sigma^{\infty-n}\Th(E^n)$. 
\end{theorem}
\begin{example}\label{ex:bordismspectra}
	A group homomorphism $\xi:G\to O$ induces a functor $f:BG\to\Line_\bS$. The Thom spectrum $Mf$ is denoted $MG$ or $M\xi$, and called \emph{$G$-bordism spectrum}. For $G=U,SO,Spin$, and $String$, we obtain the \emph{complex, oriented, spin}, and \emph{string bordism spectra}. 
\end{example}
\begin{remark}
	In \cref{ex:bordismspectra} we might take $G=\{*\}$, the one-point group, then $MG\simeq\bS$, which, if it didn't have a name, might be called \emph{framed bordism spectrum}, following the naming convention in \cref{ex:bordismspectra} and in line with the theorem that $\pi_*(\bS)\simeq\Omega^\mathrm{fr}_*$, the bordism ring of framed (trivialized tangent bundle) smooth manifolds.  
\end{remark}
Let $R$ be a ring in sets, then $R$ is a $A_\infty$-ring spectrum (actually, $E_\infty$), $\Omega^\infty R$ is equivalent to $R$ with discrete topology ($\pi_0(\Omega^\infty R)\simeq R$, as sets, and every other homotopy group vanish). In particular, $GL_1(R)$ is simply $R^\times$ with discrete topology. Consider then the fiber sequence $SO\to O\to \bZ^\times\simeq GL_1(\bZ)$.
\begin{example}
	$X=BO^{op}$ and $\alpha=w_1:BO\to\Line_\bZ$, the 1st Stiefel-Whitney class (delooping of the determinant $O\to GL_1(\bZ)$), then $Mw_1$ is a $\bZ$-module spectra. Let $i:SO\subseteq O$, then $w_1i$ factors through the point, so $M(w_1i)\simeq\bZ\wedge\Sigma^\infty_+SO$. 
\end{example}
Given $f:X^{op}\to\Line_R$ and a sequence $F\xrightarrow{i} X\xrightarrow{\pi} Y$, there is an induced sequence of Thom $R$-module spectra $MF\to MX\to MY$. If $\pi i$ factors through the point, $MF\simeq R\wedge\Sigma^\infty_+F$. 
\begin{lemma}\label{lem:adj}
	Let $R$ be a ring spectrum and $X$ a connected monoidal $\infty$-groupoid, then \[\Hom_{\Mon(\Sp)}(\Sigma^{\infty}_+X,R)\simeq\Hom_{\Mon(\sS)}(X,GL_1(R))\] 
\end{lemma}
\begin{proof}
	Since $X$ is connected, the space of homomorphisms $X\to GL_1(R)$ is equivalent to the space of homomorphisms $X\to\Omega^\infty R$, then use that $(\Sigma^\infty,\Omega^\infty)$ is a monoidal adjunction (The monoidal structure on spectra is such that $\Sigma^\infty$ is strong monoidal). 
\end{proof}
\begin{remark}
	Notice that we can weaken the result. Namely, if $X$ is 1-connected (pointed and connected), then the space of functors (of $\infty$-groupoids) $X\to GL_1(R)$ is equivalent to the space of functors $X\to\Omega^\infty R$ such that $*\to X\to\Omega^\infty R$ is an equivalence. This last space is equivalent, via the $(\Sigma^\infty_+,\Omega^\infty)$ adjunction, to the space of morphisms of spectra $\Sigma^\infty_+X\to R$, such that $\bS\to\Sigma^\infty_+X\to R$ represents a unit in $\pi_0(R)$.
\end{remark}
\begin{remark}\label{rmk:strengthen}
	Notice that we can also strengthen the result. Namely, if $X$ is a connected, commutative monoid object and $R$ is a commutative ring spectrum, then $\Omega^\infty R$ and $GL_1(R)$ are also commutative monoid objects. Using the same argument, together with the fact that $(\Sigma^\infty,\Omega^\infty)$ is actually a \emph{symmetric} monoidal adjunction, we conclude that \[\Hom_{\CMon(\Sp)}(\Sigma^\infty_+X,R)\simeq\Hom_{\CMon(\sS)}(X,GL_1(R))\] 
\end{remark}
\begin{remark}\label{rmk:moncat}
	Let $\D$ be a monoidal $\infty$-category. Consider $\cat_\infty/\D$, the $\infty$-category of functors into $\D$, with monoidal structure given by \begin{center}
		\begin{tikzcd}
			(F:\A\to\D,G:\B\to\D) \arrow[r,|->] & (\A\times\B\xrightarrow{F\times G}\D\times\D\xrightarrow{\otimes}\D)
		\end{tikzcd}
	\end{center} The monoidal unit is the functor $*\to\D$ picking out the monoidal unit of $\D$. If $\D$ is symmetric monoidal, then so is $\cat_\infty/\D$. A (commutative) monoid object in $\cat_\infty/\D$ is given by a (symmetric) monoidal category $\sC$ and a (symmetric) monoidal functor $F:\sC\to\D$. 
\end{remark}
In view of \cref{rmk:moncat}, let $R$ be commutative ring spectrum, then $\Mod_R$ is a symmetric monoidal $\infty$-category and $\Line_R$ is a symmetric monoidal $\infty$-groupoid. The category $\Grpd^{op}_\infty/\Line_R$ is then symmetric monoidal and (commutative) monoid objects are given by (symmetric) monoidal $\infty$-groupoids $X^{op}$ a (symmetric) monoidal functors $X^{op}\to\Line_R$. One can then check that $M$ is a symmetric monoidal functor, so that (commutative) monoid objects are sent to (commutative) monoid objects in $\Mod_R$, i.e. (commutative) $R$-algebras. 
\begin{example}\label{ex:tmf}
	Let $\Tmf$ be the commutative ring spectrum of topological modular forms (see \cref{rmk:names}) and $\sigma:MString\to\tmf$ the $String$-orientation of $\tmf$. In the sequence \begin{center}
		\begin{tikzcd}
			BString \arrow[r] & BO \arrow[r,"Bj"] & \Line_\bS
		\end{tikzcd}
	\end{center}all functors are symmetric monoidal, so that $MString$ is a commutative $\bS$-algebra, i.e. a commutative ring. The $String$-orientation of $\tmf$ is also a commutative ring homomorphism. In the fiber sequence $K(\bZ,3)\to BString\to BSO$, the fiber map $i:K(\bZ,3)\to BString$ is also a symmetric monoidal, so the composition 
	\begin{center}
		\begin{tikzcd}
			\Sigma^\infty_+K(\bZ,3)\arrow[r,"Mi"] & MString \arrow[r,"\sigma"] & \tmf
		\end{tikzcd}
	\end{center}is a commutative ring homomorphism. Using \cref{lem:adj}, we conclude that the induced homomorphism $K(\bZ,3)\to\Omega^\infty\tmf$ (which is a homomorphism of commutative monoid objects, given \cref{rmk:strengthen}) factors through $GL_1(\tmf)$, and so it induces a (symmetric monoidal) functor $K(\bZ,4)\to\Line_\tmf$, i.e. \emph{2-bundle gerbes twist $\tmf$}.
\end{example}
\begin{remark}\label{rmk:names}The spectrum of topological modular forms comes in three main flavors, namely:
\begin{enumerate}
	\item $\TMF$, i.e. the global sections of the spectral structure sheaf $\rO^{top}:(\aff/\M_{ell})^{op}\to\CMon(\Sp)$ on the (\' etale site of the) moduli stack of elliptic curves.
	\item $\Tmf$, i.e. the global sections of the spectral structure sheaf $\bar{\rO}^{top}:(\aff/\bar{\M}_{ell})^{op}\to\CMon(\Sp)$ on the (\' etale site of the) \emph{compactified} moduli stack of elliptic curves. The inclusion $\M_{ell}\hookrightarrow\bar{\M}_{ell}$ induces a commutative ring homomorphism $\Tmf\to\TMF$.
	\item $\tmf$, i.e. the connective cover of $\Tmf$. By definition, there is a commutative ring homomorphism $\tmf\to\Tmf$. 
\end{enumerate}In \cite{ando2010multiplicative}, $\tmf$ is used to denote our $\TMF$, in \cite{goerss2009topologicalmodularformsaftern}, $\tmf$ is used to denoted our $\Tmf$, and \cite{douglas2014topological} has the same notation as us. In \cref{ex:tmf}, we use $\tmf$ to mean the connective cover of $\Tmf$. 
\end{remark}
Let us go down one step in the chromatic ladder. 
\begin{example}\label{ex:ktheory}
	Recall the fiber sequences for $Spin$ and $Spin^c$: \[\bZ_2\to Spin\to SO,\qquad S^1\to Spin^c\to SO\] All the spaces involved are commutative groups. Applying the Thom spectrum functor to the delooped sequences, we get\[\Sigma^\infty_+K(\bZ_2,1)\to MSpin\to MSO,\qquad\Sigma^\infty_+K(\bZ,2)\to MSpin^c\to MSO \]
	Let $\sigma:MSpin\to KO$ and $\sigma^c:MSpin^c\to KU$ be the Atiyah-Bott-Shapiro orientation of real and complex $K$-theory (see \cite{atiyahbottshapiro1964clifford}).
	Similar to \cref{ex:tmf}, we get homomorphisms 
	\begin{center}
		\begin{tikzcd}
			\Sigma^\infty_+K(\bZ_2,1)\arrow[r] & MSpin \arrow[r,"\sigma"] & KO, & \Sigma^\infty_+K(\bZ,2)\arrow[r] & MSpin^c \arrow[r,"\sigma^c"] & KU
		\end{tikzcd}
	\end{center}
	Using \cref{lem:adj} again and delooping, we obtain functors $K(\bZ_2,2)\to\Line_{KO}$ and $K(\bZ,3)\to \Line_{KU}$, i.e. \emph{real, resp. complex, bundle gerbes twist real, resp. complex, $K$-theory}. 
\end{example}

\section{Twists via Picard Groupoids and Grading}
This section requires some some further details. Up until now we defined everything via $\Line_R$, however for many applications we need to work with $\Pic_R$ instead.
\begin{definition}
	Given a monoidal $\infty$-category $(\sC,\otimes,1)$, an object $M$ is \emph{invertible} if there is an object $D$ such that $D\otimes M\simeq M\otimes D\simeq 1$. The \emph{Picard $\infty$-groupoid} of $\sC$ is the sub-$\infty$-groupoid generated by invertible modules. 
\end{definition}
\begin{definition}
If $R$ is a ring spectrum, $\Mod_R$ is monoidal. Denote by $\Pic_R$ the Picard groupoid of $\Mod_R$. 	
\end{definition}
\begin{remark}
	$\Pic_R$ splits as the disjoint union of $\pi_0(\Pic_R)$-many sub-groupoids. Moreover, if $M\simeq N$, then $R\simeq M^{-1}\otimes N$, so $M^{-1}\otimes N$ is a $R$-line. In particular, every connected component of $\Pic_R$ is equivalent to $\Line_R$, so $\Pic_R\simeq\pi_0(\Pic_R)\times\Line_R$. However, this is not a monoidal equivalence for general ring spectra.  
\end{remark}
\begin{remark}
	$\Sigma^nR$ is invertible, with inverse $\Sigma^{-n}R$. In particular, there is a map $\bZ\times\Line_R\to\Pic_R$. However, this map need not be neither injective (if $R$ is $n$-periodic), nor surjective (see \cite{hill2011tmf}). 
\end{remark}
As mentioned in \cref{rmk:extthomspc}, the Thom spectrum functor makes sense for every functor $f:X^{op}\to\Mod_R$. However, all examples of twists encountered so far came from functors into $\Line_R$. An example of twist that is not the result of a $R$-line bundle is the \emph{degree shift}. 
\begin{definition}
	Denote by $M$ the \emph{Thom $R$-module spectrum} functor \[\Grpd^{op}_\infty/\Mod_R\to\Mod_R\]sending a functor $f:X^{op}\to\Mod_R$ to its colimit.  
\end{definition}
\begin{example}
	Let $f:X^{op}\to\Line_R$ be a twist. Denote by $\Sigma^nf$ the composition of $f$ with the shift functor $\Sigma^n:\Line_R\to\Pic_R$. Since $\Sigma^n$ is an equivalence, it commutes with colimits, so $$M\Sigma^nf\simeq \Sigma^nMf$$If $f=R_X$, then $M\Sigma^nf\simeq \Sigma^nR\wedge\Sigma^\infty_+X$, so  $\Sigma^nf$-twisted $R$-cohomology and $R$-homology correspond to normal $R$-cohomology and $R$-homology with a degree shift by $n$. 
\end{example}

{\footnotesize
 \bibliographystyle{alpha}
 \bibliography{main}
 }
\end{document}